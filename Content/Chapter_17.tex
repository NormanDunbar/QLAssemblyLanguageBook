\chapter{Much Ado About Previous Chapters}

\section{Introduction}
\label{ch17-intro}%hyperlabel{ch17-intro}%

George Gwilt, my faithful reader, has brought me to task on my last
    two articles. Part 15 where I wrote (ok, updated a very old 1991 utility
    which I had written) and again after Part 16 where I delved into the
    Arithmetic Package in QDOSMSQ.

I shall attempt to answer Georges concerns in this chapter.

\section{Chapter 15 -{} Dataspace Utility Problems}
\label{ch17-dataspace}%hyperlabel{ch17-dataspace}%

George makes a number of interesting points about this article and
    all I can say is, `he is absolutely correct'.

As for my small routine to convert an ASCII string into a number in
    a long word, George asks why it is not itself a sub-{}routine when I make
    such a `fuss' (my word) of reusable code.

I can only plead guilty as charged and state, for the record, that
    this is the only time I've ever written anything in assembly language
    which required me to do that conversion. To that end, and nothing else,
    the code was in-{}lined in 1991 and remained so in 2006.

However, I'm sure a general purpose ASCII-{}>{}Long could be easily
    written as a subroutine. I'm certain that there is one lurking somewhere
    inside QDOSMSQ which correctly (I hope) handles invalid characters,
    errors, overflow and so on.

I shall be creating just such a beast in the next chapter.

I feel rather unable to comment on George's own conversion routine -{}
    I never did very well at maths at school and I'm not sure exactly how
    George's code works (yet!).

\section{Chapter 16 -{} Artithmetic Package Problems}
\label{ch17-arithmetic}%hyperlabel{ch17-arithmetic}%

Shortly after that article appeared, George contacted me with a
    whole host of problems. I shall attempt to answer George's concerns below, although George
    and I have conversed in an email exchange on this subject, I think it is
    proper to publicise the results especially as they concern my previous
    chapter. Corrections are due!

{\bf George} 
... I have only one comment on
    ROOT\program{ROOT}. It is that, as explained by Dickens (QL Advanced User Guide), you do
    not need to test D0 for errors after a call to a vector since the
    condition codes are set on return. Indeed Norman does not make the test
    after calling the vector \vector{BP\_INIT} near the top of page 21. The two later
    tests on that page can thus be deleted.



{\bf Norman} 
I agree, however, it has been my
    observation in the past that only sometimes are the condition codes
    actually set on exit from QDOSMSQ. To this end I tend to always test D0 on
    exit from a QDOSMSQ call -{} just to be safe. This does mean that where I
    neglected to do this after the call to \vector{BP\_INIT} (on page 21) is where \emph{my} error was. I should have had a test there.


George points out that I don't need the two tests later on that same
    page. While technically correct, I would be inclined to leave them present
    and add in the one I missed out rather than removing the latter two. I
    like to make sure that the condition codes are correctly set by testing
    them explicitly as this saves me trying to remember which calls do set
    them and which calls don't.

{\bf George} 
ANYROOT\program{ANYROOT} is more interesting as
    it uses \vector{RI\_EXECB} to perform a string of operations. Once again the testing
    of D0 in do\_it and do\_more is not really needed. Also, I should point out
    that the three lines of op\_codes should not be between got\_ok and do\_it
    otherwise do\_it will never be done.



{\bf Norman} 
This is absolutely correct. I
    have no idea what went on here, but the code should be as follows\footnote{And, if you read it again, it actually is now.}:



\begin{lstlisting}[firstnumber=53,caption={Corrections to ANYROOT Code in Previous chapter}]
got_ok      cmpi.w  #2,d3          ; One parameter?
            bne.s   bad_param      ; Oops!
            bra.s   do_it          ; skip over the op-codes.
\end{lstlisting}

With a short branch over the op-{}codes added. I suspect that I have
    inadvertently fixed the code while running under QPC but forgotten to save
    the corrected version back to one of my DOS\_ drives prior to importing the
    code into the article. Quite honestly, the original code without the
    branch would most likely have hung the system.

I have checked my source code system and found that the same
    `broken' version is present there too, so it does look like I forgot to
    save a change back to DOS. `Mea Culpa' as they say.

{\bf George} 
Also I wonder how Norman expects
    it to work given that the address in A3, set in do\_more, is not relative
    to A6 as he suggests is necessary in his definition of \vector{RI\_EXECB} on page
    26. (But see later.)



{\bf Norman} 
I'm afraid that this was a `copy and paste' error. I copied the A1 line above it and pasted it in. While I
    remembered to change the A1 to an A3, I neglected to remove the part about
    it being relative to A6. That is incorrect as A3.L is the pointer to the
    string of bytes and is not relative to A6 at all.



{\bf George} 
It is annoying that immediately
    after the first operation the operands are in the wrong order on the
    stack. Norman has produced swap\_tos to switch the order. The code works
    well, but, since I started my programming life on machines with limited
    space and slow speed, I always try to compress and speed up any program. I
    might suggest here that you eliminate the two occurrences of \lstinline{exg d6,d7}
    \footnote{Consider them eliminated.}and instead swap the d7 and d6 in the following two lines in both
    cases.



{\bf Norman} 
I started on a ZX-{}81 with 1KB of RAM and I'm mostly self-{}taught -{} hence all the errors!



George suggests changing my code at SWAP\_TOS from this:

\begin{lstlisting}[firstnumber=93,caption={ANYROOT - Swap\_Tos - Original Code}]
swap_tos    move.l  0(a6,a1.l),d7   ; Get a long word
            move.l  6(a6,a1.l),d6   ; And another
            exg     d6,d7           ; Swap them around
            move.l  d7,0(a6,a1.l)   ; Store
            move.l  d6,6(a6,a1.l)   ; Store
            move.w  4(a6,a1.l),d7
            move.w  10(a6,a1.l),d6
            exg     d6,d7           ; Swap again
            move.w  d7,4(a6,a1.l)
            move.w  d6,10(a6,a1.l)  ; Now we have N and LN(M) swapped
\end{lstlisting}

to the following to save a couple of instructions and hence,
    valuable time and space:

\begin{lstlisting}[firstnumber=93,caption={ANYROOT - Swap\_Tos - Original Code}]
swap_tos    move.l  0(a6,a1.l),d7   ; Get a long word
            move.l  6(a6,a1.l),d6   ; And another
            move.l  d6,0(a6,a1.l)   ; Store
            move.l  d7,6(a6,a1.l)   ; Store
            move.w  4(a6,a1.l),d7
            move.w  10(a6,a1.l),d6
            move.w  d6,4(a6,a1.l)
            move.w  d7,10(a6,a1.l)  ; Now we have N and LN(M) swapped
\end{lstlisting}

Once again, George is correct -{} I must have run out of caffeine at
    that point. The two \opcode{EXG} instructions are completely unnecessary when
    written as above.

{\bf George} 
I have, however, a more radical
    suggestion. It is that you eliminate both do\_it and swap\_tos and increase
    the size of op\_codes so that the whole procedure is carried out with just
    one set of operations using \vector{RI\_EXECB}. The easy way of doing this is
    possible if you have SMSQ or Minerva both of which have additional
    operations one of which swaps TOS and NOS. The code is \$17.



{\bf Norman} 
I try to keep things as close to
    the original QL as possible so this option may not have been available to
    some of my other readers. I was, however, completely unaware of it until
    George sent me his email. I am completely surprised in finding myself to
    be the first person since QDOS was originally written to need a
    SWAP\_TOS\_NOS routine :o)



When I wrote the code originally, I was almost certain that I could
    do it an one single \vector{RI\_EXECB} operation. That was when I discovered the
    need for a swap operation and hence the break up into a single RI\_EXEC,
    manual swap and the \vector{RI\_EXECB} call. Not as elegant as I would have
    liked.

{\bf George} 
The second method is to go
    through the business of copying TOS and NOS somewhere and then returning
    them to NOS and TOS. The codes for this are in the group referred to by
    Norman as \$FF31 to \$FFFF. The place I would use for temporary storage is
    the Basic buffer. First, op\_codes would become:



\begin{lstlisting}[firstnumber=60,caption={ANYROOT - Swap\_Tos - Suggested Op Codes}]
op_codes dc.b      RI_LN,-5,-11,-6,-12,RI_DIV,RI_EXP,RI_END
\end{lstlisting}

do\_it and swap\_tos would be deleted and do\_more would have added to
    it:

\begin{lstlisting}[firstnumber=1,]
        movea.l   bv_bfbas(a6),a4
        lea       12(a4),a4        ; point to the end of 12 bytes
\end{lstlisting}

If you really think that there may not be as much as 12 bytes
    available in the Basic buffer you can add:

\begin{lstlisting}[firstnumber=1,]
        cmpa.l    bv_tkbas(a6),a4  ; A4 beyond end of buffer?
        bhi       bufful           ; oops
\end{lstlisting}

bv\_bfbas is 0 and bv\_tkbas is 8.

The codes -{}5 and -{}6 save and load 6 bytes to and from -{}6(A6,A4.L)
    and the codes -{}11 and -{}12 do the same with address -{}12(A6,A4.L).

{\bf Norman} 
Again, when I wrote the article,
    I was having a few difficulties with the save and load op-{}codes and was
    enjoying much discussion on the QL-{}USERS email list at the time. I avoided
    them until I could better understand them. As it turned out, the
    explanations simply confused the matter (for me) and I decided to leave
    them alone and simply document them at the end of my article.



Interestingly, George raises something that I was always confused by
    when Simon N Goodwin was writing in the old QL World assembly language
    series, using the Basic Buffer as a workspace. I'm sure that there could
    be a couple of pages for an article here on this very subject -{} if someone
    who knows it was prepared to write one (hint hint). :o)

{\bf George} 
... However, I do have severe
    objections to the later part of the article. This mainly relates to the op
    codes, especially those used for loading and saving numbers onto and from
    the stack. The table at the top of page 26 lists the op codes. If used in
    \vector{RI\_EXECB} these must fit into bytes. It is thus plain wrong to list the
    codes other than those from 0 to \$30 as \$FF31-{}\$FFFF. But why should Norman
    do this?



{\bf Norman} 
To answer the last question, I
    did it because I was advised by Marcel Kilgus, in an email, that \emph{I
    differed from the documentation} and that the load and save op-{}codes were
    indeed negative words and not bytes. And indeed, I quote:



" ... the opcodes \$FF31 to \$FFFF are for load/save, not \$32 to \$FF.
    Yes, the latter DO work, but it seems that's more an undocumented
    side-{}effect."

I took the advice of the man who wrote QPC and probably has
    forgotten more about Assembly Language programming that I have ever
    known!

{\bf George} 
To find out I examined the
    definitions of \vector{RI\_EXEC}/\vector{RI\_EXECB} in five different publications:

\begin{table}[htbp]
\centering
\begin{tabular}{l l l}
\toprule
\textbf{Title} & \textbf{Author} & \textbf{Errors} \\
\midrule
%
QL Technical Guide          & David Karlin \& Tony Tebby & A and B \\
QL Advanced User Guide      & Adrian Dickens & C \\
The Sinclair QDOS Companion & Andrew Pennell & None \\
QDOS Reference Manual       & Jochen Merz    & A \\
%
\bottomrule
\end{tabular}
\caption{QDOS Documentation and RI\_EXEC/RI\_EXECB Errors}
\label{tab:RIEXECErrors}
\end{table}


{\bf Norman} 

George has given a pretty thorough explanation of the differences
    between the above 5 sets of documentation and the code in a JS ROM and
    SMSQ/E -{} I can only state that I wish I had stuck with Pennell rather than
    trying to find out more! See table~\ref{tab:RIEXECErrors}.

I now skip directly to George's closing points.

{\bf George} 
... Having explained how the
    operations codes are used by \vector{RI\_EXEC} and \vector{RI\_EXECB} I return to Norman's
    article on page 26\footnote{In the original QL Today publication.}.



1. In the description of the OpCodes he lists \$FF31 to \$FFFF as the
    `save' and `load' bytes. Clearly these should be \$31 to \$FF (or 49 to
    255).

{\bf Norman} 
This is correct. Obviously, had
    I been paying more attention in class, I would have questioned Marcel's
 \emph{word} information as the \vector{RI\_EXECB} call executes a string of \emph{bytes}. Setting
    a \emph{word} in amongst the bytes would have resulted in an \$FF op-{}code being
    carried out followed by a separate and incorrect byte code, rather than a
    store or load operation.



Basically, when I say `a negative word' I really mean `a negative
    byte'.

{\bf George} 
2. In the definition of \vector{RI\_EXEC}
    it is bits 8 to 15 of D0 which should be zero and not the high word, which
    can in fact be anything.



{\bf Norman} 
Correct. \vector{RI\_EXEC} expects a byte sized op-{}code.



{\bf George} 
3. In the definition of \vector{RI\_EXECB}
    A3.L is the absolute pointer to the list of op codes. It is not relative
    to A6 as stated. (So ANYROOT\program{ANYROOT} will work after all.)



{\bf Norman} 
Yes, as admitted above, this was
    a type on my part. A3.L is an non-{} relative address.



{\bf George} 
4.1 The byte op codes from \$33
    to \$FF can be used to save and load numbers. The effect of codes \$31 and
    \$32 depend on the operating system. JS ROM and SMSQ differ here. Both
    operating systems contain oddities. In the JS ROM \$ 31 will be treated as
    a save to -{}208(A4,A6.L). However, \$30 will be treated as the operation
    NOS\^{}TOS so that you can't bring back the saved item!



SMSQE has a different, though similar quirk. The code \$31 gives a
    "not implemented" error. Code \$32 puts PI on the stack. Code \$33 saves the
    number on the stack to -{}206(A4,A6.L). As with the JS ROM this number
    cannot be reloaded, since the code to do so just puts PI on the
    stack!

4.2 Norman mentions calling "this routine with \$FF33". This is of
    course impossible with \vector{RI\_EXECB}, the op code is just \$33. You can call
    \vector{RI\_EXEC} with D0.W equal to \$FF33, or \$1C33, or \$0033 and each will have
    the same effect. He then says that the actual address used for storage
    is: $$A6.L + A4.L + (D0.W \text{ AND } \$FFFE)$$

Again, I'm afraid this is not quite true. The address is
    actually: $$((op code \text{ AND } 254) - 256)(A4,A6.L)$$

{\bf Norman} 
I agree with George's point 4.1.
    As for 4.2, the address calculation I gave is the one I found in Jochen's
    QDOS Documentation.



{\bf George} 
4.3 Pennell is not wrong in the
    way Norman suggests in his first WARNING on page 28. First, as I have
    tried to explain, the op codes are really and truly byte sized numbers (2
    to 255). Second, Pennell gives the range for loading (not saving) a
    number. His range of offsets, -{}206 to -{}2 is absolutely correct for QDOS.
    It is when Pennell says that the op codes with odd values \$31 to \$FF give
    the same range he is wrong. That range is -{}208 to -{}2, but the value -{}208
    is effectively useless. I don't think this very minor error will bother
    anyone and it certainly does not warrant a WARNING.



{\bf Norman} 
I'm not doing very well am I? Once again, I stand corrected.



{\bf George} 
4.4 Norman suggests that in fact
    you can use a byte for the codes \$31 to \$FF when using \vector{RI\_EXEC}, but that
    it is an undocumented feature. This is not true since Pennell (page 133)
    does document it.



{\bf Norman} 
I blame Marcel, it's all his
    fault. (Only kidding.) As I mentioned above, I was basing my article on
    the latest information that I had been given as a result of asking for
    clarification on the QL-{}USERS mailing list.



{\bf George} 
4.5 I would like to add real
    WARNING. It is that you can crash the program by using an odd op code
    below 50 in QDOS. This is because the code is used directly as an index
    into the programs performing the operations.



{\bf Norman} 
This is indeed true.



{\bf George} 
I hope Norman will forgive me
    for attempting to set so many things straight. The errors are not wholly
    his fault!



{\bf Norman} 
Phew, I'm glad that's over. I
    took a severe beating at the hands of George and I promise to do better in
    future!



Seriously, I'm always happy to be corrected in anything I say or
    write -{} so, if you spot anything that you disagree with, let me
    know.

And as for forgiveness, I have no problems there either.

\section{Coming Up...}
\label{ch17-the-end}%hyperlabel{ch17-the-end}%

So, that's a slightly different article this time. I hope to be back
    in deepest, darkest code again next time, especially as I have promised to
    provide a useful ASCII to long word conversion routine. I think I know
    just where I can find one.....

