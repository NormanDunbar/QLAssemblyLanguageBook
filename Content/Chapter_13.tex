\chapter{Recursion}
\index{Recursion}

\section{Introduction}
\label{ch13-intro}%hyperlabel{ch13-intro}%

After the recent musings on single and double linked lists, this
    time I'm going to delve into the murky depths of a subject I've never seen
    before discussed for QDOSMSQ assembly language. The subject is
    recursion.

Recursion is a very simple concept, but for some people, it can be
    quite difficult to get their head around it, and it never comes clear. I
    suspect for those people, trying to do it in assembly language is equally
    difficult. Lets hope I can explain it in simple enough terms for even me
    to understand!

\section{Recursion in Assembly Language}
\label{ch13-recursion}%hyperlabel{ch13-recursion}%

A recursive routine is simply a routine which calls itself until a
    certain exit condition is detected. The exit condition is very important,
    if you miss it out, your programs will loop around until such time as the
    stack fills up and the program crashes, or eats itself.

Here's a very simple example of a program that will recurse
    `forever' until it dies.

\begin{lstlisting}[firstnumber=1,caption={Very Faulty Recursive Program}]
Start    bsr.s       Recurse
         rts

Recurse  bsr.s       Recurse
         rts
\end{lstlisting}

None of the \opcode{RTS} instructions will ever get executed. because all the
    program does is calls Recurse over and over again, but each call is nested
    inside the previous call, so the (A7) stack pointer keeps going down by 4
    each time it is called as the \opcode{BSR} instruction stacks the return address
    and then branches off to the next iteration of Recurse.

Recursion educators are very fond of certain examples when teaching
    recursion. The towers of Hanoi, Factorials, Exponentiation, Fibonacci
    numbers etc. I'm no different, so here's a few small explanations and
    examples.

\subsection{Factorials}
\label{ch13-factorials}%hyperlabel{ch13-factorials}%
\index{Recursion!Factorials}

The factorial of a number is that number, multiplied by the
      factorial of the number before it. The symbol for a factorial number is the exclamation mark (!)  So, $4! = 4 * 3!$
      
      There is no concept of a negative factorial, so $-3!$, doesn't
      exist. The smallest factorial number is $1!$, which has the value of 1.

Factorial numbers imply recursion and we have the following simple
      code.

However, just before we delve into the code be aware that factorials get
      very big very quickly, $12!$ (\$1C8CFC00) is the largest that can fit into
      a single 32 bit (unsigned) register so $13!$ ( \$17328CC00) cannot fit.
      The largest factorial we can fit into a 16 bit unsigned word is only $8!$
      (\$9D80) and as the 68008 processor can only multiply 16 bit numbers,
      this means that $9!$ will be the biggest that the following routine can
      calculate without overflowing.

\begin{note}
Other processors in the MC68000 family can multiply 32 bit
        numbers.
\end{note}

On entry to the code, D0.W is the number to calculate the
      factorial of and on exit, D0.L is the result. D0.W must in range 1 to 9
      only but this routine does not perform error trapping -{} on your own head
      be it!

\begin{lstlisting}[firstnumber=1,caption={Recursive Factorial Program}]
Start       bra.s     Start2        ; Skip over result area
Result      dc.l      0             ; Result area
FirstNumber dc.w      9             ; Assume 9! by default

Start2      lea       Result,a3     ; Where to shove the answer
            moveq     #0,d0         ; Clear all 32 bits of D0.L
            move.w    4(a3),d0      ; Fetch FirstNumber
            bsr.s     Factorial     ; Do it
FRET_1      move.l    d0,(a3)       ; Shove it!
            moveq     #0,d0         ; No errors
            rts                     ; Back to SuperBasic

Factorial   move.w    d0,-(a7)      ; Stack current number
            subq.w    #1,d0         ; Calculate previous number
            bne.s     Not_zero      ; Not done unless next number is 0
FPOP        move.w    (a7)+,d0      ; Retrieve the value 1 from stack
            rts                     ; Back to FRET_1 if FirstNumber 
*                                   ; was 1 else FRET_2

Not_zero    bsr.s     Factorial     ; Lets go round again, and again
FRET_2      mulu      (a7)+,d0      ; Do the multiplication
            rts                     ; Exit
\end{lstlisting}

For the simplest case, assume we start with a value of \$0001. The
      stack will look like this at label `Start':

\begin{lstlisting}[firstnumber=1,frame=none,numbers=none]
Return address to SuperBasic.
\end{lstlisting}

As we drop through the code beginning at `Start2' and execute our
      first branch to subroutine `Factorial' the stack now looks like this
     :

\begin{lstlisting}[firstnumber=1,frame=none,numbers=none]
Return address to SuperBasic.
Return address to FRET_1
\end{lstlisting}

Tracing through the `Factorial' code, we stack the current value
      of D0.W (which is 1) so the stack now looks like this:

\begin{lstlisting}[firstnumber=1,frame=none,numbers=none]
Return address to SuperBasic.
Return address to FRET_1
$0001
\end{lstlisting}

After the subtraction, D0 has become zero, so we exit out of the
      `Factorial' code from label FPOP by popping the value \$0001 off the
      stack into D0.W leaving the stack like this:

\begin{lstlisting}[firstnumber=1,frame=none,numbers=none]
Return address to SuperBasic.
Return address to FRET_1
\end{lstlisting}

Then we execute the \opcode{RTS} instruction to return us to the label
      FRET\_1 where we store the result of \$00000001 from D0.L into the result
      area set aside for this very purpose.

So far so good, we haven't actually done any recursion yet, but
      read on. If we start with the value of \$0002 in `FirstNumber' then the
      process is slightly different. We start, as ever with the SuperBasic
      return address on the stack when we are at label `Start'.

Dropping into `Start2' and executing our first \opcode{BSR} to `Factorial',
      the stack is as above at the same point. Nothing much has changed.
      However, we then stack the current value in D0.W to give a stack as
      follows:

\begin{lstlisting}[firstnumber=1,frame=none,numbers=none]
Return address to SuperBasic.
Return address to FRET_1
$0002
\end{lstlisting}

This is slightly different. When we calculate the next number, we
      do not set D0.W to zero, so we skip out of the `Factorial' code block to
      the code at `Not\_zero' which immediately causes another \opcode{BSR} to
      `Factorial' leaving the stack as follows:

\begin{lstlisting}[firstnumber=1,frame=none,numbers=none]
Return address to SuperBasic.
Return address to FRET_1
$0002
Return address to FRET_2
\end{lstlisting}

Once again, we stack D0.W and subtract one to find that we have
      now reached zero. The stack looks like this:

\begin{lstlisting}[firstnumber=1,frame=none,numbers=none]
Return address to SuperBasic.
Return address to FRET_1
$0002
Return address to FRET_2
$0001
\end{lstlisting}

Once more, we pop the value \$0001 off the stack back into D0.W and
      then execute the \opcode{RTS} instruction. This time, however, we do not return
      to FRET\_1 but to FRET\_2 where we end up with the following stack
      arrangement:

\begin{lstlisting}[firstnumber=1,frame=none,numbers=none]
Return address to SuperBasic.
Return address to FRET_1
$0002
\end{lstlisting}

The instruction at `FRET\_2' causes the top word on the stack to be
      multiplued by D0.W the result store in D0.L. This leaves D0.L equal to
      \$00000002 which just happens to be the correct value for $2!$ and we exit
      the code by returning to `FRET\_1' where we store the result
      again.

The process is similar for all the other numbers, so $5!$ will have
      a stack looking like this when we reach, but just before we execute the
      code at `FPOP':

\begin{lstlisting}[firstnumber=1,frame=none,numbers=none]
Return address to SuperBasic.
Return address to FRET_1
$0005
Return address to FRET_2
$0004
Return address to FRET_2
$0003
Return address to FRET_2
$0002
Return address to FRET_2
$0001
\end{lstlisting}

The stack will begin to unwind as we do the sequence of \opcode{MULU} and
      \opcode{RTS} instructions at FRET\_2 as we first calculate $1!$, then $2!$, then $3!$,
      then $4!$ and finally $5!$ which is the result we return to
      SuperBasic.

To run the above code, do this, or something like it:

\begin{lstlisting}[firstnumber=1,language={}]
1000 Start = ALCHP(128)
1010 LBYTES win1_source_factorial_bin, Start
1020:
1030 DEFine PROCedure Fact(n)
1040   IF n < 1 OR n > 9 THEN
1050      PRINT n; ' is slightly out of range, 1 to 9 only please.'
1060   END IF
1070   POKE_W Start + 6, n
1080   CALL Start
1090   PRINT n; '! = '; PEEK_L(Start + 2)
1100 END DEFine Fact
\end{lstlisting}

Now, at the SuperBasic prompt, run the above to load the code, you
      only need to do this once, then just type Fact(n) where `n' is a value
      between 1 and 9 as described above in the text. The results will be
      `interesting' if you use values outside of this range.

Actually, in the interests of experiment, I tried it out. Using
      values above 9 is fine, up to a point, but zero will trash SuperBasic as
      the stack wanders down through memory and corrupts data that it shouldn't
      be anywhere near. Larger values will no doubt have the same effect, but
      anything over 9 gives incorrect results as the 16 bit \opcode{MULU} instruction
      isn't using the additional bits of the number. Anyone got a good 32 bit
      \opcode{MULU} and/or \opcode{MULS} routine they want to share?

I don't have the numeric skills to write one, and while there are
      plenty on the Web, they are, of course, someone else's work and subject
      to copyright etc.

\subsection{The Fibonacci Series}
\label{ch13-fibonacci}%hyperlabel{ch13-fibonacci}%
\index{Recursion!Fibonacci}

The Fibonacci series looks like this:

\begin{lstlisting}[firstnumber=1,frame=none,numbers=none]
1, 1, 2, 3, 5, 8, 13, 21, 34, 55 ...
\end{lstlisting}

Apart from the first two numbers, each number in the series is the
      sum of the previous two numbers. This is written as

\begin{lstlisting}[firstnumber=1,frame=none,numbers=none]
Fibonacci(N) = Fibonacci(N-1) + Fibonacci(N-2)
\end{lstlisting}

The very explanation cries out for recursion, you can see it in
      the statement above. We need to cater for the first two terms in the
      series and test for a Fibonacci(0) or Fibonacci(1) and return the value
      1 for both of these values. The slight difference between Fibonacci and
      Factorial is that we need to recurse twice for each number, once for
      ($N-{}1$) and once for ($N-{}2$). This makes the code slightly more interesting
      and the stack too.

Here's how it looks in plain and simple SuperBasic:

\begin{lstlisting}[firstnumber=1,language={}]
1000 DEFine FuNction Fibonacci(n)
1010   IF n = 0 OR n = 1 THEN RETurn 1
1020   RETurn Fibonacci(n-1) + Fibonacci(n-2)
1030 END DEFine
\end{lstlisting}

So how difficult could it be to convert the above (two) lines of
      working code into assembly language? It all depends on how easily you
      get your head around the recursion, I had to sit and stare at the screen
      for a while until I finally came up with the following code:

\begin{lstlisting}[firstnumber=1,caption={Recursive Fibonacci Program}]
Start       bra.s    Start2         ; Skip over result storage
Result      dc.l     0              ; Space for result at Start + 2
FirstNumber dc.l     9              ; Fib(9) by default = Start + 6

Start2      lea       Result,a3     ; Where to shove the answer
            move.l    4(a3),d0      ; Fetch FirstNumber
            bsr.s     Fibonacci     ; Do it
FRET_1      move.l    d0,(a3)       ; Shove it!
            moveq     #0,d0         ; No errors
            rts                     ; Back to SuperBasic

Fibonacci  cmpi.l   #2,d0           ; Special cases 0 or 1?
           bcc.s    Fib_2           ; No, D0 is 2 or more. (Unsigned!)
           moveq    #1,d0           ; Return 1 for Fib(0) or Fib(1)
           rts                      ; That's our exit clause!

Fib_2      subq.l   #1,d0           ; Calculate N-1
           move.l   d0,-(a7)        ; Stack our 'N-1' value
           bsr.s    Fibonacci       ; Work out Fib(N-1)
FRET_2     move.l   d0,-(a7)        ; Save the result of Fib(N-1)
           move.l   4(a7),d0        ; Retrieve N-1
           subq.l   #1,d0           ; Calculate N-2
           bsr.s    Fibonacci       ; Work out Fib(N-2)
FRET_3     add.l    (a7)+,D0        ; Add Fib(N-1) to Fib(N-2)
           adda.l   #4,a7           ; Tidy original N-1 off of stack
           rts                      ; And return
\end{lstlisting}

To run the above code, do this, or something like it:

\begin{lstlisting}[firstnumber=1,language={}]
1000 Start = ALCHP(128)
1010 LBYTES win1_source_fibonacci_bin, Start
1020:
1030 DEFine PROCedure Fib(n)
1040   IF n < 0 THEN
1050      PRINT n; ' is slightly out of range, 0 and over only please.'
1060   END IF
1070   POKE_L Start + 6, n
1080   CALL Start
1090   PRINT n; '! = '; PEEK_L(Start + 2)
1100 END DEFine Fib
\end{lstlisting}

This time we can use numbers larger than 9 as we are adding 32 bit
      values in the code, not multiplying. Of course, you can still pick a
      number big enough to trash the stack. Interestingly enough, Fib (30)
      executes in 1 second on my QPC setup, but the original SuperBasic
      version ran and ran and ran I CTRL-{}SPACE'd it after a while.

As an exercise, try to work out what the stack looks like for
      different values of N -{} it's an interesting lesson in mind numbing
      loops. Once you figure it out though, it gets easier.

When you are writing recursive code like the two examples above,
      you must remember two golden rules:
\begin{itemize}[itemsep=0pt]

\item{}you must always have a `get out' clause to stop
          recursion


\item{}never ever try to use other registers as storage -{} it just
          doesn't work!

\end{itemize}

In the above, we just stacked our working values and this is fine,
      but in other code, you might need to have a lot more values to stack, so
      how best to do this? The answer is quite simple, use the \opcode{LINK} and \opcode{UNLK}
      instructions which are designed to build stack frames that you can
      access using Address Register Indirect With Displacement -{} for example
      4(a5) -{} addressing mode instructions.

Out of interest, has anyone spotted a potential problem with the
      above code?

The calculation of Fib(N-{}2) duplicates most of the work done by
      Fib(N-{}1). One solution to this problem is to have an array of values
      in memory and when calculating a new value, store it in the table if it
      has not been stored already, if it has been stored already, simply
      extract it from the table.

The last two paragraphs should have given you an inkling of some
      homework -{} which will not be marked -{} feel free to try out the implied
      exercises for yourself. The only problem with the array of values is
      that you never know how big to make the table and you need some method
      of determining if the table has been initialised (to all zeros) GWASL\program{Gwasl}
      doesn't fill buffers with zeros, just with assorted random characters,
      unlike an array in SuperBasic.

The array could be set up as follows:

\begin{lstlisting}[firstnumber=1,caption={Improving the Fibonacci Code - Answers Array}]
Answers   dc.l     0         ; Fib(0), 0 indicates uninitialised table
          ds.l     1000      ; Fib(1) through Fib(1000)
\end{lstlisting}

Because GWASL won't initialise the entries for 2 to 1000 you have
      to do it yourself, as follows:

\begin{lstlisting}[firstnumber=last,caption={Improving the Fibonacci Code - Array Initialisation}]
Init      lea       Answers,a3     ; Start of answer array
          cmpi.l    #0,(a3)        ; Has table been initialised yet?
          bne.s     Done           ; Yes, exit
          move.w    1000,d0        ; 1000 = 1001 entries
I_Loop    clr.l     (a3)+          ; Clear this entry, point at next
          dbra      d0,I_Loop      ; Do the rest
          lea       Answers,a3     ; Answer(0)
          move.l    #1,(a3)+       ; Initialised to Fib(0)
          move.l    #1,(a3)        ; An initialise Fib(1) as well.
\end{lstlisting}

Code like the above should be called at the start of the file so
      that the initialisation is only ever performed once per session. Making
      multiple calls to the Fibonacci code will only require the table be
      initialised once.

When the code has calculated Fib(N-{}1) then it can store the result
      in D0.L into the table. As N-{}1 is on the stack, it can be retrieved into
      a spare register -{} say D1 -{} and converted to an offset by shifting it
      right twice (LSL.L \#2,D1) and using that as an offset into the answer
      table.

Now, when asked to calculate a value, check the offset into the
      table and if it is not zero, return that as your answer -{} no recursion
      and much faster. You'll have to remember to limit the number of
      allowable values if using a table -{} you could end up corrupting some
      random bits of memory and the amount of space you need to ALCHP will go
      up as a result of the table -{} check it after assembly to see how big it
      is.

Have fun and if you feel brave, Dilwyn wrote a SuperBasic version
      of `The Towers Of Hanoi' some time back, why not convert that to
      Assembly language:o)

\section{Coming Up...}
\label{ch13-the-end}%hyperlabel{ch13-the-end}%

The next chapter takes a bit of a breather from all this hard work
    writing code. In it, I'll discuss \emph{my} own personal
    methods of writing code.

