\chapter{Application Sub-{}Window Menus}

\section{Introduction}
\label{ch29-intro}%hyperlabel{ch29-intro}%

At the end of the last chapter, I promised to continue looking at
    Application Sub-{}Windows by adding Application Sub-{}Window Menus to them.

Effectively, there are two different types of application sub-{}window
    menus:
\begin{itemize}[itemsep=0pt]

\item{}Static -{} which are defined in the program source code and never
        change;

\item{}Dynamic -{} which may change as the applications runs.

\end{itemize}

In this chapter, we shall look at the Static Application Sub-{}Window
    Menus only. In a future chapter, we shall look at Dynamic Menus as they are much more difficult to set up correctly.

\section{Static Application Sub-{}Window Menus}
\label{ch29-app-menu-static}%hyperlabel{ch29-app-menu-static}%

Static menus, as I shall call them from this point onwards, are
    created by the developer as s/he writes the program. As the program runs,
    static menus do not change -{} other than setting entries to available or
    unavailable as required.

We can use SETW\program{SETW} to create static menus.
    All that is required is for the developer to decide on the menu options,
    the required layout of rows and columns, and what to do when the user
    clicks on an option -{} although this latter option is not needed by
 SETW\program{SETW}, only in the application's code.

We need to design a new window using SETW\program{SETW}, so proceed to execute the
    utility and proceed as follows:
\begin{enumerate}
\item{When prompted for `name\$' enter
 \emph{AppMenuTest1}. I'd like to use the name \emph{AppMenuTest1Win},
          but that is too big for SETW. When
          finished, the file \emph{AppMenuTest1\_asm} is easily
          renamed to \emph{AppMenuTest1Win\_asm}.
}
\item{On the `Alter Text' screen.
\begin{itemize}[itemsep=0pt]

\item{}Press N for new, type `X' (without the quotes) then
            ENTER.


\item{}Press N for new, type `Application Menu Test 1' (without the
            quotes) then ENTER


\item{}Press N for new, type `One' then enter.
\item{}Press N for new, type `Two' then enter.
\item{}Press N for new, type `Three' then enter.
\item{}Press N for new, type `Four' then enter.
\item{}Press N for new, type `Five' then enter.
\item{}Press N for new, type `Six' then enter.
\item{}Press N for new, type `Seven' then enter.
\item{}Press N for new, type `Eight' then enter.
\item{}Press N for new, type `Nine' then enter.
\item{}Press N for new, type `Ten' then enter.

\item{}Press ESC.


\end{itemize}
}
\item{On the `Alter Sprite' screen.
\begin{itemize}[itemsep=0pt]

\item{}Press ESC.

\end{itemize}
}
\item{On the `Alter Blob' screen.
\begin{itemize}[itemsep=0pt]

\item{}Press ESC.

\end{itemize}
}
\item{On the `Alter Patt' screen.
\begin{itemize}[itemsep=0pt]

\item{}Press ESC.

\end{itemize}
}
\item{Number of main windows = 1.
}
\item{Number of Loose Items = 1.
}
\item{Number of Information windows = 2.
}
\item{For Information Window 1 of 2, the number of IW Objects =
        1.
}
\item{For information windows 2 of 2, the number of IW Objects =
        0.
}
\item{Number of application windows = 1.
}
\item{Application windows menu items = 10.
}
\item{For main window 1:
\begin{itemize}[itemsep=0pt]

\item{}Shadow = 2


\item{}Border size = 1


\item{}Border colour = colour\_ql -{}>{} black


\item{}Paper colour -{} colour\_ql -{}>{} white


\item{}Sprite = arrow

\end{itemize}
}
\item{Presentation of loose Items:
\begin{itemize}[itemsep=0pt]

\item{}Press N for `system palette defaults'


\item{}Confirm N when prompted again for defaults


\item{}Border size = 1


\item{}Border colour = colour\_ql -{}>{} black


\item{}Unavailable background = colour\_ql -{}>{} white


\item{}Unavailable Ink = colour\_ql -{}>{} grey


\item{}Available background = colour\_ql -{}>{} white


\item{}Available Ink = colour\_ql -{}>{} black


\item{}Selected background = colour\_ql -{}>{} green


\item{}Selected Ink = colour\_ql -{}>{} black

\end{itemize}
}
\item{Loose Item 1:
\begin{itemize}[itemsep=0pt]

\item{}Type = text


\item{}Object -{}>{} select the `X' text object


\item{}Selection key = ESC

\end{itemize}
}
\item{Information Window 1:
\begin{itemize}[itemsep=0pt]

\item{}Border size = 0


\item{}Paper = colour\_ql -{}>{} No 92

\end{itemize}
}
\item{Object 1:
\begin{itemize}[itemsep=0pt]

\item{}Type = text


\item{}Object -{}>{} select the `Application Window Test' text
            object.


\item{}Colour = colour\_ql -{}>{} black


\item{}Xcsize = 0


\item{}Ycsize = 0

\end{itemize}
}
\item{Information Window 2:
\begin{itemize}[itemsep=0pt]

\item{}Border size = 1


\item{}Border colour = ql\_colour -{}>{} black


\item{}Paper = colour\_ql -{}>{} white

\end{itemize}
}
\item{Application Window 1:
\begin{itemize}[itemsep=0pt]

\item{}Border size = 1


\item{}Border colour = colour\_ql -{}>{} black


\item{}Paper colour = colour\_ql -{}>{} white


\item{}Sprite = arrow


\item{}Selection key = TAB


\item{}Presentation of Menu Items\begin{itemize}[itemsep=0pt]

\item{}Select N for system palette defaults


\item{}Select N for defaults, again.


\item{}Border size = 1.


\item{}Border colour = ql\_colour -{}>{} black.


\item{}Unavailable background = ql\_colour -{}>{} white.


\item{}Unavailable ink = ql\_colour -{}>{} grey.


\item{}Available background = ql\_colour -{}>{} white.


\item{}Available ink = ql\_colour -{}>{} black.


\item{}Selected background = ql\_colour -{}>{} green.


\item{}Selected ink = ql\_colour -{}>{} black.


\item{}Scroll arrow = ql\_colour -{}>{} white.


\item{}Scroll bar = ql\_colour -{}>{} black.


\item{}Scroll background = ql\_colour -{}>{} red.


\item{}When prompted for the ten menu items, select the text
                  items `One' through `Ten'. Give each one a selection key of
                  the digit that matches the number described by the text
                  object. For example, `One' has a key of `1', `Two' has `2'
                  and so on up to `Ten' which has selection key `0'
                  (Zero).

\end{itemize}


\end{itemize}
}
\item{Main window size: (Use the arrow keys to change the size, press
        ENTER when correct)
\begin{itemize}[itemsep=0pt]

\item{}Width = 220


\item{}Height = 140


\item{}Do you want a variable window = N


\item{}Set the origin to 0,0 (Press ENTER when correct)

\end{itemize}
}
\item{Loose Item 1: (Toggle hit/position with F2. Press ENTER when
        correct)
\begin{itemize}[itemsep=0pt]

\item{}Hit size = 10 x 10


\item{}Position = 206 x 3

\end{itemize}
}
\item{Information Window 1: (Toggle size/position with F2. Press ENTER
        when correct)
\begin{itemize}[itemsep=0pt]

\item{}Size = 220 x 16


\item{}Position = 0 x 0


\item{}Object position = 2 x 2

\end{itemize}
}
\item{Information Window 2: (Toggle size/position with F2. Press ENTER
        when correct)
\begin{itemize}[itemsep=0pt]

\item{}Size = 216 x 14


\item{}Position = 2 x 125

\end{itemize}
}
\item{Application Window 1: (Toggle size/position with F2. Press ENTER
        when correct)
\begin{itemize}[itemsep=0pt]

\item{}Size = 208 x 104


\item{}Position = 2 x 18

\end{itemize}
}
\end{enumerate}

When you have completed this procedure, and
 SETW\program{SETW} has exited, you should save the file
 ram1\_AppMenuTest1\_asm to a safer place and rename it
    to AppMenuTest1Win\_asm. The file should look similar
    to the following, although I have added some extra comments to my copy of
    the generated code.

\section{The Generated Code}
\label{ch29-app-menu-code}%hyperlabel{ch29-app-menu-code}%

The file should look similar to the following, although I have added
    some extra comments to my copy of the generated code.

\begin{note}
I have removed a few sections of the following file in order to
      reduce duplication of chunks of code in the magazine. These sections are
      discussed below.
\end{note}

\begin{lstlisting}[firstnumber=1,caption={AppMenuTest1Win\_asm}]
; AppMenuTest1Win_asm.

; Undefined Labels - need to be defined elsewhere in my own code.

;        ahit2_0 - menu item 0, hit routine.
;        ahit2_1 - menu item 1, hit routine.
;        ahit2_2 - menu item 2, hit routine.
;        ahit2_3 - menu item 3, hit routine.
;        ahit2_4 - menu item 4, hit routine.
;        ahit2_5 - menu item 5, hit routine.
;        ahit2_6 - menu item 6, hit routine.
;        ahit2_7 - menu item 7, hit routine.
;        ahit2_8 - menu item 8, hit routine.
;        ahit2_9 - menu item 9, hit routine.
;        asmnu0  - User defined setup routine.
;        adraw0  - User defined draw routine.
;        ahit0   - application window 0 hit action routine.
;        afun0_0 - Loose item 0 hit action routine.

; Labels for External Use
;     mst0    - menu items status area
;     wst0    - Window status area
;     wd0     - Window definition address
;     ww0_0   - Window default size
;     ww0_1   - Window button size

SYS_SPR  dc.w     0,1,2,3,4,5,6,7,8,9,10,11,12,13,14,15,16,17,18,
                  19,20,21,22,23,24,25,26,7,28,29,30,31,32,33,
                  34,35,36,37

; Text object for "Close" loose item.
txt0     dc.w      txt0_e-2-txt0
         dc.b      "X"
txt0_e   ds.b      0
         ds.w      0

; Text object for caption bar.
txt1     dc.w      txt1_e-2-txt1
         dc.b      "Application Menu Test 1"
txt1_e   ds.b      0
         ds.w      0

;----------------------------------------------------------------------
; **** Text objects for the menu items. Removed - see text.

; **** Menu items list. Removed - see text.

; **** Row list. Removed - see text.

; **** Spacing list. Removed - see text.
;----------------------------------------------------------------------

; Application window list.
app_list0
         dc.w      appw0-*
         dc.w      0


;----------------------------------------------------------------------
; **** Application window 0 definition. Removed - see text.
;----------------------------------------------------------------------


; Information Object(s).
pobl0
         dc.w      138       xsize
         dc.w      10        ysize
         dc.w      2         xorg
         dc.w      2         yorg
         dc.b      0         type
         dc.b      0         spar
         dc.l      0         spce
         dc.w      txt1-*    pobj *
         dc.w      -1

; Information Window(s).
infw0
         dc.w      220       xsize
         dc.w      16        ysize
         dc.w      0         xorg
         dc.w      0         yorg
         dc.w      0         flag
         dc.w      0         borw
         dc.w      526       borc
         dc.w      92        papr
         dc.w      pobl0-*   pobl *
         dc.w      216       xsize
         dc.w      14        ysize
         dc.w      2         xorg
         dc.w      125       yorg
         dc.w      0         flag
         dc.w      1         borw
         dc.w      0         borc
         dc.w      7         papr
         dc.w      0         pobl *
         dc.w      -1        end

; Loose Item(s).
litm0
         dc.w      10,10     xsize, ysize
         dc.w      206,3     xorg, yorg
         dc.b      0,0       xjst, yjst
         dc.b      0,3       type, skey
         dc.w      txt0-*    pobj *
         dc.w      0         item
         dc.w      afun0_0-* pact *
         dc.w      -1        end

litm1
         dc.w      16404,12  xsize, ysize
         dc.w      0,0       xorg, yorg
         dc.b      0,0       xjst, yjst
         dc.b      0,0       type, skey
         dc.w      0         pobj *
         dc.w      0         item
         dc.w      0         pact *
         dc.w      -1        end

; Window definition
wd0
         dc.w      220       xsize
         dc.w      140       ysize
         dc.w      0         xorg
         dc.w      0         yorg
         dc.w      258       flag
         dc.w      1         borw
         dc.w      0         borc
         dc.w      7         papr
         dc.w      0         sprt *
         dc.w      1         curw
         dc.w      0         curc
         dc.w      7         uback
         dc.w      255       uink
         dc.w      0         ublob *
         dc.w      0         upatt *
         dc.w      7         aback
         dc.w      0         aink
         dc.w      0         ablob *
         dc.w      0         apatt *
         dc.w      4         sback
         dc.w      0         sink
         dc.w      0         sblob *
         dc.w      0         spatt *
         dc.w      0         help
         dc.w      220       xsize
         dc.w      140       ysize
         dc.w      infw0-*   pinfo *
         dc.w      litm0-*   plitem *
         dc.w      app_list0-*         pappl *
         dc.w      16384     xsize
         dc.w      12        ysize
         dc.w      0         pinfo *
         dc.w      litm1-*   plitem *
         dc.w      0         pappl *
         dc.w      -1

; Sizes
ww0_0    equ       670
ww0_1    equ       148

; Status Areas:
; Menu item status area.
mst0     ds.b      10
mst0_e   ds.b      0
         ds.w      0

; Window status area.
wst0     ds.b      65
wst0_e   ds.b      0
         ds.w      0
\end{lstlisting}

Much of the above is similar to when was discussed in a previous
    article. If you think back to that article on application sub-{}windows, we
    simply used the hit routine to print text all over the application
    sub-{}window. That was about as simple as it gets -{} other than not actually
    having a hit routine I suppose! Adding a menu to an application
    sub-{}window means we have quite a lot more work to do at the coding stage
    as we have to consider the following:
\begin{itemize}[itemsep=0pt]

\item{}Defining the menu objects -{} these can be text, sprite etc. Every
        menu item must be defined.


\item{}Defining the menu items list -{} when we have defined each menu
        object, we then have to build a list of all the menu items that we
        wish to include in our final menu.


\item{}Defining the menu row list -{} when we have the menu items list
        defined, we amalgamate that list into a menu row list which defines
        the start and end of each row in the menu.


\item{}Defining the spacing lists -{} the row list defines the hit size
        and the spacing for each row and each column in the menu.


\item{}Define the application sub-{}window -{} menus need their own section
        in the application sub-{}window definition.

\end{itemize}

So much to do just to show a menu in a window, lets get on and do
    it.

\section{Menu Objects}
\label{ch29-app-menu-objects}%hyperlabel{ch29-app-menu-objects}%

The first stage is to define the various objects that will be
    incorporated into the menu. In this example, I have used ten separate text
    objects (as they are the simplest). You can use any of the various Pointer
    Environment object types if you wish. The code generated by
 SETW\program{SETW} for these items is as follows.

\begin{lstlisting}[firstnumber=1,caption={AppMenuTest1Win\_asm - Menu Objects}]
; **** Text objects for the menu items.

txt2     dc.w      txt2_e-2-txt2
         dc.b      "One"
txt2_e   ds.b      0
         ds.w      0


; **** Txt3 through txt10 deleted for brevity.


txt11    dc.w      txt11_e-2-txt11
         dc.b      "Ten"
txt11_e  ds.b      0
         ds.w      0
\end{lstlisting}

As you can see from the above, I have omitted to show text objects
    txt3 through txt10 as there is really no need to take up space in the
    magazine with repetitive data.

The above code simply defines ten separate text objects -{} `One',
    `Two', ..... `Nine' and `Ten' to be used, later, in our static
    menu.

You should be aware, also, that the line numbers above (and indeed, in the following snippets) bear no resemblance to the ones in the original file.

\section{Menu Items (and Index) List}
\label{ch29-app-menu-item-list}%hyperlabel{ch29-app-menu-item-list}%

The next section of code that I have removed from the main listing
    above is the menu items list. This is shown below, but please note that
    once again, I have removed the vast majority of the code for
    brevity.

\begin{lstlisting}[firstnumber=last,caption={AppMenuTest1Win\_asm - Menu Item List}]
; **** Menu items list. 

meos2    dc.b      1,0             ; x_justification, y_justification 
         dc.b      0,49            ; Item type, selection key
         dc.w      txt2-*          ; Pointer to object
         dc.w      0               ; Item number. (-1 for indexes)
         dc.w      ahit2_0-*       ; Hit routine for this item
;                                  ; Zero for indexes


; NOTE: Menu items 1 through 8 removed here for brevity.


         dc.b      1,0             ; x_justification, y_justification 
         dc.b      0,48            ; Item type, selection key
         dc.w      txt11-*         ; Pointer to object
         dc.w      9               ; Item number. (-1 for indexes)
         dc.w      ahit2_9-*       ; Hit routine for this item
;                                  ; Zero for indexes
\end{lstlisting}

The label meos2 is the start of the menu items list. We told SETW\program{SETW}
    that there would be 10 items, and so, there will be 10 menu items in this
    list. Each one has the structure described in \tablename~\ref{tab:ApplicationSubWindowMenuItemListEntry}.

\begin{table}[htbp]
\centering
\begin{tabular}{c c p{0.7\textwidth}}
\toprule
\textbf{Offset} & \textbf{Size} &\textbf{Description}  \\
\midrule
%
0 & 1 & X (Horizontal) justification.
\begin{itemize}[itemsep=0pt]
\item{} -ve = Right justified
\item{}0 = Centred
\item{}+ve = Left Justified
\end{itemize}
The justification value is the number of pixels from the edge of the hit area that the object is
to be positioned at.\\

1 & 1 & Y (Vertical) justification.
\begin{itemize}[itemsep=0pt]
\item{}-ve = Bottom justified
\item{}0 = Centred
\item{}+ve = Top Justified
\end{itemize}
The justification value is the number of pixels from the edge of the hit area that the object is
to be positioned at.\\

2 & 1 & Item type:
\begin{itemize}[itemsep=0pt]
\item{}0 - Text object
\item{}2 - Sprite object
\item{}4 - Blob object
\item{}6 - Pattern object
\end{itemize}
\\

3 & 1 & Selection key, upper cased if necessary.\\

4 & 2 & Relative pointer to the actual object.\\

6 & 2 & Item number. If this is an index items list, set to -1 for all items.\\

8 & 2 & relative pointer to the hit/action routine for this particular item. For index item lists, must be zero.\\
%
\bottomrule
\end{tabular}
\caption{Application Sub-Window Menu Item List Entry}
\label{tab:ApplicationSubWindowMenuItemListEntry}
\end{table}

 


Each item in a menu items list is 10 bytes in length.

Looking at the first menu object generated by
 SETW\program{SETW}, we see that it is left justified and one
    pixel from the start of the left end of the hit area. Vertically, it is
    centred within the hit area. It is a text object and has a selection key
    of 49, which is the code for a digit `1' on the keyboard. The text object
    itself is the text `One' as shown above in the preceding section. The item
    is numbered zero and the hit routine for this item is defined to be at
    label ahit2\_0.

So far, so good. We have defined a list of objects and then gathered
    them into a list of menu items. The menu items list -{} in this case -{} is in
    a contiguous section of memory, it need not be so. The row list defines
    the menu ordering, that comes next.

The above structure is used to define menu item lists and also index
    lists. An index is drawn by the \pe{WM\_INDEX} vector (which also draws pan and
    scroll bars \& arrows -{} if necessary). Indices are best thought of as
    the row and column headings -{} similar to a spreadsheet, for example, where
    columns have letters and rows have numbers to identify them.

WMAN\program{WMAN} takes
    care of aligning the indices with the contents of the static menu.

\section{Row List}
\label{ch29-app-menu-row-list}%hyperlabel{ch29-app-menu-row-list}%

The row list takes the various menu items, defined above, and
    organises them into rows -{} surprisingly enough. For every row you wish to
    have in your menu, you need a single row list entry. As SETW\program{SETW} tries to make
    as few rows and/or columns as it can -{} it tries to fit as much as possible
    into a given space -{} what SETW\program{SETW} generates may not be what you want. In the
    default case for our SETW\program{SETW} session, we have been given two rows and thus,
    five columns, for our ten menu items.

Our generated row list is as follows:

\begin{lstlisting}[firstnumber=last,caption={AppMenuTest1Win\_asm - Row List}]
; **** Row list.

drow0    dc.w      0+meos2-*       ; Pointer to row 0 start = item 0.
         dc.w      50+meos2-*      ; Pointer to row 0 end = item 4.

         dc.w      50+meos2-*      ; Pointer to row 1 start = item 5.
         dc.w      100+meos2-*     ; Pointer to row 1 end = item 9.
\end{lstlisting}

Each row list item contains two pointers, the first is to the start
    of the first entry in the menu items list entry for this row. The second
    pointer is to the first byte past the end of the last menu items list
    entry for this row.

Given the above then, we can see that the first row, starting at
    label drow0, begins at meos2 (relative to the pointer itself -{} as usual).
    Meos2 is a list of 10 sets of 10 byte entries defining all ten items in
    our menu. The first row ends at meos2 + 50, which happens to be the very
    first byte of the menu items list for menu item number 5 (ie, the sixth
    menu item -{} we count from zero)

The second row list entry starts at meos2 + 50 and ends at meos2 +
    100. These pointers are to menu items list item number 5 and at the first
    byte past the very last menu items list entry. As the following diagram
    attempts to display:

\begin{lstlisting}[frame=none,numbers=none,caption={Relationship between the Row List \& Menu Items List} ]
Meos2+0     Menu Item 0 <-----+          Start of row 0.   
Meos2+10    Menu Item 1       |  
Meos2+20    Menu Item 2       |
Meos2+30    Menu Item 3       |
Meos2+40    Menu Item 4       |
Meos2+50    Menu Item 5 <--------+--+    End of row 0, start of row 1.
Meos2+60    Menu Item 6       |  |  |  
Meos2+70    Menu Item 7       |  |  |  
Meos2+80    Menu Item 8       |  |  |
Meos2+90    Menu Item 9       |  |  |
Meos2+100   equ  *      <--------------+ End of row 1.
                              |  |  |  | 
Drwo0       Pointer to -------+  |  |  |
            Pointer to ----------+  |  |
            Pointer to -------------+  |
            Pointer to ----------------+
\end{lstlisting}

If the menu items list is a single chunk of memory, then each row
    start pointer is equal to the previous row end pointer -{} except for the
    first row. As in the example above, the end pointer for row 0 is the same
    address as the start pointer for row 1.

Now that we have our rows defined, we have to set up the spacing
    lists for each row and each column in the menu.

\section{Spacing Lists}
\label{ch29-app-menu-space-lists}%hyperlabel{ch29-app-menu-space-lists}%

Each menu item in our static menu has a given hit area and a
    spacing. The hit area defines where the pointer can be to make the item
    beneath it the current item, this is normally indicated by a border being
    drawn around the current item. A HIT or a DO within the hit area, or a
    press of the selection keystroke while the pointer is within the
    application sub-{}window, will cause the appropriate menu item action
    routine to be executed.

The spacing defines how many pixels across -{} or down depending on
    whether this is the column or row spacing list -{} there are between the
    start of `this' menu item and the start if the `next' one. The spacing
 \emph{must} include an allowance for the border to be drawn
    around the current item.

\begin{lstlisting}[firstnumber=42,caption={AppMenuTest1Win\_asm - Spacing Lists}]
; **** Spacing lists.

; Spacing list. Defines width of hit area for each COLUMN and spacing
; between columns. (5 columns.)

spls0    dc.w      34      ; Hit area width column 0
         dc.w      36      ; Space between this column and the next.

         dc.w      34      ; Hit area width column 1
         dc.w      36      ; Space between this column and the next

         dc.w      34      ; Hit area width column 2
         dc.w      36      ; Space between this column and the next

         dc.w      34      ; Hit area width column 3
         dc.w      36      ; Space between this column and the next

         dc.w      34      ; Hit area width column 4
         dc.w      36      ; Space between this column and the next


; Spacing list. Defines height of hit area for each ROW and spacing
; between rows. (2 rows.)

spls1    dc.w      10      ; Hit area height row 0
         dc.w      12      ; Space between this row and the next

         dc.w      10      ; Hit area height row 1
         dc.w      12      ; Space between this row and the next
\end{lstlisting}

The first list above, at label spls0 defines the columns in our
    menu. We already know that SETW\program{SETW} has decreed
    that there shall be five columns and two rows, so the column spacing list
    has five entries, one for each column. Each entry consists of a pair of
    words -{} the first defines the width of the column (or the height of the
    row) and the second defines the space between this column and the
    next.

In the example above, we see that SETW\program{SETW} has calculated that our
    widest text object is 5 characters wide -{} this corresponds to `Three',
    `Seven' and `Eight' -{} and has allocated 34 pixels of hit area for each
    column. The spacing for each columns is set to the (border width * 2) plus
    the hit area width. It must be twice the border width as there is a border
    on each side (or top \& bottom).

The spacing list for the rows shows a height of ten pixels for the
    hit area and taking the border into consideration again, a spacing of 12
    pixels between the tops of each row.

\section{Menu Section of Application Window Definition}
\label{ch29-app-menu-win-def}%hyperlabel{ch29-app-menu-win-def}%

The application window definition needs an extra section adding
    after the normal definition, to cover the need for a static menu. In
    addition, two entries in the normal definition part are amended (from what
    we used for an application sub-{}window without a menu -{} see last time) to
    point to the:
\begin{itemize}[itemsep=0pt]

\item{}User defined setup routine, or zero if not required.


\item{}User defined drawing routine, or zero if not required.

\end{itemize}

The new style application window definition is as follows:

\begin{lstlisting}[firstnumber=1,caption={AppMenuTest1Win\_asm - Application Window Definition}]
; **** Application window 0 definition.

appw0    dc.w      208      ; Width in pixels (+ scaling)
         dc.w      104      ; Height in pixels (+ scaling)
         dc.w      2        ; X origin, relative to 0 in main window
         dc.w      18       ; Y origin, relative to 0 in main window
         dc.w      256      ; Flag - bit 7 set = clear window
;                           ;      - bit 1 set = disable cursor keys
         dc.w      1        ; Border width
         dc.w      0        ; Border colour
         dc.w      7        ; Paper colour
         dc.w      0        ; Pointer to pointer sprite, or 0 for arrow
\end{lstlisting}

The first part is exactly as we used last time, nothing different to
    see here. Following the above, we have this:

\begin{lstlisting}[firstnumber=last,caption={AppMenuTest1Win\_asm - Application Window Definition}]
; Note the following for menus.

         dc.w      asmnu0-* ; User defined setup routine, or 0
         dc.w      adraw0-* ; User defined drawing routine, or 0
         dc.w      ahit0-*  ; Application window hit routine
         dc.w      0        ; Control routine, or 0
         dc.w      0        ; Max X control sections (splits)
         dc.w      0        ; Max Y control sections (splits)
         dc.b      9        ; Selection key
         dc.b      0        ; Spare byte - must be 0
\end{lstlisting}

The first two entries in the above definition are the new ones.
    These are our pointers to a user defined setup routine and a user defined
    drawing routine. You will notice that the application window still has
    its own hit routine, even though it contains a menu and each and every
    menu item has a dedicated hit routine of its own. Note also, in this small
    example,that our settings for the pan and scroll sections are all unused.
    We'll come back to those in a future chapter.

The user defined setup code would normally consist of a single line
    as follows:

\begin{lstlisting}[firstnumber=1,caption={AppMenuTest1Win\_asm - Application Window Setup Routine}]
asmnu0  jmp wm_smenu(a2)    ; Vector $08
\end{lstlisting}

Similarly, the user defined drawing routine need only perform the
    following tasks:

\begin{lstlisting}[firstnumber=last,caption={AppMenuTest1Win\_asm - Application Window Drawing Routine}]
adraw0  jmp wm_index(a2)    ; Vector $34
        bne.s adexit        ; Bale out on errors
        jmp wm_mdraw(a2)    ; Vector $20
adexit  rts        
\end{lstlisting}

The call to \pe{WM\_INDEX} is not required unless your menu has been
    defined to have sections and/or index items\footnote{George Gwilt has discovered that anything to do with these index items is not actually implemented in the WMAN code. Looks like Tony Tebby had a good idea, that couldn't be fulfilled.}. What are index items? Think
    of a spreadsheet, each row has a number and each column has a letter.
    These are the index items. Our example is not using index items, however,
    if it did then we would set them up exactly as per the menu items list,
    except, for indexes the list entries have no hit routine (set to zero) and
    the item number is always -{}1.

\begin{note}
If the pointer to the user defined drawing routine is zero, then
 WMAN\program{WMAN} will still draw the application
      sub-{}window's border and, unless the flag is set to say `do not clear',
      will clear it to the defined paper colour. If you find missing menus in
      your application sub-{}windows, check that you have a drawing
      routine!
\end{note}

Following on from the above, there is a brand new section dedicated
    to the menu.

\begin{lstlisting}[firstnumber=23,caption={AppMenuTest1Win\_asm - Application Window Menu Area Definition}]
; The following section is required when an application sub-window
; contains a menu.

         dc.w      mst0-wst0 ; Pointer to menu status area. (See text)

         dc.w      1        ; Current Item, border width
         dc.w      0        ; Current Item, border colour

         dc.w      7        ; Unavailable background colour
         dc.w      255      ; Unavailable ink colour
         dc.w      0        ; Unavailable blob pointer
         dc.w      0        ; Unavailable pattern pointer

         dc.w      7        ; Available background colour
         dc.w      0        ; Available ink colour
         dc.w      0        ; Available blob pointer
         dc.w      0        ; Available pattern pointer

         dc.w      4        ; Selected background colour
         dc.w      0        ; Selected ink colour
         dc.w      0        ; Selected blob pointer
         dc.w      0        ; Selected pattern pointer

         dc.w      5        ; Number of columns in the menu
         dc.w      2        ; Number of rows in the menu
         dc.w      0        ; X offset to start of menu
         dc.w      0        ; Y offset to start of menu
         dc.w      spls0-*  ; Pointer to column spacing list
         dc.w      spls1-*  ; Pointer to row spacing list
         dc.w      0        ; Pointer to column index list
         dc.w      0        ; Pointer to row index list
         dc.w      drow0-*  ; Pointer to menu row list
\end{lstlisting}

The first new entry we need is a pointer to the menu items status
    area. This has been defined for us, by SETW\program{SETW}, at
    label mst0. There should be a single byte for each menu item. Note
    however, that we need to have this status area pointer defined as relative
    to the window status area. Hence the calculation in the above
    definition.

\begin{note}
This fact is not very clearly documented in the PE documentation.
      I had an extended conversation with George on this setting as I had
      never seen the fact that the menu status area pointer is relative to the
      window status area -{} George had a pencilled in note in his copy of the
      documentation indicating this need. Obviously, I didn't.
\end{note}

Next up, we see the menu attributes -{} border width and colour, item
    paper and ink, blobs and patterns for unavailable, available and selected
    items.

After the attributes section, we define the menu itself with details
    of how many columns there are, how many rows, offsets to the start of the
    menu and the pointers to the various sections discussed above.

\section{Application Sub-{}Window Menu Item Hit Routines}
\label{ch29-app-menu-item-hit}%hyperlabel{ch29-app-menu-item-hit}%

In addition to the application sub-{}window's own hit routine, as
    described previously, each and every item in the menu (Static or dynamic)
    may also have a hit routine. This routine could be a single one for all,
    or a separate one for each menu item. It depends on how the program is
    designed.

\begin{note}
Whenever a program has a static or dynamic menu, there
 \emph{must} be a hit routine for the application
      sub-{}window containing the menu. The absolute minimum code in the hit
      routine is as follows:

\begin{lstlisting}[firstnumber=1,caption={AppMenuTest1Win\_asm - Application Window Hit Routine}]
ahit0   jmp wm_hit(a2)            ; Vector $34
\end{lstlisting}

If you do not have the above code present in the hit routine for
      the application sub-{}window, then when you attempt to hit or do a menu
      item, nothing will work. The above code does not need an \opcode{RTS}.
\end{note}

On entry to a menu item hit routine various registers are set with
    specific parameters as described in \tablename~\ref{tab:MenuItemHitRoutineRegisters}.


\begin{table}[htbp]
\centering
\begin{tabular}{l p{0.8\textwidth}}
\toprule
\textbf{Register} &\textbf{Description}  \\
\midrule
%
D1.L & Virtual column/row for the hit menu item.\\
D2.W & Item number.\\
D4.L & An event number. This can only be 0 or pt\_\_do (16).\\
A0.L & Channel id.\\
A1.L & Pointer to the menu status area.\\
A2.L & WMAN vector.\\
A3.L & Pointer to sub-window definition.\\
A4.L & Pointer to window working definition.\\
%
\bottomrule
\end{tabular}
\caption{Menu Item Hit Routine Registers}
\label{tab:MenuItemHitRoutineRegisters}
\end{table}

Registers not mentioned above are free for use as they are not used
    by the hit routine.

Hit routines should exit with D5 -{} D7, A0 and A4 preserved to the
    same value that they had on entry to the routine. D1 -{} D3, A1 -{} A3, A5 and
    A6 are undefined on exit (which means that they don't care what value they
    have.) D4.B must be either zero or a window event to be set on
    exit.

D0 should contain zero or an error code and the SR must be set
    according to the value in D0 on exit.

\begin{note}
D3, on return from a hit routine, should normally be returned as
      per its value on entry. It is not used by \pe{WM\_RPTR} however, it is used by
      \pe{WM\_RPTRT} (read pointer with return on timeout) from
 WMAN\program{WMAN} 1.5 onwards. Wm\_rptr ignores the upper
      word of D3. If your read pointer loop is using the \pe{WM\_RPTRT} vector
      instead, and you have changed the value of D3 within the hit code, you
      must clear the high word on exit.
\end{note}

On exit, if D0 is clear and the status (Z) bit is set, control will
    return to the \pe{WM\_RPTR} loop and not to your application's code. To return
    to your own code, the hit routine needs to set at least one event bit in
    the event vector which can be done by returning a suitable value in D4.B
    on exit.

If an error is detected within the hit code, then it should exit
    with the appropriate error code in D0 and the status register set
    accordingly.

\section{Coming Up...}
\label{ch29-the-end}%hyperlabel{ch29-the-end}%

So, that's the end of this exciting chapter. We have designed a
    window and looked deep into the structures involved in defining a static
    menu. 
    
In the upcoming chapter we'll add some code and play around. We might even see if
    it's possible to take the design from SETW\program{SETW} and
    massage it to suit our own needs and considerations.

