\chapter{The 6800x Instruction Set}

\section{Introduction}
\label{ch2-intro}%\hyperlabel{ch2-intro}%

In part one, we learned some really boring stuff. Address modes are
    not what I would call interesting reading, and I suppose that most of you
    who are still reading this, would agree.

At this point, however, it gets worse. We are now going to delve
    into the instruction set of the processor.

\section{The MOVE Instruction Family}\mc6800x{MOVE}
\label{ch2-move}%\hyperlabel{ch2-move}%

The most common instruction in the entire world, is probably the
    \opcode{MOVE} instruction. It is actually wrongly named as it really does a COPY
    rather than a MOVE. The format of the \opcode{MOVE} instruction is:


\begin{lstlisting}[firstnumber=1,]
          MOVE    source,destination
\end{lstlisting}


or:


\begin{lstlisting}[firstnumber=1,]
          MOVE.size    source,destination
\end{lstlisting}


The data in source is copied to the destination. For
    example,
\begin{lstlisting}[firstnumber=1,]
          MOVE    D0,D1
\end{lstlisting}
takes whatever data
    is in data register 0 (zero) and copies it into data register 1. How much
    data is moved? In this case, because no size is specified, the default
    size is always WORD so a single word of data is moved from D0 to D1. As
    there is space for 2 words in each of these registers, which word is moved
    from D0 to which word in D1?

All instructions work from the `lowest' end of the register towards
    the highest (with the exception of \opcode{MOVEP} -{} see below). So, in the above
    example, the lowest 16 bits of D0 are copied to the lowest 16 bits of D1.
    The data in D0 is not altered in any way whatsoever. The same cannot be
    said for D1 as the original data in D1 has been replaced -{} but only the
    lowest 16 bits. The highest word has not been altered.

If D0 contained \$01020304 and D1 contained \$11223344 then after the
    above move, D0 would be unchanged and D1 would contain
    \$1122\emph{0304}. 
    
If the size of the instruction had been
    specified, as follows:
\begin{lstlisting}[firstnumber=1,]
          MOVE.B    D0,D1
\end{lstlisting}
     Then only the lowest byte of D1 would have been altered. In this case D1
    would have contained \$112233\emph{04} after the move. If the
    size specifier had been `L' for LONG than the entire 4 bytes in D1 would
    have been overwritten by the 4 bytes from D0. After a long sized \opcode{MOVE},
    both D0 and D1 would contain \$01020304.

Because the move takes place into a data register the condition
    codes are affected. To copy data into an address register use the \opcode{MOVEA}
    instruction, but always remember that it does not affect the flags in the
    condition code register.

The changes that will take place every time a data register or
    memory location is used as the destination for a \opcode{MOVE} are:
\begin{itemize}[itemsep=0pt]

\item{}X flag is never affected. It remains as it is.


\item{}N flag is set if the data moved was negative. If the data was
        positive, N is cleared.


\item{}V is always cleared. You cannot move a value into a register
        that causes an overflow.


\item{}C is always cleared for similar reasons.


\item{}Z is set if the data moved was zero. It is cleared if it was any
        other value.

\end{itemize}

The \opcode{MOVE} instruction has many variations, most of them simple and
    easy to understand. These are:

\opcode{MOVE} as described above.

\opcode{MOVE CCR}\mc6800x{MOVE CCR} -{} this is the MOVE \emph{to} CCR instruction which appears on the standard QL's 68008 processor.
There is another variant, the MOVE \emph{from} CCR which does not exist on that CPU. the size is always word although the upper 8 bits are
    ignored -{} effectively a byte sized move. The format of the two instructions are, but only the first is available on a standard QL:

\begin{lstlisting}[firstnumber=1,]
          MOVE    source,CCR
          MOVE    CCR,destination
\end{lstlisting}

Executing this instruction results in the condition codes being set
    as follows:
\begin{itemize}[itemsep=0pt]

\item{}X is set to bit 4 of source


\item{}N is set to bit 3 of source


\item{}Z is set to bit 2 of source


\item{}V is set to bit 1 of source


\item{}C is set to bit 0 of source

\end{itemize}

All the other bits are simply ignored.

\opcode{MOVE SR}\mc6800x{MOVE SR} -{} the size is always word and may not be specified in the
    instruction. This instruction copied the 16 bits of the condition code
    register to the destination. The instruction format is:

\begin{lstlisting}[firstnumber=1,]
          MOVE    SR,destination
\end{lstlisting}

When the instruction has been carried out, the lower 16 bits of the
    destination contain a copy of the Status Register of the processor. The
    actual data in the status register is unaffected by the move.

There is a complimentary instruction to move data into the status
    register which is:

\begin{lstlisting}[firstnumber=1,]
          MOVE    source,SR
\end{lstlisting}

Which takes the lower 16 bits of the source data and copies it into
    the status register. The lower 8 bits are used to change the flags in the
    CCR or Condition Codes Register (See \opcode{MOVE CCR} above). The SR is affected
    according to the lower 16 bits of the source data as follows:
\begin{itemize}[itemsep=0pt]

\item{}T is set to bit 15 of source


\item{}S is set to bit 13 of source


\item{}III is set to bits 10, 9 and 8 of source


\item{}X is set to bit 4 of source


\item{}N is set to bit 3 of source


\item{}Z is set to bit 2 of source


\item{}V is set to bit 1 of source


\item{}C is set to bit 0 of source

\end{itemize}

The other bits are simply ignored. There is a slight problem, the
    instruction \lstinline{MOVE source,SR} must be executed in Privileged mode or it will
    cause a `Privilege Violation Exception' which on a normal QL will simply
    lock it up. (Exceptions are covered later on in the series.)

\begin{warning}
Note: on the 68010 and up, the \lstinline{MOVE SR,destination} becomes a
      privileged instruction. There is a new instruction \lstinline{MOVE CCR,destination}
      which allows access to the CCR part of the SR. Programs written for the
      68000 and 68008 may require to be re-{}written with this in mind.
\end{warning}

\opcode{MOVE USP}\mc6800x{MOVE USP} -{} A long sized instruction which copies data into the User
    Stack Pointer (USP) also knows as A7. This instruction is also privileged
    and requires that the system is running in supervisor mode. The format of
    the instruction is:

\begin{lstlisting}[firstnumber=1,]
          MOVE    source,USP 
          MOVE    USP,destination
\end{lstlisting}

Both source and destination must be an address register. None of the
    condition codes are affected by this instruction.

Why does this have to be run in supervisor mode? Well, if not, a
    privilege violation exception will be generated and these instructions
    allow the operating system to set the value of a job's stack
    pointer.

If you remember, there are two A7 registers, one used for supervisor
    mode and the other for user mode. Only one can be in use at any one time.
    This instruction allows the supervisor to set the USP without affecting
    its own version of the A7 register. Not used much, if at all on the
    QL.

\opcode{MOVEA}\mc6800x{MOVEA} -{} the contents (remember that word!) of the source is moved
    into an address register. This instruction is either word or long sized
    and does not affect the condition codes. The format is:

\begin{lstlisting}[firstnumber=1,]
          MOVEA.size    source,An
\end{lstlisting}

Beware because if you move a word sized source, it will be sign
    extended to long (bit 15 will be copied into bits 16 to 31) before the
    data is copied into the address register.

For example:

\begin{lstlisting}[firstnumber=1,]
          MOVEA.W    #$0001,A0
\end{lstlisting}

This will set A0 to \$00000001 after the move. Bit 15 of the data is
    a zero so this is copied into all the upper 16 bits of A0. The lower 16
    bits are simply a direct copy of the data.

\begin{lstlisting}[firstnumber=1,]
          MOVEA.W    #$8000,A0
\end{lstlisting}

This will set A0 to \$FFFF8000 after the move. Bit 15 is a one and
    this is copied into all the upper 16 bits of A0. The lower 16 are again a
    copy of the data.

Don't forget about sign extension!

\opcode{MOVEM}\mc6800x{MOVEM} -{} a word or long sized instruction which allows you to copy
    data to or from a number of registers in a single instruction. The format
    of the instruction is:

\begin{lstlisting}[firstnumber=1,]
          MOVEM    register_list,destination
          MOVEM    source,register_list
\end{lstlisting}

None of the condition codes are affected by this instruction.

The instruction is most often used to store a number of registers on
    the stack on entry to a subroutine, and to reinstate the original values
    on exit from the subroutine. The instruction stores the registers starting
    with D0, then D1 and so on up to D7, then the address registers are stored
    in order from A0 to A7 -{} assuming all registers are specified.

A register list takes the format of a starting register name, a
    hyphen then a finish register name. Another form is a start register name
    a slash and another register name. The two formats can be mixed to give
    almost endless possibilities. The following are all register list examples
   :

\begin{lstlisting}[frame=none,numbers=none]
D1-D4
A0-A3
D1/D4-D7
D0-D2/D4/D7/A0-A3/A6
\end{lstlisting}

The hyphen means that all registers from the starting one to the
    finish one (inclusive) will be moved to the destination. The slash signals
    that there \emph{might be} a `gap' in the register list, but not necessarily so. The above examples mean:

\begin{lstlisting}[frame=none,numbers=none]
D1 and D2 and D3 and D4
A0 and A1 and A2 and A3
D1 and D4 and D5 and D6 and D7
D0 and D1 and D2 and D4 and D7 and A0 and A1 and A2 and A3 and A6.
\end{lstlisting}

And finally, this explains why the slash only signifies that there \emph{might} be a gap:

\begin{lstlisting}[frame=none,numbers=none]
D1/D2/D3/D4
\end{lstlisting}


The list can be specified in any order (unless the assembler rules
    differently) as each register detected is used to set a single bit in a 16
    bit word. This word is used by the processor to determine which of the
    registers are to be copied.
    
    Regardless of the order in which the registers are specified in the instruction, they are always stored in memory in order from D0 to D7 then A0 to A7 - for all the registers that are specified that is.

This instruction will be most often used in its Post decrement and
    pre-{}increment forms:

\begin{lstlisting}[firstnumber=1,]
          MOVEM.L    D0-D3,-(A7) 
          MOVEM.L    (A7)+,D0-D3
\end{lstlisting}

You cannot use the post-increment addressing mode when moving register data into memory, and likewise, when moving memory data into registers, you cannot use the pre-decrement addressing mode.

\begin{warning}
It should be noted, that when moving \emph{word} sized data from memory into registers, that these words will be \emph{sign extended to a long word} prior to being stored in the register.
\end{warning}


\opcode{MOVEP}\mc6800x{MOVEP} -{} Probably the strangest instruction in the 68000 set. It is used for 8 bit peripherals on a 16 bit data bus.

This
    instruction transfers data from a data register to alternating bytes in
    memory. The data is transferred from the data register starting from the
    highest 8 bits, then the next 8 bits and so on. This is a word or long
    sized instruction. The condition code flags are not affected. (I have
    never used or seen this instruction used on the QL.) The formats are
   :

\begin{lstlisting}[firstnumber=1,]
          MOVEP.size    Dn,displacement(An)
          MOVEP.size    displacement(An),Dn
\end{lstlisting}

The size is long or word, Dn is any data register, An is any address
    register and the displacement is added to the address register to get the
    first address to be filled with data. An example might make things
    clearer. If we assume that D0 holds \$11223344 and A1 holds the address
    \$000200000 then the instruction:

\begin{lstlisting}[firstnumber=1,]
          MOVEP.L    D0,0(A1)
\end{lstlisting}

Copies the highest byte of D0 (\$11) into address \$20000, the next
    highest (\$22) into address \$20002, the next byte (\$33) into address \$20004
    and finally the lowest byte of D0 (\$44) into address \$20006. Addresses
    \$20001, \$20003 and \$20005 are not affected.

Had the displacement and A1 combined created an odd address then the
    odd addresses would have been filled with data and the even ones would not
    have been affected.

\opcode{MOVEQ}\mc6800x{MOVEQ} -{} This is a very useful instruction and you will see it used
    on many occasions in QL assembly language programs. It is the `Move Quick'
    instruction and is used to quickly move any value between -{}128 and 127
    into any data register. The value is sign extended to 32 bits or long
    sized and so fills the entire data register. The format is:

\begin{lstlisting}[firstnumber=1,]
          MOVEQ    #data,Dn
\end{lstlisting}

The flags are affected by this instruction as follows:
\begin{itemize}[itemsep=0pt]

\item{}X flag is never affected. It remains as it is.


\item{}N flag is set if the data moved was negative. If the data was
        positive, N is cleared.


\item{}V is always cleared. You cannot move a value into a register
        that causes an overflow.


\item{}C is always cleared for similar reasons.


\item{}Z is set if the data moved was zero. It is cleared if it was any
        other value.

\end{itemize}

Remember, only 8 bit values are allowed and these must be between
    -{}128 and 127.

A number of 68000 instructions have this `quick' mode, but why is it
    quick? Let us compare the \lstinline{MOVEQ #0,D0} with its equivalent \lstinline{MOVE.L #0,D0}.
    We simply see two different forms of what is effectively the same
    instruction, the QL's processor sees things a bit differently, as follows
   :

First \lstinline{MOVEQ #0,D0} is a 16 bit instruction in memory. \lstinline{MOVE.L #0.D0} is
    also a 16 bit instruction but it is followed in memory by a long word (32
    bit) holding the data, in this case zero. This makes the \opcode{MOVEQ} instruction
    3 times smaller than the \opcode{MOVE.L} one. As the processor has less data to
    fetch from memory, it takes less time to read the instruction and its
    data, therefore it is quicker. Looking at the 68008 timing chart, it takes
    the \opcode{MOVEQ} instruction 8 clock cycles to execute and the \opcode{MOVE.L} 24 clock
    cycles.

And that is about it for the 68008's \opcode{MOVE} instructions. This is
    probably the instruction with the most variants and as I said before,
    probably the most used instruction in any program.

\subsection{Exercise}
\label{ch2-exercise-1}%\hyperlabel{ch2-exercise-1}%

1. Write down the correct instruction which will copy 4 bytes of
      data from address \$20000 into data register D7.

2. What is the fastest way to get the 8 bit value of 17 into all
      32 bits of register D2?

3. What instruction would you use to copy the lowest 16 bits of
      register D1 into the lowest 16 bits of register D3? What happens to the
      data in D1 after the move and what happens to the data that is currently
      held in D3?

4. How would you place the lowest byte of D1 into a memory
      location which is 10 bytes further on from the address currently held in
      A0?

5. Why is the \opcode{MOVE} instruction `wrongly' named?

6. What does a privileged instruction require before it can be
      executed?

7. What happens if a privileged instruction is executed in user
      mode?

8. How many data registers does the 68008 have and how many
      address registers?

9. What values are set in each of the condition codes when the
      instruction \lstinline{MOVEQ #0,D1} is executed?

10. What values are set if the instruction executed was \lstinline{MOVEA.L #0,A0}?

\subsection{Answers}
\label{ch2-ch2-answers-1}%\hyperlabel{ch2-ch2-answers-1}%

1. MOVE.L \$20000,D7

2. MOVEQ \#17,D2 or MOVEQ \#\$11,D2

3. MOVE.W D1,D3. Nothing happens to the data in D1. The highest
      word on D3 is not affected but the lower word is overwritten by the
      lowest word from D1.

4. MOVE.B D1,10(A0) or MOVE.B D1,\$0A(A0).

5. The MOVE instruction actually copies data from source to
      destination, it does not move it in the traditional sense of `it was
      over there but it has been moved to over there'.

6. The processor must be in supervisor mode.

7. A privilege exception will be generated (and the QL will
      probably hang).

8. There are 8 data registers and 9 address registers but only one
      of the A7 `twins' can be used at a time.

9. The Z flag is set to one and all the rest are reset to zero
      except the X flag which is unaffected and keeps its previous
      value.

10. No flags are changed. They all keep their previous
      values.

\section{The CMP Instruction Family}\mc6800x{CMP}
\label{ch2-cmp}%\hyperlabel{ch2-cmp}%

While all this talk of moving data around, be it in memory or within
    the processor's internal registers, is `interesting', being able to move
    data is not much use if you cannot do anything with it when you have moved
    it. As the condition codes are affected by data movements we can sometimes
    determine the value of the data we moved. This is of course true only if
    we want to know if the value we moved was zero, or not zero, positive or
    negative but that's about as accurate as we can get using the \opcode{MOVE}
    instruction.

If we need to compare two values we will need to use the \opcode{CMP} family
    of instructions. \opcode{CMP} stands for `Compare' and allows data to be compared
    against specific values, registers or memory contents.

The general format of the \opcode{CMP} instruction is:

\begin{lstlisting}[firstnumber=1,]
          CMP.size    source, Dn
\end{lstlisting}

The \opcode{CMP} instruction has the effect of carrying out a subtraction of
    the source from the destination register without changing the destination at all. What it does change is the condition codes, and these will be set as follows
   :
\begin{itemize}[itemsep=0pt]

\item{}X flag is never affected. It remains as it is.


\item{}N flag is set if the result was negative. If the result was
        positive, N is cleared.


\item{}V is set if the result caused an overflow otherwise
        cleared.


\item{}C is set if a `borrow' was generated and cleared
        otherwise.


\item{}Z is set if the result was zero. It is cleared if it was any
        other value.

\end{itemize}

This instruction can be carried out in all three sizes -{} byte, word
    or long, however, if the source is an address register, only word and long sizes are allowed.

One of the common uses of this instruction, and perhaps the easiest
    to understand, is testing to see whether two values are the same. If they
    are then the result of the `subtraction' of source from destination will
    always be zero. If the result is zero then the Z flag can be tested
    (somehow -{} we shall see later) and then some actions taken if it is set
    while others can be taken if it is not set.

The instruction:

\begin{lstlisting}[firstnumber=1,]
          CMP.L    D1,D2
\end{lstlisting}

Will set the Z flag if the same value is present in both D1 and D2.
    If they are different, then the Z flag will not be set.

There are only four variations of the \opcode{CMP} instruction -{} unlike \opcode{MOVE}
    which has a few more. The first is simply \opcode{CMP} itself. This is used when
    comparing with a data register as in the above example. The source,
    however, can be any of the 68000 addressing modes -{} although you cannot
    compare an address register and a data register using the BYTE size. This
    means that:

\begin{lstlisting}[firstnumber=1,]
          CMP.W    A0,D2 
\end{lstlisting}

is a legal instruction, but that:

\begin{lstlisting}[firstnumber=1,]
          CMP.B    A0,D2
\end{lstlisting}

Is not. It is of course allowed that the data be POINTED to by an
    address register, as in:

\begin{lstlisting}[firstnumber=1,]
          CMP.B    0(A0),D2
\end{lstlisting}

Which compares the byte of data at the address held in A0 with the
    byte of data held in the lowest byte of register D2.

\opcode{CMPA}\mc6800x{CMPA} -{} is the form of the instruction used when comparing against a
    destination which is an address register. It is very similar to the \opcode{CMP}
    variation, but only word and long sized comparisons can be made. If the
    word size is used, then watch out for the old favourite pitfall of sign
    extension. Whatever word sized data is used for the source of this
    comparison will be sign extended up to a long word and then compared with
    the entire 32 bits of the address register.

This means that:

\begin{lstlisting}[firstnumber=1,]
          CMPA.W    #$FFFF,A3
\end{lstlisting}

Would set the Z flag if and only if A3 contained the value of
    \$FFFFFFFF but would not set it if A3 contained the value \$0000FFFF.
    Beware. If at all possible, make your code explicit. So if you want to
    test A3 as having \$FFFF in its lower word, use \lstinline{CMPA.L \#\$FFFF,A3} instead of
    the word sized version.

\opcode{CMPI}\mc6800x{CMPI} -{} is the third variation and this one is used when testing any
    address mode destination (except PC relative or an address register's
    contents) against source data which is, quite simply, a number. This
    variation can be used in all 3 sizes. The format of the instruction is
   :

\begin{lstlisting}[firstnumber=1,]
          CMPI.size    #data,destination
\end{lstlisting}

If the destination is a data register, then the instruction is
    equivalent to the \opcode{CMP} instruction.

\opcode{CMPM}\mc6800x{CMPM} -{} is the final variation. It is used to compare one memory
    location with another. It can be used in all 3 sizes but can only be used
    in a single address mode -{} address register with post-increment. The format
    is always:

\begin{lstlisting}[firstnumber=1,]
          CMPM.size    (An)+,(An)+
\end{lstlisting}

The two address registers are pointers to the memory addresses to be
    compared and after this instruction, the flags have been set according to
    the result of the `subtraction' while both address registers have been
    incremented by 1, 2 or 4 depending upon the size of the data being
    compared.

\section{Signed and Unsigned Numbers}
\label{ch2-signed-unsinged}%\hyperlabel{ch2-signed-unsinged}%

Before we take a closer look at the condition codes and how we can
    use them to alter the flow of a program -{} that is, how we can implement
    loops, if then else etc, we need to take a break and discuss the
    differences between signed and unsigned numbers.

When we \opcode{MOVE} some data into a data register the same number can
    actually mean two different things. Confused? You will be!

If we use an 8 bit number as an example, the data \$FF can either
    mean 255 or minus one. In a 16 bit example, \$FFFF can mean either 65535 or -{}1 and
    in a 32 bit long word, \$FFFFFFFF means either $2^{32}-1$ or $-{}1$. 
    
The important thing to
    remember is that it is \emph{you}, the programmer, who decides which version is
    in use at any particular time.

Ok, how does it work? The 68000 family of processors can use signed
    or unsigned numbers. If the signed version is in use then the number will
    be either negative (less than zero) or positive (zero or greater). If
    unsigned numbers are being used then the value will always be positive.
    How can the processor tell the difference?

The answer to the question `is this number signed or unsigned?' is
    either `yes' or `no' equivalent to one or zero in binary terms. This
    implies that a single bit can be used to hold the sign of the number and
    this is exactly how it happens. By convention the most significant bit of
    the number holds the sign. A one indicates that the number is negative
    while a zero indicated that it is not.

Those of you who are thinking ahead of me now might well be saying
    `but surely using a single bit of the register will reduce the amount of
    numbers that can be represented by a factor of two?'. Not quite.

In binary, the numbers representing the hexadecimal values \$00 to
    \$0F will all fit into a half byte or nibble. A nibble is 4 bits and each
    bit represent a single power of two in the number.

Just as 1231 means $(1 * 10^{3}) + (2 * 10^{2}) + (3 * 10^{1}) + (1 * 10^{0})$, which is, $(1 * 10 * 10 * 10) + (2 * 10 * 10) + (3 * 10) + (1 * 1)$
    which is, $1000 + 200 + 30 + 1$ which is the number we have at the start of
    all this, the same is true in binary.

The binary nibble 1010 is $(1 * 2^{3}) + (0 * 2^{2}) + (1 * 2^{1}) + (0 * 2^{0})$, which is $(1 * 2 * 2 * 2) + (0 * 2 * 2) + (1 * 2) + (0 * 1)$, which is
    $8 + 0 + 2 + 0$, which is 10 in decimal with converts to \$0A in hexadecimal.

All the possible values that can be held in an unsigned nibble are
    0000 (zero) up to 1111 (15 or \$0F) and conversion is a matter of adding up
    each power of two in the number. From the right we have $2^{0}$ which is simply one. Then $2^{1}$ or two and so on.

In an unsigned nibble the most significant bit $(2^{3})$ is used to hold
    the sign, so all numbers below unsigned 7 are positive while those `above' 7 are actually negative and so are below 0.

If the highest bit was not the sign bit it would represent $2^{3}$ or 8.
    To convert into a signed value simply negate the 8 to get minus 8, and add
    all the other bit values to it. Taking the same binary example of 1010 as
    above, this is now
$(-{}1 * 2^{3}) + (0 * 2^{2}) + (1 * 2^{1}) + (0 * 2^{0})$. This 
    gives minus 8 plus 2 which is minus 6. This implies that for a signed
    number the range is minus 8 to plus 7 which is still a possible 16 values as with
    the unsigned version, just shifted slightly down the number scale.

That is the only difference between signed and unsigned numbers. The
    ranges of values in a byte are minus 128 to plus 127, in a word it is
    minus 32768 to plus 32767 and for a long word it is minus 2147483648 to
    plus 2147483647.

When dealing with signed numbers any number which has a 8, 9, A, B,
    C, D, E or F in the most significant digit (hex that is) is negative. All
    the rest are positive. I find the quickest way to find the equivalent
    negative value is to subtract from $2^{bits}$. For example $-1$ in an 8 bit byte is $2^{8}-1$ which is $256 -1$ which is $255$. $255$ in hex is \$FF which is
    the 8 bit representation of $-1$. Similarly, $-10$ is $256 -10$ = 146 which is
    \$F6. Use 65,536 for 16 bit words and 4,294,967,296 for 32 bit long words.

Enough for now. Just remember when coding a program in assembler
    that numbers can be two different values at the same time. You determine
    which one is appropriate at any one time. It is far easier to consider
    unsigned numbers all the time but this might not be applicable. Writing a
    program to record the number of sheep jumping over a fence need never use
    signed numbers, while the amount of money in your bank account probably
    will. Just remember to be consistent.

\section{Testing Condition Codes and Branching}\mc6800x{Bcc}
\label{ch2-testing-branching}%\hyperlabel{ch2-testing-branching}%

As you may remember when data is \opcode{MOVE}d into a \emph{data} register or
memory address, certain condition codes are set or unset. These codes can be used, along
with the results of a \opcode{CMP} instruction and/or the discussion of signed and unsigned numbers
above, to determine program flow. To change the flow, we use the branch instruction also
known as \opcode{Bcc} or Branch on condition code. The general format of a \opcode{Bcc} instruction
is:

\begin{lstlisting}[firstnumber=1,]
          Bcc    label 
\end{lstlisting}

The label part defines where the branch will be to (the destination)
    and is an offset from the current program counter and of course may be
    positive or negative.

A branch instruction is equivalent to a SuperBasic GOTO command. Much frowned upon
by purists, but useful in certain situations. Never say `Never use a GOTO' because in
assembly language you almost always have one!

There are a number of `branch' instructions that look at the
    condition codes and change the course of your program according to what
    they find. There are 14 of these and some appear remarkably similar to
    others. They are listed in Table~\ref{tab:BranchOnConditionInstructions}:

\begin{table}[htbp]
\centering
\begin{tabular}{l l l p{0.4\textwidth}}
\toprule
\textbf{Branch} & \textbf{Name} & \textbf{Signed/unsigned} & \textbf{Description}\\
\midrule
BCC & Branch Carry Clear      & Unsigned & The branch is executed if the carry flag is not set -{} ie zero.\\
BCS & Branch Carry Set        & Unsigned & The branch is executed if the carry flag is set -{} ie one.\\
BEQ & Branch Equal            & Both     & Branch only if the result of the last operation caused the zero flag to be set. MOVEQ \#0,D0 for example.\\
BGE & Branch Greater or Equal & Signed   & Branch if the last operation resulted in a signed number that was zero or greater.\\
BGT & Branch Greater Than     & Signed   & Branch if the last result was greater that zero.\\
BHI & Branch Higher           & Unsigned & Branch if the last result was greater than zero.\\
BLE & Branch Less or Equal    & Signed   & Branch if the last result was zero or less.\\
BLS & Branch Lower or Same    & Unsigned & Same as BLE, but `equal' replaced by `same'.\\
BLT & Branch Less Than        & Signed   & Branch only if the last result was less than zero.\\
BMI & Branch Minus            & Signed   & Branch if the result of the last operation was negative. Ie less than zero but not including zero.\\
BNE & Branch Not Equal        & Both     & Branch if the last operation resulted in a non-{}zero outcome. CMPI.L \#1,D1 if D1.L is not holding the value 1.\\
BPL & Branch Plus             & Signed   & Branch if the result of the last operation is positive ie zero or greater.\\
BVC & Branch oVerflow Clear   & Both     & Branch if the last operation left the V flag unset.\\
BVS & Branch oVerflow Set     & Both     & Branch if the last operation left the V flag set.\\
\bottomrule
\end{tabular}
\caption{Branch on condition instructions.}
\label{tab:BranchOnConditionInstructions}
\end{table}

There is one more branch instruction that does not care about the
    flags, this is the \opcode{BRA} or Branch unconditionally instruction. It is the
    most like a GOTO instruction as that is its exact purpose -{} goto some
    other place in the program.

If the displacement value will fit into a single byte (-{}128 to +127)
    then a `short' branch will take place. This entire instruction fits into a
    single word. If the displacement is zero, then this would normally
    indicate a short branch to the next instruction in the program. As this is
    where the PC is pointing anyway the zero displacement is used to signify a
    long branch and the word following is used as a 16 bit displacement
    allowing relative values between -{}32768 to +32767.

The short branch is written as \opcode{Bcc.S} with the dot and `s' indicating
    the shortness. Most assemblers default to the long branch which adds 2
    bytes to your program for every \opcode{Bcc} instruction in it. I find the `best'
    way to reduce the `wasted' bytes is to make all branches short and the
    assembler will reject those which are out of range.

One of the most confusing aspects of assembly language programming
    for new and experienced coders alike is `which are the signed and unsigned
    tests?' I always have to look it up and I have never found a place where
    all the tests are listed together with the signed and unsigned
    comparisons. You won't have this problem as I have listed them all in Table~\ref{tab:SignedAndUnsignedTests}.

\begin{table}[htbp]
\centering
\begin{tabular}{l l l}
\toprule
\textbf{Test for} & \textbf{Signed} & \textbf{Unsigned}\\
\midrule
Greater Equal & BGE & BCC\\
Greater than & BGT & BHI\\
Equal & BEQ & BEQ\\
Not Equal & BNE & BNE\\
Less Equal & BLE & BLS\\
Less than & BLT & BCS\\
Negative & BMI & Not applicable\\
Positive & BPL & Not applicable\\
\bottomrule
\end{tabular}
\caption{Signed \& Unsigned Tests.}
\label{tab:SignedAndUnsignedTests}
\end{table}


In the above description of the \opcode{Bcc} instructions I state, for
    example, that the \opcode{BNE} instruction will branch if the last result was not
    zero. This is not quite the case. If I had just loaded a data register
    with some value which was not zero then the branch would be taken, as in
    the following fragment of code:

\begin{lstlisting}[firstnumber=1,]
          MOVE.L    (A0),D1 
          BNE.S     Somewhere 
\end{lstlisting}

If, on the other hand, I was comparing two registers then the branch
    would have been taken if they did not have exactly the same contents
   :

\begin{lstlisting}[firstnumber=1,]
          CMP.L     D3,D4 
          BNE.S     not_equal
          BHI.S     greater
\end{lstlisting}

So you can see that there are more ways to use these conditional
    branches. Bear in mind, however, that the \opcode{CMP} is simply a subtraction with
    the answer `thrown away' and it is that discarded result that is being checked. One
    other area of confusion is which register is greater in the \opcode{BHI}
    instruction above?

In a \opcode{CMP} instruction it should be read as Destination \opcode{CMP} source. If this is
followed by a \opcode{Bcc} then it means branch if the destination is
\emph{condition} source. So in the above code fragment, we will branch to
the label `greater' if and only if D4 is greater than D3.

There are other instructions that affect the flow of a program and
    these are the `looping' constructs or \opcode{DBcc} as they are written. These are
    the `Decrement and branch \emph{until} condition. Confused? All will be revealed
    in the next chapter.

\section{Coming Up...}
\label{ch2-the-end}%\hyperlabel{ch2-the-end}%

In the next chapter we will take a closer
    look at some more branching instructions and start thinking about the project\footnote{The project was to be a disassembler named ``QLTdis''\program{QLTdis}, but it fell dramatically by the wayside and is not discussed further in this book.}.
