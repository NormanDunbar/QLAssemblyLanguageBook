\chapter{Assorted Revisions And Ramblings!}

\section{Introduction}
\label{ch19-intro}%hyperlabel{ch19-intro}%

Greetings from the basement!

We have moved house and are getting settled.\footnote{That was written originally in 2007. Treat with a pinch of salt now!} We have still got a lot of
        boxes to unpack and things to find, but we are getting there. I have a new
        `office' deep down in the basement where it is nice and cool. This is the first in
        the Assembler series to come from the basement.

With all the upheaval of getting moved and unpacked etc, I have not got a
        lot of code for you this time, hopefully, you won't be too bored by this episode
        in which I go over bits and pieces of assembly language programming that causes me
        grief.

It all started when I was having a think the other day about life in general
        and assembly language in particular. I was pondering on the bits of programming in
        assembler that I always get wrong, or have to really think about -{} and still get
        wrong.

\section{SIGNED And UNSIGNED Tests}
\label{ch19-signed-unsigned}%hyperlabel{ch19-signed-unsigned}%

I don't know about you, but I seem to have severe difficulties in
        remembering which are the signed and which are the unsigned tests. I have to
        confess that I always have a list of them written down (or printed out) and stuck
        to my work area -{} wherever that happens to be.

Table~\ref{tab:SignedAndUnsignedTests} is a reminder of the `cc' code to use in a \opcode{Bcc} or whatever for signed and unsigned comparisons:

\begin{table}[htbp]
\centering
\begin{tabular}{l l l}
\toprule
\textbf{Desired Test} & \textbf{Signed} & \textbf{Unsigned} \\
\midrule
%
Greater Equal  & GE & CC  \\
Greater Than   & GT & HI  \\
Equal/Zero     & EQ & EQ  \\
Not Equal/Zero & NE & NE  \\
Less Equal     & LE & LS  \\
Less Than      & LT & CS  \\
Negative       & MI & n/a  \\
Positive       & PL & n/a  \\
%
\bottomrule
\end{tabular}
\caption{Signed and Unsigned Tests}
\label{tab:SignedAndUnsignedTests}
\end{table}

So, if D0.B contains the value \$FF it represents either 255 (unsigned) or -{}1
        (signed). You, as the programmer should know whether the value is considered
        signed or not and can make the correct comparison checks.

The EQ and NE tests are interesting in that they either mean `two values are
        [not] the same' when comparing things such as memory and registers, or two
        registers etc, or, when having just loaded a register with a value, they mean `the
        value just loaded into a data register is [not] zero'.

The following code examples are identical in result, one is just quicker
        than the other:

\begin{lstlisting}[firstnumber=1,]
    MOVE.W (A1),D0
    BEQ.S  D0Zero
    ....
\end{lstlisting}

and

\begin{lstlisting}[firstnumber=1,]
    MOVE.W (A1),D0
    CMPI.W #0,D0
    BEQ.S  D0Zero
    ...
\end{lstlisting}

\section{Which Way Round Is The `Subtraction' In CMP?}
\label{ch19-cmp-way-round}%hyperlabel{ch19-cmp-way-round}%

If I see \lstinline{CMPI.W #1234,D0} then it is obvious, I am comparing D0.W with the
        value 1234. That's easy. However, when I see \lstinline{CMP.W D0,D1} I lose the plot.

What am I comparing here is it D0 with D1 or the other way around. My brain
        hurts already.

Is the value of 1234 subtracted from D0 or is the value in D0 subtracted
        from 1234. Which way round is the subtraction and the resulting setting of
        flags?

The answer, I note from part 2 of this series is that the source register is
        subtracted from the destination register exactly as a \opcode{SUB} instruction would do,
        the result is simply discarded. So in the instruction \lstinline{CMP.W D0,D1} the flags are
        set according to D1.W minus D0.W.

It is assumed that after this pseudo-{}subtraction, some \opcode{Bcc}, \opcode{Scc} or \opcode{DBcc} instruction will no doubt check the flags and do something useful with the
        result.

\section{Which CC Code To Use After CMP}
\label{ch19-cc-code-after-cmp}%hyperlabel{ch19-cc-code-after-cmp}%

Leading on from the above, I never remember which `cc' code to use after a
        \opcode{CMP} -{} although, having written out the above it is becoming clearer. The following
        code gives me the willies time after time:

\begin{lstlisting}[firstnumber=1,]
    ...
    CMP.L D0,D1
    BHI   somewhere
    ...
\end{lstlisting}

This fragment has everything that confuses me, almost. It has a \opcode{CMP} followed
        by a `cc' instruction -{} so I have to think about the two `problem areas' I mention
        above. Signed or unsigned and which register is causing the HI to be true or
        false.

Well, the HI is, from my table above, unsigned and using my new found
        knowledge of the \opcode{CMP} instruction I know (for a short while at least) that the flag
        are set to the result of (D1.L -{} D0.L) but which way around does it go
        again?

The \opcode{BHI} should be read as `branch if destination register HI source
        register' in the preceding \opcode{CMP} or \opcode{SUB} or whatever was used to set the flags. So,
        using this explanation, I now know that the code above branches if D1.L is higher
        (in an unsigned manner) than D0.L.

This leads me to surmise that the following pseudo-{}code:

\begin{lstlisting}[firstnumber=1,]
IF unsigned (D1.L > D0.L) THEN
...
ELSE
...
END IF
\end{lstlisting}

Becomes:

\begin{lstlisting}[firstnumber=1,]
IF    CMP.L D0,D1    Flags = result of D1.L - D0.L
      BHI.s THEN     D1 is indeed (unsigned) greater than than D0
ELSE  ...            D0 is less or equal to D1 (the ELSE bit)
      BRA.s ENDIF    Skip over the THEN clause
THEN  ...            Do the THEN stuff
ENDIF ...            Together again.
\end{lstlisting}

Alternatively, reverse the jumps to look more like the pseudo code:

\begin{lstlisting}[firstnumber=1,]
IF    CMP.L D0,D1    Flags = result of (D1.L - D0.L)
      BLS.s ELSE     D1 is not HI (unsigned) than D0
THEN  ...            D1 is indeed HI than D0, (the THEN bit)
      BRA.s ENDIF    Skip over the ELSE clause
ELSE  ...            Do the ELSE stuff
ENDIF ...            Together again.
\end{lstlisting}

Maybe we should think about writing our assembly language in pseudo code and
        having a pre-{}processor convert it into the real assembler code ....

\section{Loops With Conditions}
\label{ch19-loop-conditions}%hyperlabel{ch19-loop-conditions}%

The instruction format for decrement and branch on condition instructions is
        \opcode{DBcc} where `cc' is one of the many condition codes noted above.

So, you have an area of RAM full of data and you go looking through it for
        the first occurrence of a specific byte value, let's say \$00, and you know that
        the leading word of the data defines the length in bytes. So, the following
        fragment would do the job -{} assuming A0.L points to the data and D1.W holds a
        valid data length.

\begin{lstlisting}[firstnumber=1,]
LOOP    CMP.B #$00,(A0)+
        DBcc D1,LOOP
ENDLOOP ...
\end{lstlisting}

What we need to figure out is which `cc' we require and also, what is result
        when we get to the ENDLOOP label if we found a zero byte or if we didn't.

One way we will end up at ENDLOOP is when our counter in D1 expires -{}
        reaches minus 1 -{} that indicates that we ran out of data before finding what we
        wanted. But, what happens if we find a zero byte -{} and which `cc' do we
        need.

If we remember that \opcode{DBcc} really means `test condition and decrement if false
        and branch' then we should be ok. Alternatively:

\begin{lstlisting}[firstnumber=1,]
    IF 'cc' is FALSE THEN
       D1 = D1 - 1
       IF D1 <> -1 Then
         GOTO LOOP
       ELSE
         GOTO ENDLOOP
       END IF
    ELSE
       GOTO ENDLOOP
    END IF
\end{lstlisting}

So, we want to check for a zero byte, we can use the `EQ' test -{} remember EQ
        means we have hit a zero or two values are not equal -{} and our code now
        becomes:

\begin{lstlisting}[firstnumber=1,]
LOOP    CMP.B #$00,(A0)+
        DBEQ D1,LOOP
ENDLOOP ...
\end{lstlisting}

So, we have reached ENDLOOP and we need to know if we hit a zero byte or if
        we ran out of data. How to tell?

Well the good news is that the \opcode{DBcc} instructions do not alter the flags. So
        on exit from a \opcode{DBcc} loop, if the `cc' is still true, then the condition was met
        and the loop terminated before the counter ran out. All we have to do is retest
        with the same condition as follows:

\begin{lstlisting}[firstnumber=1,]
LOOP    CMP.B #$00,(A0)+
        DBEQ D1,LOOP
ENDLOOP BEQ.S FoundZeroByte
\end{lstlisting}

In this case, we check for the EQ condition which tells us that the loop
        terminated early. We can test the inverse condition as well to see if the loop
        expired without hitting the required condition:

\begin{lstlisting}[firstnumber=1,]
LOOP    CMP.B #$00,(A0)+
        DBEQ D1,LOOP
ENDLOOP BNE.S NotFound
\end{lstlisting}

Which we see makes the branch if the loop expired when the counter in D1.W
        hit minus 1. I propose that we rename this family of instructions to `Decrement
        and Branch UNLESS condition'. That makes more sense to me.

\section{Do I TST.L D0 After TRAPs And Vectors?}
\label{ch19-trap-vector}%hyperlabel{ch19-trap-vector}%

I always get corrected on this one, either by George or Simon. For years I
        have always done this:

\begin{lstlisting}[firstnumber=1,]
    ...
    TRAP #1
    TST.L D0
    BNE HandleError
    ...
\end{lstlisting}

Which is fine for a \opcode{TRAP} call -{} it \emph{has} to be done this
        way. However, for a vector call it is different:

\begin{lstlisting}[firstnumber=1,]
    ...
    MOVE.W UT_GTSTR,A2
    JSR (A2)
    TST.L D0
    BNE HandleEror
    ...
\end{lstlisting}

This is \emph{wrong} -{} I do not need to test D0 after a \emph{vectored} utility call. The
        reason I do after a \opcode{TRAP} and don't after a vector is quite subtle and was only
        recently pointed out to me by Simon when it all became very clear indeed.

A TRAP call is treated as an exception and to return from an exception
        handler, you use the \opcode{RTE} instruction. To return from a vectored call, it is an \opcode{RTS}
        instruction. The difference between the two is that the \opcode{RTE} restores the status
        register as well as the program counter. \opcode{RTS} simply restores the program
        counter.

So, all these years where I've been testing D0 on return from vectors I've
        been wasting clock cycles when I need not have done. The status register is
        correctly set on exit from a vectored utility but has only D0 is set on return
        from a TRAP.

Simple, but it has caught me out for years. I now need to unlearn my habit
	        of coding a \lstinline{TST.L D0} every time I use a vectored utility.

Happy coding.

\section{Coming Up...}
\label{ch19-the-end}%hyperlabel{ch19-the-end}%

After many years of studious avoidance, I'm biting the bullet and attempting
        to learn to code programs under the Pointer Environment.

