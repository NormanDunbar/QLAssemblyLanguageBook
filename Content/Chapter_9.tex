\chapter{Subroutines}

\section{Introduction}
\label{ch9-intro}%\hyperlabel{ch9-intro}%

Here we are in part 9 of the series on assembly language for the QL and what we
    will look at today are subroutines.

\section{Subroutines}
\label{ch9-subroutines}%\hyperlabel{ch9-subroutines}%

A subroutine is simply a piece of code that you call lots of times
    within your program. Because it is called so many times, you extract the
    working code, move it somewhere safe and add an \opcode{RTS} at the end. This is
    your subroutine -{} in its draft form!

Where the code used to be in the main source, now simply has a 
\lstinline{BSR sub_routine} in its place. The more times a routine is called, the bigger
    the saving in your typing and memory usage in the final program. Another
    major advantage of using subroutines is that you only need to change or
    correct them once -{} of course, if you make a mistake then every call to
    that subroutine is flawed as well!

For example, in a program you have written, you might find that you
    write the same piece of code numerous times to clear the screen, something
    like that shown in \lstlistingname~\ref{lst:ExampleOfRepetitiveCode}:

\begin{lstlisting}[firstnumber=1,caption={Example of Repetitive Code},label={lst:ExampleOfRepetitiveCode}]
    start   blah blah blah
       :
       :
        move.l  channel_id,a0       ; First channel id
        moveq    #sd_clear,d0       ; CLS
        moveq    #infinite,d3       ; Infinite timeout
        trap     #3                 ; CLS title window
       :
       :
        move.l  other_channel_id,a0 ; Another channel id
        moveq    #sd_clear,d0       ; CLS
        moveq    #infinite,d3       ; Infinite timeout
        trap     #3                 ; CLS title window
       :
       :
        move.l  another_id,a0       ; And another channel id
        moveq    #sd_clear,d0       ; CLS
        moveq    #infinite,d3       ; Infinite timeout
        trap     #3                 ; CLS title window
       :
       :
        rts                         ; All done
\end{lstlisting}

and so on. The above code looks duplicated and where you have
    duplication, you can usually -{} but not always -{} extract the duplicate code
    to a subroutine. We can now rewrite the code above as follows:

\begin{lstlisting}[firstnumber=1,caption={Example of Non-repetitive Code},label={lst:ExampleOfNonRepetetiveCode}]
start   blah blah blah
       :
       :
        move.l  channel_id,a0       ; First channel id
        bsr     cls
       :
       :
        move.l  other_channel_id,a0 ; Another channel id
        bsr     cls
       :
       :
        move.l  another_id,a0       ; And another channel id
        bsr     cls
       :
       :
        rts                         ; All done

*-------------------------------------------------------------------
* Subroutine to clear the SCR or CON channel whose ID is held in A0.
*-------------------------------------------------------------------
cls     moveq    #sd_clear,d0       ; CLS
        moveq    #-1,d3             ; Infinite timeout
        trap     #3                 ; CLS title window
        rts
\end{lstlisting}

The code that does the setting up of the various parameters for the
    system call to clear a channel has been extracted and placed at the end
    all by itself. An \opcode{RTS} instruction has been added to allow us to go back to
    where we came from. The second piece of code is easier (?) to read and
    will be smaller when finished.

So that is all there is to it. If you remember back to the boring
    part of this series (what do you mean `which boring part?') where I
    discussed the inner workings of the \opcode{BSR} instruction, you will remember
    that \opcode{BSR} stacks the address of the instruction that will be executed next
    (after the \opcode{BSR}), jumps to the address given and continues executing from
    there until it finds an \opcode{RTS} instruction.

The \opcode{RTS} instruction stops the program in its tracks, sets the PC (2
    points if you can remember what PC stands for ...) to the address that was
    stacked and proceeds to execute from there again. Those of you who are
    ahead of me at this point will realise that the \opcode{RTS} instruction takes the
    top 4 bytes off of the stack \emph{regardless} of what they are. If they are a
    valid return address then fine, no problems. If, on the other hand, they
    are some data, then who knows what will happen when the \opcode{RTS} is
    executed.

For this reason, it is very important that your stack should be
    exactly the same on the way out of a subroutine as it was on the way in.
    Don't do something like that shown in \lstlistingname~\ref{lst:ExampleOfAMessedUpStack}, for example:

\begin{lstlisting}[firstnumber=1,caption={Example of a Messed up Stack!},label={lst:ExampleOfAMessedUpStack}]
start   blah blah blah
       :
       :
        move.l  channel_id,a0       ; First channel id
        bsr     cls
       :
       :
        rts

*---------------------------------------------------------------------
* BROKEN subroutine to clear the SCR or CON channel whose ID is in A0.
*---------------------------------------------------------------------
cls     move.l   d0,-(a7)           ; Preserve D0 until later
        moveq    #sd_clear,d0       ; CLS
        moveq    #-1,d3             ; Infinite timeout
        trap     #3                 ; CLS title window
        rts                         ; Program explodes here!
\end{lstlisting}

In this example, the old value of D0.L is on the stack on top of the
    return address. When the \opcode{RTS} instruction is executed it doesn't know (or
    care) about what is on the stack, it just grabs the top 4 bytes and sets
    the PC to that `address'. (You get 2 points if you remembered PC = Program
    Counter!)

\section{Building A Library}
\label{ch9-build-library}%\hyperlabel{ch9-build-library}%

As you progress with assembly language programming, you may find
    that you build up a lot of subroutines in your programs. What to do with
    them all?

Why not build yourself a library of routines that you can include in
    every program that needs them. This way, you have a full set of tried and
    tested bits of code -{} which you should document somewhere -{} that can be
    reused over and over again. The rest of this article will help you on your
    way by building a number of useful (well, I have found them to be useful
    over the years) subroutines that you can use.

\section{Documentation}
\label{ch9-documentation}%\hyperlabel{ch9-documentation}%

As with all good things, documentation is a must. If you have a
    large number of useful routines then they should be documented somewhere.
    This will allow you to look for a routine in your library and from that,
    find out its input \& output parameters and which file it lives
    in.

A suitable template that you could use for each subroutine is as
    follows:

\begin{lstlisting}[firstnumber=1,frame=none,numbers=none]
*---------------------------------------------------------------------
* NAME
* DEPENDENCY (1)
* DEPENDENCY (2)
* PURPOSE
* INPUTS
* OUTPUTS
*---------------------------------------------------------------------
\end{lstlisting}

The above looks very like comments in a source file -{} this implies
    that we could add the documentation to the source file and then run a
    utility program to extract the details and store them in a text file -{}
    which you can edit and/or print as desired.

Having a standard header above each subroutine also implies that you
    could write a utility program to scan the entire library and ask you which
    ones you want to include in your output file -{} which will be your source
    file for your next masterpiece -{} before extracting them all and writing
    them to this file.

As for the subroutines themselves, I mentioned above that they exist
    in a draft form when you simply extract the code from the `wordy' source
    and add an \opcode{RTS} to the end. This is fine, but it could be that you need to
    preserve certain registers so that the code calling the subroutine doesn't
    need to keep saving and restoring them. The updates required are:

Check which registers will be used by the code explicitly -{} save
    them before and restore them after the main part of the subroutine
    code.

Check which system calls are made by the subroutine and look up the
    QDOSMSQ documentation to see which registers are trashed by the system
    call. Add these registers to the save and restore routines.

Save the registers as the first line of code in the subroutine and
    restore them as the line immediately before the \opcode{RTS} (or as near to the \opcode{RTS} as possible).

Always have the subroutine return an error code and/or the flags set
    to signal if an error occurred or not.

An actual example is shown in \lstlistingname~\ref{lst:ASubroutineExample}:

\begin{lstlisting}[firstnumber=1,caption={A Subroutine Example},label={lst:ASubroutineExample}]
*---------------------------------------------------------------------
* NAME          CLS
*---------------------------------------------------------------------
* DEPENDENCY    None
* PURPOSE       To clear a screen/console channel.
* DESCRIPTION   Clears the screen channel whose ID is supplied in A0.
* INPUTS:
*               A0.L = channel ID
* OUTPUTS:
*               D0 = Error code
*               Z flag set if no errors, unset otherwise.
*---------------------------------------------------------------------
cls             move.l   d1/d3/a1,-(a7) ; Corrupted by SD_CLEAR
                moveq    #sd_clear,d0   ; SD_CLEAR defined in GWASL
                moveq    #-1,d3         ; Infinite timeout
                trap     #3             ; CLS the window
                move.l   (a7)+,d1/d3/a1 ; Restore corrupted registers
                tst.l    d0             ; Set Z flag if all ok
                rts
\end{lstlisting}

In the above example, I have extended the `cls' code from our
    original subroutine as follows:
\begin{itemize}[itemsep=0pt]

\item{}-{} I have added a documentation header comment.


\item{}-{} I have preserved D3 because I use it in the code
        myself.


\item{}-{} I have preserved D1 and A1 because the QDOSMSQ documentation
        states that these two registers are `undefined' on return from the
        system trap \trap{SD\_CLEAR}.


\item{}-{} I have restored all 3 of these registers before the
        \opcode{RTS}.


\item{}-{} I have added a \lstinline{TST.L D0} instruction to set the Z flags
        according to whether an error was detected or not.

\end{itemize}

Note that although D0 is used by the code and by the system call, I
    have not preserved it. This is quite simply because I use D0 to return any
    error codes back to the caller. As I have documented its corruption in the
    header, I assume that the user of the subroutine will read this and know
    all about it!

Bullet proofing the code like this helps to reduce unexpected bugs
    in your programs when you forget to save a register and after a subroutine
    call, assume it still has the same value as before. I know, I have been
    there. Of course, there is not much you can do to prevent the
    documentation you use from being wrong (been there too) but at least you
    did your best!

\section{The Subroutine Library}
\label{ch9-the-library}%\hyperlabel{ch9-the-library}%

Onwards with the code for my (useful) subroutines.

\section{STR\_COPY}
\program{STR\_COPY}
\label{ch9-STR_COPY}%\hyperlabel{ch9-STR_COPY}%

\begin{lstlisting}[firstnumber=1,caption={STR\_COPY}]
*---------------------------------------------------------------------
* NAME          STR_COPY
*---------------------------------------------------------------------
* DEPENDENCY    None
* PURPOSE       Copy the string at (A2) over the string at (A1).
* DESCRIPTION   Copy the string whose address is passed in A2 over the
*               string whose address is passed in A1 thus overwriting
*               the old contents of the receiving string.
* INPUTS:
*               A1.L = Address of the receiving string
*               A2.L = Address of the sending string
* OUTPUTS:
*               A1.L = Address of the receiving string (preserved)
*               A2.L = Address of the sending string (preserved)
*---------------------------------------------------------------------
str_copy    movem.l d0/a1-a2,-(a7)  ; Preserve working register
            move.w  (a2)+,d0        ; Get size of 'from' string
            move.w  d0,(a1)+        ; Set new size of 'to' string
            bra.s   sc_next         ; Skip the dbra stuff first time
sc_moveb    move.b  (a2)+,(a1)+     ; Move a single byte
sc_next     dbra    d0,sc_moveb     ; And the rest
            movem.l (a7)+,d0/a1-a2  ; Restore working registers
            rts                     ; Exit
\end{lstlisting}

\section{STR\_APPEND}
\program{STR\_APPEND}
\label{ch9-STR_APPEND}%\hyperlabel{ch9-STR_APPEND}%

\begin{lstlisting}[firstnumber=1,caption={STR\_APPEND}]
*---------------------------------------------------------------------
* NAME          STR_APPEND
*---------------------------------------------------------------------
* DEPENDENCY    STR_COPY
* PURPOSE       Append string at (A2) to the end of string at (A1).
* DESCRIPTION   Copy the string whose address is passed in A2 to the
*               end of the string whose address is passed in A1. The 
*               old contents of both strings will be preserved - 
*               except A1 which will be extended of course!
* INPUTS:
*               A1.L = Address of the receiving string
*               A2.L = Address of the sending string
* OUTPUTS:
*               A1.L = Address of the receiving string (preserved)
*               A2.L = Address of the sending string (Preserved)
*---------------------------------------------------------------------
str_append  movem.l  d0/a1-a2,-(a7) ; Save the working register
            move.w  (a2)+,d0        ; Size of 'from' string
            move.w  (a1),d1         ; Size of 'to' string
            add.w   d0,(a1)+        ; New size of 'to' string
            adda.w  d1,a1           ; New 'to' string end position
            bra.s   sc_next         ; Copy bytes over using STR_COPY
                                    ; D0 is restored by STR_COPY
                                    ; STR_APPEND exits via STR_COPY.
\end{lstlisting}

\section{STR\_REVERSE}
\program{STR\_REVERSE}
\label{ch9-STR_REVERSE}%\hyperlabel{ch9-STR_REVERSE}%

\begin{lstlisting}[firstnumber=1,caption={STR\_REVERSE}]
*---------------------------------------------------------------------
* NAME          STR_REVERSE
*---------------------------------------------------------------------
* DEPENDENCY    None
* PURPOSE       Reverse the bytes in the string at (A1).
* DESCRIPTION   Reverses the bytes in the string with address (A1).
* INPUTS:
*               A1.L = Address of the string to be reversed
* OUTPUTS:
*               A1.L = Address of the string to be reversed (Preserved)
*---------------------------------------------------------------------
str_reverse move.l  d0-d1/a1-a2,-(a7)   ; Save working registers
            move.l  a1,a2               ; Copy start address
            move.w  (a1)+,d0            ; Fetch length word
            beq.s   sr_quit             ; Nothing to do
            adda.w  d0,a2               ; Near the end of the string
            addq.l  #1,a2               ; The last char in the string
            lsl.w   #1,d0               ; D0 = INT(D0/2)
            bra.s   sr_next             ; Skip the first one for DBRA
sr_loop     move.b  (a2),d1             ; Fetch the last character
            move.b  (a1),(a2)           ; Move the first byte to last
            move.b  d1,(a1)+            ; Move the last byte to first
            subq.l  #1,a2               ; And adjust last
sr_next     dbra    d0,sr_loop          ; And do the rest
sr_quit     movem.l (a7)+,d0-d1/a1-a2   ; Restore the working registers
            rts
\end{lstlisting}

\section{STR\_INSERT}
\program{STR\_INSERT}
\label{ch9-STR_INSERT}%\hyperlabel{ch9-STR_INSERT}%

\begin{lstlisting}[firstnumber=1,caption={STR\_INSERT}]
*---------------------------------------------------------------------
* NAME          STR_INSERT
*---------------------------------------------------------------------
* DEPENDENCY    STR_APPEND
* PURPOSE       Insert string at (A2) into string at (A1) at pos D0.W.
* DESCRIPTION   Inserts the string with address at (A2) into the string
*               with address (A1) at the position passed in D0.W so the
*               first character in the inserted string is at (A1,D0.W)
*               after the insertion. (0 is the very first character!)
*               If D0 >= length (A1) then call STR_APPEND.
* INPUTS:
*               A1.L = Address of the receiving string
*               A2.L = Address of the string to be inserted
*               D0.W = Position (starting at 0) where to insert before
* OUTPUTS:
*               D0 = Error code
*               A1.L = Address of the receiving string (preserved)
*               A2.L = Address of the string to be inserted (preserved)
*               Z flag set if no errors, unset otherwise.
*---------------------------------------------------------------------
str_insert  cmp.w   d0,(a1)             ; Are we appending perhaps?
            bge     str_append          ; Yes, easy case to deal with!
            tst.w   d0                  ; Is there anything in D0?
            bge.s   si_ok               ; Yes, negatives are bad!
            moveq   #-15,d0             ; Bad parameter
            rts                         ; Z is unset, D0 = error code

si_ok       movem.l d1/a1-a4,-(a7)      ; Save those workers
            move.l  a1,a3               ; A3 = Address of A1 string
            adda.w  (a1),a3             ; Plus the size ...
            addq.l  #2,a3               ; A3 = last char of string +1
            move.l  a3,a4               ; A4 = new last char afterwards
            adda.w  (a2),a4             ; Add the extra length
            addq.l  #2,a4               ; And now we are there (+1)
            move.w  (a2),d1             ; Size of inserted string
            bra.s   si_dnext            ; Skip dbra
si_dmove    move.b  -(a3),-(a4)         ; Move a byte
si_dnext    dbra    d1,si_dmove         ; Do the rest
            move.w  (a2),d1             ; Refetch the inserted length
            adda.w  d1,a2               ; A2 nearly at the last char
            addq.w  #2,a2               ; One past the last character
            bra.s   si_inext            ; Skip dbra stuff
si_imove    move.b  -(a3),-(a4)         ; Insert a byte
si_inext    dbra    d1,si_imove         ; Insert the rest
            movem.l (a7)+,d1/a1-a4      ; Restore those workers
            clr.l   d0                  ; No errors
            rts
\end{lstlisting}

\section{STR\_COMP}
\program{STR\_COMP}
\label{ch9-STR_COMP}%\hyperlabel{ch9-STR_COMP}%

\begin{lstlisting}[firstnumber=1,caption={STR\_COMP}]
*---------------------------------------------------------------------
* NAME          STR_COMP
*---------------------------------------------------------------------
* DEPENDENCY    None
* PURPOSE       To compare two strings for exact equality
* DESCRIPTION   Compare the strings at (A1) and (A2) for equality.
*               Numbers in the string are considered as well.
*               Equivalent to 'IF (A1$ = A2$)'
* INPUTS:
*               A1.L = First string
*               A2.L = Second string
* OUTPUTS:
*               D0 = Result of comparison.
*                    -1 = A1 string is < A2 string
*                     0 = A1 string = A2 string
*                    +1 = A1 string > A2 string
*               A1.L = First string (preserved)
*               A2.L = Second string (preserved)
*---------------------------------------------------------------------
str_comp    movem.l a0-a2,-(a7)         ; Must preserve workers
            moveq   #2,d0               ; Case &  numbers considered
sc_params   move.l  a1,a0               ; Uses different registers
            move.l  a2,a1               ; So swap them over
            move.w  UT_CSTR,a2          ; Fetch the vector address
            jsr     (a2)                ; Compare strings using ROM
            movem.l (a7)+,a0-a2         ; Restore working registers
            tst.l  d0                   ; Make sure Z is set/unset
            rts
\end{lstlisting}

\section{STR\_COMPI}
\program{STR\_COMPI}
\label{ch9-STR_COMPI}%\hyperlabel{ch9-STR_COMPI}%

\begin{lstlisting}[firstnumber=1,caption={STR\_COMPI}]
*---------------------------------------------------------------------
* NAME          STR_COMPI
*---------------------------------------------------------------------
* DEPENDENCY    STR_COMP
* PURPOSE       To compare two strings for approximate equality
* DESCRIPTION   Compare the strings at (A1) and (A2) for equality with
*               numbers considered but not letter case.
*               Equivalent to 'IF (A1$ == A2$)'
* INPUTS:
*               A1.L = First string
*               A2.L = Second string
* OUTPUTS:
*               D0 = Result of comparison.
*                    -1 = A1 string is < A2 string
*                     0 = A1 string == A2 string
*                    +1 = A1 string > A2 string
*               A1.L = First string (preserved)
*               A2.L = Second string (preserved)
*---------------------------------------------------------------------
str_compare movem.l a0-a2,-(a7)         ; Must preserve workers
            moveq   #3,d0               ; Numbers + Case insignificant
            bra     sc_params           ; Jump into STR_COMP
\end{lstlisting}

\section{FILE\_CLOSE}
\program{FILE\_CLOSE}
\label{ch9-FILE_CLOSE}%\hyperlabel{ch9-FILE_CLOSE}%

\begin{lstlisting}[firstnumber=1,caption={FILE\_CLOSE}]
*---------------------------------------------------------------------
* NAME          FILE_CLOSE
*---------------------------------------------------------------------
* DEPENDENCY    None
* PURPOSE       Close the channel passed in A0
* DESCRIPTION   Close the file channel with QDOS ID in A0. To prevent
*               any original QL systems from serious problems, checks
*               for #0 being closed and ignores it.
* INPUTS:
*               A0.L = Channel ID to be closed
* OUTPUTS:
*               D0 is preserved as IO_CLOSE does not return errors
*               except NOT OPEN and we ignore these here! The Z flag 
*               is indeterminate after this subroutine.
*               A0.L is returned undefined to avoid channel reuse.
*---------------------------------------------------------------------
file_close  cmpa.l  #0,a0               ; Test for SuperBasic #0
            beq.s   fc_exit             ; Ignore it
            move.l  d0,-(a7)            ; Preserve the worker
            moveq   #io_close,d0        ; Prepare to close it
            trap    #2                  ; Close it
            move.l  (a7)+,d0            ; Restore the worker
fc_exit     rts
\end{lstlisting}

\section{FILE\_OPEN}
\program{FILE\_OPEN}
\label{ch9-FILE_OPEN}%\hyperlabel{ch9-FILE_OPEN}%

\begin{lstlisting}[firstnumber=1,caption={FILE\_OPEN}]
*---------------------------------------------------------------------
* NAME          FILE_OPEN
*---------------------------------------------------------------------
* DEPENDENCY    None
* PURPOSE       To open a file like 'OPEN #3,filename'
* DESCRIPTION   Opens a file in mode 0 (old exclusive device) The 
*               filename is passed in A0. The current job assumes
*               ownership of the channel. May need a TRAP #4 before 
*               calling if the filename is relative A6 when called.
* INPUTS:
*               A0.L = Pointer to filename
* OUTPUTS:
*               A0.L = Channel id.
*               D0 = Error code
*               Z flag set if no errors, unset otherwise.
*---------------------------------------------------------------------
file_open       movem.l d1-d3,-(a7)     ; Those workers need saving
                moveq   #0,d3           ; Old exclusing device mode
fo_params       moveq   #IO_OPEN,d0     ; Trap code
                moveq   -1,d1           ; Current job owns the channel
                trap #2                 ; Open it
                movem.l (a7)+,d1-d3     ; Restore workers
                tst.l   d0              ; Make sure Z is set/unset
                rts
\end{lstlisting}

\section{FILE\_OPENIN}
\program{FILE\_OPENIN}
\label{ch9-FILE_OPENIN}%\hyperlabel{ch9-FILE_OPENIN}%

\begin{lstlisting}[firstnumber=1,caption={FILE\_OPENIN}]
*---------------------------------------------------------------------
* NAME          FILE_OPENIN
*---------------------------------------------------------------------
* DEPENDENCY    FILE_OPEN
* PURPOSE       To open a file like 'OPEN_IN #3,filename'
* DESCRIPTION   Opens a file in mode 1 (old shared device) The 
*               filename is passed in A0. The current job assumes
*               ownership of the channel. May need a TRAP #4 before 
*               calling if the filename is relative A6 when called.
* INPUTS:
*               A0.L = Pointer to filename
* OUTPUTS:
*               A0.L = Channel id.
*               D0 = Error code
*               Z flag set if no errors, unset otherwise.
*---------------------------------------------------------------------
file_openin     movem.l d1-d3,-(a7)     ; Those workers need saving
                moveq   #1,d3           ; Old shared device mode
                bra     fo_params       ; Do the rest via FILE_OPEN
\end{lstlisting}

\section{FILE\_OPENNEW}
\program{FILE\_OPENNEW}
\label{ch9-FILE_OPENNEW}%\hyperlabel{ch9-FILE_OPENNEW}%

\begin{lstlisting}[firstnumber=1,caption={FILE\_OPENNEW}]
*---------------------------------------------------------------------
* NAME          FILE_OPENNEW
*---------------------------------------------------------------------
* DEPENDENCY    FILE_OPEN
* PURPOSE       To open a file like 'OPEN_NEW #3,filename'
* DESCRIPTION   Opens a file in mode 2 (new exclusive device) The 
*               filename is passed in A0. The current job assumes
*               ownership of the channel. May need a TRAP #4 before 
*               calling if the filename is relative A6 when called.
* INPUTS:
*               A0.L = Pointer to filename
* OUTPUTS:
*               A0.L = Channel id.
*               D0 = Error code
*               Z flag set if no errors, unset otherwise.
*---------------------------------------------------------------------
file_opennew    movem.l d1-d3,-(a7)     ; Those workers need saving
                moveq   #2,d3           ; New exclusive device mode
                bra     fo_params       ; Do the rest via FILE_OPEN
\end{lstlisting}

\section{FILE\_OPENOVER}
\program{FILE\_OPENOVER}
\label{ch9-FILE_OPENOVER}%\hyperlabel{ch9-FILE_OPENOVER}%

\begin{lstlisting}[firstnumber=1,caption={FILE\_OPENOVER}]
*---------------------------------------------------------------------
* NAME          FILE_OPENOVER
*---------------------------------------------------------------------
* DEPENDENCY    FILE_OPEN
* PURPOSE       To open a file like 'OPEN_OVER #3,filename'
* DESCRIPTION   Opens a file in mode 3 (new overwrite device) The
*               filename is passed in A0. The current job assumes
*               ownership of the channel. May need a TRAP #4 before 
*               calling if the filename is relative A6 when called.
* INPUTS:
*               A0.L = Pointer to filename
* OUTPUTS:
*               A0.L = Channel id.
*               D0 = Error code
*               Z flag set if no errors, unset otherwise.
*---------------------------------------------------------------------
file_openover   movem.l d1-d3,-(a7)     ; Those workers need saving
                moveq   #3,d3           ; New overwrite device mode
                bra     fo_params       ; Do the rest via FILE_OPEN
\end{lstlisting}

\section{FILE\_OPENDIR}
\program{FILE\_OPENDIR}
\label{ch9-FILE_OPENDIR}%\hyperlabel{ch9-FILE_OPENDIR}%

\begin{lstlisting}[firstnumber=1,caption={FILE\_OPENDIR}]
*---------------------------------------------------------------------
* NAME          FILE_OPENDIR
*---------------------------------------------------------------------
* DEPENDENCY    FILE_OPEN
* PURPOSE       To open a file like 'OPEN_DIR #3,devicename'
* DESCRIPTION   Opens a file in mode 4 (directory) The filename is
*               passed in A0. The current job assumes ownership of 
*               the channel. May need a TRAP #4 before calling if the 
*               filename is relative A6 when called.
* INPUTS:
*               A0.L = Pointer to filename
* OUTPUTS:
*               A0.L = Channel id.
*               D0 = Error code
*               Z flag set if no errors, unset otherwise.
*---------------------------------------------------------------------
file_opendir    movem.l d1-d3,-(a7)     ; Those workers need saving
                moveq   #4,d3           ; Directory mode
                bra     fo_params       ; Do the rest via FILE_OPEN
\end{lstlisting}

\section{FILE\_GET\_HEAD}
\program{FILE\_GET\_HEAD}
\label{ch9-FILE_GET_HEAD}%\hyperlabel{ch9-FILE_GET_HEAD}%

\begin{lstlisting}[firstnumber=1,caption={FILE\_GET\_HEAD}]
*---------------------------------------------------------------------
* NAME          FILE_GET_HEAD
*---------------------------------------------------------------------
* DEPENDENCY    None
* PURPOSE       To read the 64 bytes header for a file. (already open)
* DESCRIPTION   Reads a 64 byte file header for the open file with ID
*               in A0.L into the buffer  whose address is passed
*               in A1.L. This buffer must be at least 64 bytes long!
* INPUTS:
*               A0.L = Channel ID
*               A1.L = Address of 64 byte buffer to put header into
* OUTPUTS:
*               D0 = Error code
*               D1 = Size of header read into buffer
*               A0 = Channel id (preserved)
*               A1 = Address of buffer (preserved)
*               Z flag set if no errors, unset otherwise.
*---------------------------------------------------------------------
file_get_head   movem.l d2-d3/a0-a1,-(a7) ; Save working registers
                moveq   #FS_HEADR,d0    ; Get trap code
                moveq   #64,d2          ; Buffer size
fgh_rest        moveq   #-1,d3          ; Infinity is a big thing
                trap    #3              ; Do it
                movem.l (a7)+,d2-d3/a0-a1 ; Restore workers
                tst.l   d0              ; Set flags
                rts                     ; Return to caller
\end{lstlisting}

\section{FILE\_SET\_HEAD}
\program{FILE\_SET\_HEAD}
\label{ch9-FILE_SET_HEAD}%\hyperlabel{ch9-FILE_SET_HEAD}%

\begin{lstlisting}[firstnumber=1,caption={FILE\_SET\_HEAD}]
*---------------------------------------------------------------------
* NAME          FILE_SET_HEAD
*---------------------------------------------------------------------
* DEPENDENCY    FILE_GET_HEAD
* PURPOSE       To write a 64 bytes header for a file. (already open)
* DESCRIPTION   Writes a 64 byte file header for the open file with ID
*               in A0.L from the buffer whose address is passed in
*               A1.L. This buffer must be at least 64 bytes long!
* INPUTS:
*               A0.L = Channel ID
*               A1.L = Address of 64 byte buffer holding the header
* OUTPUTS:
*               D0 = Error code
*               D1 = Size of header written from buffer
*               A0 = Channel id (preserved)
*               A1 = Address of buffer (preserved)
*               Z flag set if no errors, unset otherwise.
*---------------------------------------------------------------------
file_set_head   movem.l d2-d3/a0-a1,-(a7) ; Save working registers
                moveq   #FS_HEADS,d0    ; Get trap code
                bra     fgh_rest        ; Do rest via FILE_GET_HEAD
\end{lstlisting}

\section{PRINT}
\program{PRINT}
\label{ch9-PRINT}%\hyperlabel{ch9-PRINT}%

\begin{lstlisting}[firstnumber=1,caption={PRINT}]
*---------------------------------------------------------------------
* NAME          PRINT
*---------------------------------------------------------------------
* DEPENDENCY    None
* PURPOSE       To send the string at (A1) to the channel in A0.
* DESCRIPTION   This routine prints a QDOS string (word then bytes) to
*               the channel ID passed in A0. The string starts at A1.
* INPUTS:
*               A0.L = Channel ID
*               A1.L = Address of a QDOS format string to be printed.
* OUTPUTS:
*               D0 = Error code
*               A0 = Channel id (preserved)
*               A1 = Address of buffer (preserved)
*               Z flag set if no errors, unset otherwise.
*---------------------------------------------------------------------
print           move.l a1, -(a7)        ; Preserve the buffer address
                movea.w ut_mtext,a2     ; Print a string utility
                jsr     (a2)            ; Print it
                move.l  (a7)+,a1        ; Restore the buffer address
                tst.l   d0              ; Check for errors
                rts
\end{lstlisting}

\section{LINE\_FEED}
\program{LINE\_FEED}
\label{ch9-LINE_FEED}%\hyperlabel{ch9-LINE_FEED}%

\begin{lstlisting}[firstnumber=1,caption={LINE\_FEED}]
*---------------------------------------------------------------------
* NAME          LINE_FEED
*---------------------------------------------------------------------
* DEPENDENCY    None
* PURPOSE       To send a linefeed character to the channel in A0.
* DESCRIPTION   This routine prints a linefeed character to the channel
*               ID passed in A0.
* INPUTS:
*               A0.L = Channel ID
* OUTPUTS:
*               D0 = Error code
*               A0 = Channel id (preserved)
*               Z flag set if no errors, unset otherwise.
*---------------------------------------------------------------------
line_feed       movem.l d1/a1,-(a7)     ; Preserve registers 
                moveq   #io_sbyte,d0    ; Send one byte to channel
                moveq   #linefeed,d1    ; Byte to send = linefeed
                moveq   #infinite,d3    ; Timeout
                trap    #3              ; Do it
                movem.l (a7)+,d1/a1     ; Restore
                tst.l   d0              ; Set Z if errors
                rts
\end{lstlisting}

\section{INPUT}
\program{INPUT}
\label{ch9-INPUT}%\hyperlabel{ch9-INPUT}%

\begin{lstlisting}[firstnumber=1,caption={INPUT}]
*---------------------------------------------------------------------
* NAME          INPUT
*---------------------------------------------------------------------
* DEPENDENCY    None
* PURPOSE       To obtain input from the user via the channel ID in A0.
* DESCRIPTION   This routine allows the user to type into a buffer
*               (which is part of this subroutine) up to a maximum of
*               256 bytes. A channel ID in A0 is used.
* INPUTS:
*               A0.L = Channel ID
* OUTPUTS:
*               D0 = Error code
*               D1.W = Number of characters typed EXCLUDING ENTER 
*                      if D1.W = 0, user simply pressed ENTER.
*               A0 = Channel id (preserved)
*               A1 = Start of buffer (word count of string user typed)
*               Z flag set if no errors, unset otherwise.
*---------------------------------------------------------------------
input       movem.l d2-d3,-(a7)         ; Preserve working registers
            lea     i_buffer+2,a1       ; Our buffer address plus 2
            move.l  a1,-(a7)            ; Save it on the stack
            moveq   #io_fline,d0        ; Input some bytes (inc LF)
            moveq   #256,d2             ; Buffer size maximum
            moveq   #infinite,d3        ; Inifinite timeout
            trap    #3

            move.l  (a7)+,a1            ; Restore buffer pointer
            subq.w  #1,d1               ; Subtract LF character
            move.w  d1,-(a1)            ; Save length and set A1
            movem.l (a7)+,d2-d3         ; Restore those workers
            tst.l   d0                  ; Did it all work?
            rts

i_buffer    ds.w    128+1               ; 256 chars + 1 word
\end{lstlisting}

\section{JOB\_HEADER}
\program{JOB\_HEADER}
\label{ch9-JOB_HEADER}%\hyperlabel{ch9-JOB_HEADER}%

\begin{lstlisting}[firstnumber=1,caption={JOB\_HEADER}]
*---------------------------------------------------------------------
* NAME          JOB_HEADER
*---------------------------------------------------------------------
* DEPENDENCY    None
* PURPOSE       Code required to define a QDOSMSQ job header.
* DESCRIPTION   Defines a job header ready to be filled in by the user.
*               The user will fill in his/her own jobname between the 
*               quotes and the assembler will do the rest.
* INPUTS:      None.
* OUTPUTS:     None.
*---------------------------------------------------------------------
start       bra.s   prog_start          ; Short jump to program start
            dc.l    0                   ; Spare. 
            dc.w    $4afb               ; Job id must be at location 6
prog_name   dc.w    prog_start-prog_name-2  ; Length of job name
            dc.b    ''                  ; YOUR JOBNAME HERE

prog_start  PUT YOUR CODE HERE
\end{lstlisting}

\section{MEM\_ALLOC}
\program{MEM\_ALLOC}
\label{ch9-MEM_ALLOC}%\hyperlabel{ch9-MEM_ALLOC}%

\begin{lstlisting}[firstnumber=1,caption={MEM\_ALLOC}]
*---------------------------------------------------------------------
* NAME          MEM_ALLOC
*---------------------------------------------------------------------
* DEPENDENCY    None
* PURPOSE       Allocate an area of memory on the heap.
* DESCRIPTION   Allocate an area of memory, size as per D0.L, and
*               return the address of the allocated area in A0.L. D0 
*               is set to any error code and the Z flag will be set if
*               no errors occurred, reset otherwise.
* INPUTS:
*               D0.L = Size, in bytes, of memory area to be allocated
* OUTPUTS:
*               A0.L = Base address of the memory area allocated
*               D0 = Error code
*               Z flag set if no errors, unset otherwise.
*---------------------------------------------------------------------
mem_alloc       movem.l d1-d3/a1-a3,-(a7)   ; Save working registers
                move.l  d0,d1               ; Space required in D1
                moveq   #MT_ALCHP,d0        ; Set the trap
                moveq   #-1,d2              ; For the current job
                trap    #1                  ; Do it
                movem.l (a7)+,d1-d3/a1-a3   ; Restore working registers
                tst.l   d0                  ; Set flags
                rts
\end{lstlisting}

\section{MEM\_DEALLOC}
\program{MEM\_DEALLOC}
\label{ch9-MEM_DEALLOC}%\hyperlabel{ch9-MEM_DEALLOC}%

\begin{lstlisting}[firstnumber=1,caption={MEM\_DEALLOC}]
*---------------------------------------------------------------------
* NAME          MEM_DEALLOC
*---------------------------------------------------------------------
* DEPENDENCY    None
* PURPOSE       Deallocate an already allocated area of memory
* DESCRIPTION   Deallocate a previously allocated area of memory, the 
*               address of which is passed in A0.L.
* INPUTS:
*               A0.L = Address of area to deallocate
* OUTPUTS:
*               A0.L = zero to avoid using the memory again!
*               D0 = Error code
*               Z flag set if no errors, unset otherwise.
*---------------------------------------------------------------------
mem_dealloc     movem.l d1-d3/a1-a3,-(a7)   ; Save working registers
                moveq   #MT_RECHP,d0        ; Set the trap
                trap    #1                  ; Do it
                movem.l (a7)+,d1-d3/a1-a3   ; Restore registers
                suba.l  a0,a0               ; Blank the memory address
                tst.l   d0                  ; Set flags
                rts
\end{lstlisting}

\section{SCR\_MODE}
\program{SCR\_MODE}
\label{ch9-SCR_MODE}%\hyperlabel{ch9-SCR_MODE}%

\begin{lstlisting}[firstnumber=1,caption={SCR\_MODE}]
*---------------------------------------------------------------------
* NAME          SCR_MODE
*---------------------------------------------------------------------
* DEPENDENCY    None
* PURPOSE       Check the mode & set if required
* DESCRIPTION   Checks for the mode passed in D0 and if not correct, 
*               change to that mode.
* INPUTS:
*               D0.B = 4 or 8 for required mode
* OUTPUTS:
*               D0 = Error code
*               Z flag set if no errors, unset otherwise.
*---------------------------------------------------------------------
scr_mode        move.l   d1-d2/d7/a3,-(a7)  ; Save working registers
                move.b   d0,d7              ; Save required mode
                cmpi.w   #4,d7              ; Is mode 4 required?
                bne.s    scrm_8             ; Nope.
                clr.b    d7                 ; Mode 4 requires 0
scrm_8          moveq    #mt_dmode,d0
                moveq    #-1,d1             ; Read current mode
                moveq    #-1,d2             ; Read current display type
                trap     #1                 ; Do it
                tst.l    d0                 ; Did it work?
                bne.s    scrm_exit          ; No, bale out
                cmp.b    d1,d7              ; Correct mode?
                beq.s    scrm_exit          ; Don't set mode if ok
                moveq    #mt_dmode,d0       ; Else, set it
                move.b   d7,d1              ; Get the mode from D7
                trap     #1                 ; Set mode
                move.l   (a7)+,d1-d2/d7/a3  ; Restore registers
                tst.l    d0                 ; Set Z flag if ok
scrm_exit       rts
\end{lstlisting}

\section{CLS}
\program{CLS}
\label{ch9-CLS}%\hyperlabel{ch9-CLS}%

\begin{lstlisting}[firstnumber=1,caption={CLS}]
*---------------------------------------------------------------------
* NAME          CLS
*---------------------------------------------------------------------
* DEPENDENCY    None
* PURPOSE       To clear a screen/console channel.
* DESCRIPTION   Clears the screen channel whose ID is supplied in A0.
* INPUTS:
*               A0.L = channel ID
* OUTPUTS:
*               D0 = Error code
*               A0.L = channel ID (preserved)
*               Z flag set if no errors, unset otherwise.
*---------------------------------------------------------------------
cls             move.l   d1/d3/a1,-(a7) ; These are corrupted
                moveq    #sd_clear,d0   ; CLS
                moveq    #-1,d3         ; Infinite timeout
                trap     #3             ; CLS the window
                move.l   (a7)+,d1/d3/a1 ; Restore corrupted registers
                tst.l    d0             ; Set Z flag if ok
                rts
\end{lstlisting}

\section{SCR\_PAPER}
\program{SCR\_PAPER}
\label{ch9-SCR_PAPER}%\hyperlabel{ch9-SCR_PAPER}%

\begin{lstlisting}[firstnumber=1,caption={SCR\_PAPER}]
*---------------------------------------------------------------------
* NAME          SCR_PAPER
*---------------------------------------------------------------------
* DEPENDENCY    None
* PURPOSE       Set the PAPER colour for the given channel ID.
* DESCRIPTION   Sets the paper colour for the screen channel whose ID 
*               is passed in A0, to the colour code supplied in D0.W.
* INPUTS:
*               D0.W = colour code for paper colour
*               A0.L = Channel ID.
* OUTPUTS:
*               D0 = Error code
*               A0.L = channel ID (preserved)
*               Z flag set if no errors, unset otherwise.
*---------------------------------------------------------------------
scr_paper       move.l   d1/d3/a1,-(a7) ; These will be corrupted
                move.w   d0,d1          ; Get the paper colour
                moveq    #sd_clear,d0   ; CLS
                moveq    #-1,d3         ; Infinite timeout
                trap     #3             ; Set PAPER colour (not STRIP)
                move.l   (a7)+,d1/d3/a1 ; Restore corrupted registers
                tst.l    d0             ; Set Z flag if all ok
                rts
\end{lstlisting}

\section{SCR\_PAPER\_SB}
\program{SCR\_PAPER\_SB}
\label{ch9-SCR_PAPER_SB}%\hyperlabel{ch9-SCR_PAPER_SB}%

\begin{lstlisting}[firstnumber=1,caption={SCR\_PAPER\_SB}]
*---------------------------------------------------------------------
* NAME          SCR_PAPER_SB
*---------------------------------------------------------------------
* DEPENDENCY    SCR_PAPER
* DEPENDENCY    SCR_STRIP
* PURPOSE       Set the PAPER & STRIP colour for the given channel ID.
* DESCRIPTION   Sets the paper & strip colour for the screen channel 
*               whose ID is passed in A0, to the colour code supplied 
*               in D0.W. Works like SuperBasic's PAPER command.
* INPUTS:
*               D0.W = colour code for paper & strip colour
*               A0.L = Channel ID.
* OUTPUTS:
*               D0 = Error code
*               A0.L = channel ID (preserved)
*               Z flag set if no errors, unset otherwise.
*---------------------------------------------------------------------
scr_paper_sb    move.w  d0,d1           ; Save the colour 
                bsr     scr_paper       ; Set the paper colour
                bne.s   spsb_exit       ; Tets for errors
                move.w  d1,d0           ; Get the colour code again
                bsr     scr_strip       ; Set the strip colour
scsb_exit       rts
\end{lstlisting}

\section{SCR\_INK}
\program{SCR\_INK}
\label{ch9-SCR_INK}%\hyperlabel{ch9-SCR_INK}%

\begin{lstlisting}[firstnumber=1,caption={SCR\_INK}]
*---------------------------------------------------------------------
* NAME          SCR_INK
*---------------------------------------------------------------------
* DEPENDENCY    None
* PURPOSE       Set the INK colour for the given channel ID.
* DESCRIPTION   Sets the ink colour for the screen channel whose ID is 
*               passed in A0, to the colour code supplied in D0.W.
* INPUTS:
*               D0.W = colour code for ink colour
*               A0.L = Channel ID.
* OUTPUTS:
*               D0 = Error code
*               A0.L = channel ID (preserved)
*               Z flag set if no errors, unset otherwise.
*---------------------------------------------------------------------
scr_ink         move.l   d1/d3/a1,-(a7) ; These will be corrupted
                move.w   d0,d1          ; Get the ink colour
                moveq    #sd_clear,d0   ; CLS
                moveq    #-1,d3         ; Infinite timeout
                trap     #3             ; Set INK colour
                move.l   (a7)+,d1/d3/a1 ; Restore registers
                tst.l    d0             ; Set Z flag if all ok
                rts
\end{lstlisting}

\section{SCR\_STRIP}
\program{SCR\_STRIP}
\label{ch9-SCR_STRIP}%\hyperlabel{ch9-SCR_STRIP}%

\begin{lstlisting}[firstnumber=1,caption={SCR\_STRIP}]
*---------------------------------------------------------------------
* NAME          SCR_STRIP
*---------------------------------------------------------------------
* DEPENDENCY    None
* PURPOSE       Set the STRIP colour for the given channel ID.
* DESCRIPTION   Sets the strip colour for the screen channel whose ID 
*               is passed in A0, to the colour code supplied in D0.W.
* INPUTS:
*               D0.W = colour code for strip colour
*               A0.L = Channel ID.
* OUTPUTS:
*               D0 = Error code
*               A0.L = channel ID (preserved)
*               Z flag set if no errors, unset otherwise.
*---------------------------------------------------------------------
scr_strip       move.l  d1/d3/a1,-(a7)  ; These will be corrupted 
                move.w  d0,d1           ; Get the paper colour
                moveq   #sd_clear,d0    ; CLS
                moveq   #-1,d3          ; Infinite timeout
                trap    #3              ; Set STRIP colour (not PAPER)
                move.l  (a7)+,d1/d3/a1  ; Restore corrupted registers
                tst.l   d0              ; Set Z flag if all ok
                rts
\end{lstlisting}

\section{COLOURS}
\program{COLOURS}
\label{ch9-COLOURS}%\hyperlabel{ch9-COLOURS}%

\begin{lstlisting}[firstnumber=1,caption={COLOURS}]
*---------------------------------------------------------------------
* NAME          COLOURS
*---------------------------------------------------------------------
* DEPENDENCY    None
* PURPOSE       Define names for the various QDOSMSQ colours
* DESCRIPTION   Not really a subroutine, a set of equates which define
*               names for the 8 colours available on a 'standard'
*               QDOSMSQ machine.
* INPUTS:      None
* OUTPUTS:     None
*---------------------------------------------------------------------
black           equ     0
blue            equ     1
red             equ     2
magenta         equ     3
green           equ     4
cyan            equ     5
yellow          equ     6
white           equ     7
\end{lstlisting}


\section{The Librarian}\program{The Librarian}
\label{ch9-librarian}%\hyperlabel{ch9-librarian}%

Ok, so there you have a few of my favourite routines, all you need
    now is a librarian to sort them out for you. Ok, I give up, here is one
    for you -{} this is very basic and not super at all (sorry for that pun) it
    is up to you to expand on this if you want.

Some suggestions would be:
\begin{itemize}[itemsep=0pt]

\item{}Make PE aware?


\item{}Add better/some error trapping.


\item{}Save the dependencies so that the user need not enter them
        manually.


\item{}Check what has just been entered with what has already been
        entered to avoid duplications.


\item{}Reduce the number of file open/closes etc (On the Library
        file)


\item{}Convert to assembler -{} Ha, now you're quaking in your boots
       !

\end{itemize}

I have omitted line numbers from the following listing.

\begin{lstlisting}[firstnumber=1,language={}]
CLS
Output = 3
INPUT 'Library name: ' LibraryName$
INPUT 'Output file name: '; Output$
OPEN_NEW #Output, Output$
:
REPeat main_loop
    INPUT 'Routine name (ENTER to quit): '; Name$
    IF (Name$ = '')
        EXIT MainLoop
    END IF
    ExtractName Name$
END REPeat MainLoop
:
CLOSE #Output
PRINT "Done."
STOP
:
:
DEF PROCedure ExtractName(ReqdName$)
    LOCal A$, Library, FoundName$
    Library = Output + 1
    OPEN_IN #Library, LibraryName$
    REPeat LibLoop
        IF (EOF #Library)
            EXIT LibLoop
        END IF
        INPUT #Library, A$
        IF (A$(1 TO 6) == "* NAME")
            FoundName$ = A$(17 TO)
            IF (FoundName$ == ReqdName$)
                PRINT "Found subroutine: " & ReqdName$
                GetDependencies(Library)
                ExtractCode(Library)
                CLOSE #Library
                RETurn
            ENDIF
        END IF
    END REPeat LibLoop
    PRINT "Cannot find: " & ReqdName$
END DEFine ExtractName
:
:
DEF PROCedure GetDependencies(Channel)
    LOCal A$
    REPeat DependLoop
        IF (EOF #Channel)
            RETurn
        END IF
        INPUT #Channel, A$
        IF (A$(1 TO 12) == "* DEPENDENCY")
            IF (A$(17 TO 20) == "None")
                PRINT "No dependencies"
                Return
            END IF
            PRINT "Dependency required: " & A$(17 TO)
        END IF
        IF (A$(1 TO 9) == "* PURPOSE"
            RETurn
        END IF
    END REPeat DependLoop
END DEFine GetDependencies
:
:
DEF PROC ExtractCode(Channel)
    LOCal A$
    REPeat FindCodeLoop
        IF (EOF #Channel)
            RETurn
        END IF
        INPUT #Channel, A$
        IF (A$(1 TO 5) == "*----"
            EXIT FindCodeLoop
        END IF
    END REPeat FindCodeLoop
    REPeat WriteCodeLoop
        IF (EOF #Channel)
            RETurn
        END IF
        INPUT #Channel, A$
        IF (A$(1 TO 5) == "*----"
            EXIT WriteCodeLoop
        END IF
        PRINT #Output, A$
    END REPeat WriteCodeLoop
    PRINT "Extracted."\\
END DEFine ExtractCode
\end{lstlisting}

\subsection{So how does this lot work?}
\label{ch9-how-it-works}%\hyperlabel{ch9-how-it-works}%

After asking for your details etc, it simply enters a loop asking
      you for the next routine to be extracted. This name is passed to
      ExtractName which opens the library file and scans it looking for all
      those lines which start `* NAME'. Once it finds one, it tests to see if
      this line includes the name you are looking for.

Note that this version of the program assumes you are using
      \emph{exactly} the same format in your comments as I am above and as per the
      following description:
\begin{itemize}[itemsep=0pt]

\item{}Column 1 = An asterisk, the comment marker for most assemblers
          I have used.

\item{}Column 2 = A space.


\item{}Columns 3 to 16 = Parameter name, eg NAME, DEPENDENCY
          etc.      

\item{}Columns 17 onwards = Parameter details etc.

\end{itemize}

If the name found is the same as the one you requested, the
      dependencies are extracted and listed. You are advised to note these
      dependencies and enter them as the next routine to extract. Try not to
      duplicate names etc as the program doesn't test for duplicates.

Once all dependencies have been listed, The code is extracted and
      written to the output file.

A sample session follows:

\begin{lstlisting}[firstnumber=1,frame=none,numbers=none]
Library name: Win1_GWASL_Library_lib
Output file name: Win1_source_MyNextProject_asm
Routine name (ENTER to quit): Colours
Found subroutine: COLOURS
No dependencies
Extracted.

Routine name (ENTER to quit): Scr_paper_sb
Found subroutine: SCR_PAPER_SB
Dependency required: SCR_PAPER
Dependency required: SCR_STRIP
Extracted.

Routine name (ENTER to quit): Scr_paper
Found subroutine: SCR_PAPER
No dependencies
Extracted.

Routine name (ENTER to quit): SCR_STRIP
Found subroutine: SCR_STRIP
No dependencies
Extracted.

Routine name (ENTER to quit):
Done.
\end{lstlisting}

So there you have it and now you can enhance it as required to
      suit your own purposes. Remember, my version expects the comments to be
      in the format given above. Additionally, no comments are written to the
      output file but you can easily amend the code in ExtractCode to do the
      needful. Enjoy.

\section{Coming Up...}
\label{ch9-the-end}%\hyperlabel{ch9-the-end}%

In the next chapter we shall be looking at the thorny subject of
    coding single and doubly linked lists in assembler.

