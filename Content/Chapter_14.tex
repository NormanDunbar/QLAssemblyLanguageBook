\chapter{Program Development}

\section{Introduction}
\label{ch14-intro}%hyperlabel{ch14-intro}%

In this chapter I'll be going through the way I tend to write my
    assembly language (and indeed, all my other languages too) programming
    from the initial thought to the `final completed' program. I put `final'
    and `completed' in quotes because programs never ever reach that stage
    there are always bugs to fix and improvements to be made.

\section{Program Development in Assembly Language}
\label{ch14-program-development}%hyperlabel{ch14-program-development}%

Program development is the art of starting from nothing more than an
    idea and progressing with various stages until the thought becomes
    reality, or is discarded as unworkable.

We don't all do things the same way, and assembly language
    programming is no different -{} we all do it in a manner that is comfortable
    for us. The following lets you have a brief glance into my own
    methods.

\subsection{The Initial Thought.}
\label{ch14-initial-thought}%hyperlabel{ch14-initial-thought}%

The initial idea for a program comes at the most inopportune
      moments I have found. I've had `great' ideas at three in the morning, at
      other times when I was in the bath reading a novel, while driving to
      work and so on. The fact is, you never know when an idea will suddenly
      appear, so be prepared and have a bit of paper and a pen handy -{} not
      while driving of course -{} to jot down your ideas before they vanish from
      memory forever.

\subsection{Work It Out.}
\label{ch14-work-it-out}%hyperlabel{ch14-work-it-out}%

Sometimes, given a little thought, the initial idea is found to be
      not so good after all and the project is abandoned there and then. Those
      ideas that get through need to be fleshed out a little to see just how
      good they are.

If they get past this stage, we can start to jot down the basic
      structure of our program. I personally tend to start with `the big idea'
      and break it down into stages before breaking these down into smaller
      stages and so on until I have a set of small (hopefully) self contained
      routines at the bottom. This is top down development and used to be
      quite popular.

\subsection{Start Writing Code.}
\label{ch14-start-coding}%hyperlabel{ch14-start-coding}%

At this point, armed with your list of routines, you can begin to
      write down your initial thoughts for the code you want to write to make
      the 'big idea' come to fruition. Having all the routines broken down by
      the previous stage, you know where repeated code can be extracted to a
      sub-{}routine and so on.

I tend to use a pencil to write code at this stage and arm myself
      with a decent rubber (eraser for my American readers!) because mistakes
      will be made. I also arm myself with three books:
\begin{itemize}[itemsep=0pt]

\item{}Andy Pennell's QDOS manual.


\item{}Andy Pennell's Assembly Language Programming book.


\item{}My trusty copy of the Motorola MC68000/MC68008 Programmer's
          Guide.

\end{itemize}

I also have a cheap narrow feint ruled A4 sized notepad to do my
      coding on. I then let my brain run away with itself to see how many
      different mistakes I can make in as short a time as possible.

Even after all these years, I still write down assembly code that
      just isn't legal syntax and this is sometimes `obvious' when I look over
      the code, but usually I notice when George's trusty assembler (GWASL)
      \program{Gwasl}complains about something in my code.

As I produce code for one routine. I usually find myself needing
      another, so I note it down on my list and carry on. This `stepwise
      refinement' of my rough draft usually produces code that will be typed
      in using my trusty \program{PFE - Programmers File Editor}PFE text editor. This isn't a QL program, it runs on
      Windows, but I've used it for many years to write code and I prefer it.
      It allows me to save code in Linux format -{} which just happens to be the
      same as the QL's format and I like it.\footnote{Since that was originally written I've converted over to Linux.}

Once I have the code typed into a file, it gets saved to my C:\textbackslash{} or
      D:\textbackslash{} drive ready for import into \program{QPC}QPC. Within \program{QPC}QPC, my code files are
      copied from the DOS\_ device to my RAM\_ disc and \program{Gwasl}GWASL is called into
      action. It almost never assembles first time.

QED\program{QED} is fired up and I make my changes to the RAM\_ version, saving
      the file to DOS\_ as a backup. Once I have a code file that actually
      assembles, I save the whole lot to WIN1\_SOURCE\_ and get ready to test it
      all out.

\subsection{Testing The Code.}
\label{ch14-testing}%hyperlabel{ch14-testing}%

I tend to look on the bright side of most things, and running my
      own code is always fun. I simply EX the binary file and see what
      happens. Usually, it's a crash or system lock-{}up and I have to reboot.
      At least rebooting \program{QPC}QPC takes a lot less time than rebooting
      Windows.

So, I know that there is at least one bug in my code and so it's
      bug hunting time again. After reloading, I run my next test with a code
      listing and \program{JMON}\program{QMON}JMON/QMON to trace through the code.

I wrote an article recently about debugging with \program{JMON}\program{QMON}JMON/QMON so I
      wont go into great detail here. (See Appendix~\ref{debugging-with-qmon2}~\nameref{debugging-with-qmon2})

Suffice to say, my initial trace starts
      off with me single stepping up to each sub-{}routine call, then let each
      sub-{}routine run as a single unit. This way, I tend to quickly find out
      where my major problem lies.

After another reboot -{} if required -{} I use the procedures outlined
      in my \program{JMON}\program{QMON}JMON/QMON article (See Appendix~\ref{debugging-with-qmon2}~\nameref{debugging-with-qmon2}) to set a break point at the 'broken'
      sub-{}routine, and I run the code to that point. From there on, I trace
      the code one line at a time until I hit a sub-{}sub-{}routine and let that
      run as a unit again. Once more, I quickly narrow my search for the main
      problem down to a single (or a couple) of small bits of code.

This code is then breakpointed and tested again, but in single
      step mode all the way through.

Eventually I either find the offending line(s) and fix them, or I
      find out which conditional branch I've got the wrong way round -{} I have
      been known to \opcode{BCC} when I should have used \opcode{BCS} and so on.

The rest of the process is similar to the above. It may not be the
      best in the world, but it works for me and I can quickly get debugged
      code finished and start `tarting' it all up.

To show how easy most of the above is, I am going to work through
      a full example of `an idea' from initial rough draft onwards to the
      finished code. I'm still working on this code at the moment and will not
      be writing up the article until I'm finished. I shall be documenting the
      process as I go and will write the article up from that.

You will no doubt have read some of my rants and raves about the
      Disassembler I'm writing as a project for this series. It has been
      developed bit by bit without any of the above `discipline' so it has
      suffered from an extremely large number of errors, some stupidity on my
      part and a couple of rewrites in places. As I've said before, this is
      not how I wanted to write the utility but I'm somewhat stuck with it
      now. It shows how much better things are when you do it
      properly.

%Next time, we'll get down to the design and writing of a small (I
%      hope) utility piece of code to make handling menus in your assembly
%      language programs a bit easier and standardised. It's not going to be a
%      world beater in programming excellence, but you'll see how I develop
%      programs from start to finish.

\section{Coming Up...}
\label{ch14-the-end}%hyperlabel{ch14-the-end}%

Coming up in the next chapter we have a discussion of problems with QDOSMSQ EXECable
files being downloaded from the internet and how to best recreate the correct dataspace
settings.

