\chapter{6800x Exceptions And Exception Handling}\index{Exceptions}

\section{Introduction}
\label{ch6-introduction}%\hyperlabel{ch6-introduction}%

In this chapter, we are going to get stuck into a fairly complex
    part of the 6800x processor's workings -{} exception processing.

\section{Exceptions}
\label{ch6-exceptions}%\hyperlabel{ch6-exceptions}%

As mentioned in the instruction summary in past articles, the QL
    processor runs in two modes -{} user and supervisor -{} and some instructions
    cannot be run in user mode without causing an exception to be generated. I
    promised to explain what these exceptions are, so here goes ....

An exception is an event or happening that causes the processor to
    deviate from its normal course of action and to jump to a predetermined
    place in the operating system where it starts executing a piece of code
    that handles such events. In QDOS (the QL's operating system) many of
    these routines have been `botched' in an effort to save on memory and
    others simply do nothing. This is unfortunate, however, all is not
    lost.

All 68000 series processors have an area of memory set aside to hold
    the exception table. This table is 1024 bytes long and holds a full set of
    exception vectors -{} basically a long word holding the address of the
    sub-{}routine that handles the appropriate exception. In QDOS this table is
    only partially there as will become clear. There are 256 vectors normally,
    each one being 4 bytes long. Vector zero is at address zero in the memory
    map and vector 255 is at address \$3FC.

The vector table \emph{should} look like Table~\ref{tab:MC6800xExceptionTable}.

\begin{table}[htbp]
\centering
\begin{tabular}{l l l} % Set this correctly ...
\toprule
\textbf{Vector} & \textbf{Address} & \textbf{Purpose}\\
\midrule
%
000 & 0000 & Reset -{} SSP value \\
001 & 0004 & Reset -{} USP value \\
002 & 0008 & Bus Error \\
003 & 000C & Address Error \\
004 & 0010 & ILLEGAL Instruction \\
005 & 0014 & Divide by zero \\
006 & 0018 & CHK instruction \\
007 & 001C & TRAPV instruction \\
008 & 0020 & Privilege violation \\
009 & 0024 & Trace \\
010 & 0028 & Line 1010 emulator \\
011 & 002C & Line 1111 emulator \\
012 & 0030 & Reserved for Motorola \\
013 & 0034 & Reserved for Motorola \\
014 & 0038 & Reserved for Motorola \\
015 & 003C & Uninitialised Interrupt \\
016 & 0040 & Reserved for Motorola \\
... & ... & ... \\
024 & 0060 & Spurious interrupt \\
025 & 0064 & Interrupt level 1 \\
026 & 0068 & Interrupt level 2 \\
027 & 006C & Interrupt level 3 \\
028 & 0070 & Interrupt level 4 \\
029 & 0074 & Interrupt level 5 \\
030 & 0078 & Interrupt level 6 \\
031 & 007C & Interrupt level 7 \\
032 & 0080 & TRAP \#0 \\
033 & 0084 & TRAP \#1 \\
034 & 0088 & TRAP \#2 \\
035 & 008C & TRAP \#3 \\
036 & 0090 & TRAP \#4 \\
037 & 0094 & TRAP \#5 \\
038 & 0098 & TRAP \#6 \\
039 & 009C & TRAP \#7 \\
040 & 00A0 & TRAP \#8 \\
041 & 00A4 & TRAP \#9 \\
042 & 00A8 & TRAP \#10 \\
043 & 00AC & TRAP \#11 \\
044 & 00B0 & TRAP \#12 \\
045 & 00B4 & TRAP \#13 \\
046 & 00B8 & TRAP \#14 \\
047 & 00BC & TRAP \#15 \\
048 & 00C0 & Reserved for Motorola \\
... & ... & ... \\
064 & 0100 & User vector 1 \\
... & ... & ... \\
255 & 03FF & User vector 192 \\
%
\bottomrule
\end{tabular}
\caption{MC6800x Exception Table}
\label{tab:MC6800xExceptionTable}
\end{table}

It can be seen that a huge number of the vectors are reserved for
    Motorola to use in future processors. The User vectors look interesting,
    but have been obliterated by some of the code in QDOS and cannot be
    used.

On the QL, the vectors are actually as per Table~\ref{tab:QDOSExceptionTable}.

\begin{table}[htbp]
\centering
\begin{tabular}{l l l l} % Set this correctly ...
\toprule
\textbf{Vector} & \textbf{Address} & \textbf{Purpose} & \textbf{Comment}\\
\midrule
%
000 & 0000 & Reset & SSP value \\
001 & 0004 & Reset & USP value \\
002 & 0008 & Bus Error & Ignored by QDOS \\
003 & 000C & Address Error & May be redefined \\
004 & 0010 & ILLEGAL Instruction & May be redefined \\
005 & 0014 & Divide by zero & May be redefined \\
006 & 0018 & CHK instruction & May be redefined \\
007 & 001C & TRAPV instruction & May be redefined \\
008 & 0020 & Privilege violation & May be redefined \\
009 & 0024 & Trace & May be redefined \\
010 & 0028 & Line 1010 emulator & Unusable \\
011 & 002C & Line 1111 emulator & Unusable \\
012 & 0030 & Reserved for Motorola & Unusable \\
013 & 0034 & Reserved for Motorola & Unusable \\
014 & 0038 & Reserved for Motorola & Unusable \\
015 & 003C & Uninitialised Interrupt & Unusable \\
016 & 0040 & Reserved for Motorola & Unusable \\
... & ... & ... & {} \\
024 & 0060 & Spurious interrupt & Ignored by QDOS \\
025 & 0064 & Interrupt level 1 & Ignored by QDOS \\
026 & 0068 & Interrupt level 2 & QL System interrupt \\
027 & 006C & Interrupt level 3 & Ignored by QDOS \\
028 & 0070 & Interrupt level 4 & Ignored by QDOS \\
029 & 0074 & Interrupt level 5 & Ignored by QDOS \\
030 & 0078 & Interrupt level 6 & Ignored by QDOS \\
031 & 007C & Interrupt level 7 & Hangs the QL -{} May be redefined \\
032 & 0080 & TRAP \#0 & Make a call to QDOS \\
033 & 0084 & TRAP \#1 & Make a call to QDOS \\
034 & 0088 & TRAP \#2 & Make a call to QDOS \\
035 & 008C & TRAP \#3 & Make a call to QDOS \\
036 & 0090 & TRAP \#4 & Make a call to QDOS \\
037 & 0094 & TRAP \#5 & Ignored by QDOS -{} May be redefined \\
038 & 0098 & TRAP \#6 & Ignored by QDOS -{} May be redefined \\
039 & 009C & TRAP \#7 & Ignored by QDOS -{} May be redefined \\
040 & 00A0 & TRAP \#8 & Ignored by QDOS -{} May be redefined \\
041 & 00A4 & TRAP \#9 & Ignored by QDOS -{} May be redefined \\
042 & 00A8 & TRAP \#10 & Ignored by QDOS -{} May be redefined \\
043 & 00AC & TRAP \#11 & Ignored by QDOS -{} May be redefined \\
044 & 00B0 & TRAP \#12 & Ignored by QDOS -{} May be redefined \\
045 & 00B4 & TRAP \#13 & Ignored by QDOS -{} May be redefined \\
046 & 00B8 & TRAP \#14 & Ignored by QDOS -{} May be redefined \\
047 & 00BC & TRAP \#15 & Ignored by QDOS -{} May be redefined \\
048 & 00C0 & Reserved for Motorola & Unusable \\
... & ... & ... & {}\\
064 & 0100 & User vector 1 & Unusable \\
... & ... & ... & {}\\
255 & 03FF & User vector 192 & Unusable \\
%
\bottomrule
\end{tabular}
\caption{QDOS Exception Table}
\label{tab:QDOSExceptionTable}
\end{table}

\begin{note}
All vectors marked ``Unusable'' have been botched in the ROM and
      have bits of code in place of the vectors. So you can see not much is
      left. The designers of QDOS didn't have enough room in the early ROMs to
      fit all the code in. Some early QLs came with a ROM `dongle' hanging out of
      the external ROM slot so that all the code could fit. 

Later versions got rid of the ROM dongle, but the exception vector table had then been
      `redesigned' to make the code fit. Luckily the developers did allow a number of
      the exceptions to be redefined so that programmers could write their own
      routines to handle these exceptions.
\end{note}

\section{Working QDOS Exceptions}
\label{ch6-qdos-exceptions}%\hyperlabel{ch6-qdos-exceptions}%


\begin{description}

\item[RESET] -{} vectors 0 and 1 -{} these two vectors are simply the values that
    are put into the SSP and USP on system power up. Vector 0 gives the value
    for the stack pointer for supervisor mode and vector 1 gives the stack
    pointer for user mode.

\item[ADDRESS ERROR] -{} this occurs whenever the processor tries to do a
    word or long sized operation or access at an odd address. For example, the
    following code fragment will cause an address error:\footnote{Well, it will on a standard QL's MC68008 processor. With QPC's emulation of an MC68020 however, it will work quite happily.}

\begin{lstlisting}[firstnumber=1,]
          MOVEA.L   #1,A1
          MOVE.W   (A1),D0
\end{lstlisting}

On a normal QL this will usually cause the system to hang, but as
    the vector can be redefined, we can use it to point to an address that can
    correctly handle this error. More on this later.

\item[ILLEGAL INSTRUCTION] -{} this occurs when an instruction is executed
    that is not a valid instruction for the processor, or when the \opcode{ILLEGAL}
    instruction is executed. Illegal usually crashes the QL, but can be
    handled by our own routines.

\item[DIVIDE BY ZERO] -{} This should be obvious. This is ignored on the QL,
    but can be redefined for our own use.

\item[CHK INSTRUCTION] -{} Called when the \opcode{CHK} instruction is used and the
    value in a data register is out of bounds, Ignored on the QL but
    redefinable.

\item[TRAPV INSTRUCTION] -{} Called when the \opcode{TRAPV} instruction is executed
    and the V flag is set. Ignored by the QL but, once again, is
    redefinable.

\item[PRIVILEGE VIOLATION] -{} When a program running in user mode attempts
    to execute an instruction that is privileged, this exception is raised.
    Ignored by the QL, but redefinable.

\item[TRACE] -{} If the trace (T) bit is set in the Status Register, this
    execption is generated after each instruction. Can be redefined to call
    code in a machine code monitor program, but usually ignored by the
    QL.

\item[INTERRUPT LEVEL 2] -{} there are 7 levels of interrupt on a normal
    68000 series processor, but only one is used on the QL. The level 2
    interrupt is generated by the internal electronics and causes the keyboard
    to be scanned, the scheduler to switch tasks etc. Levels 1 and 3 to 6 are
    ignored on the QL.

\item[INTERRUPT LEVEL 7] -{} Level 7 is the non-{}maskable interrupt and is
    raised when you press CTRL ALT 7 together. When the QL hardware was being
    built and debugged, some external equipment was `bolted on' and this
    combination of keys caused a level 7 interrupt which activated the
    debugging equipment. Unfortunately, when the QL went into production, the
    code was left in and pressing these keys together is a pretty good way to
    trash the system. May be redefined for our own use -{} this could be fun
   !

\item[TRAP \#0] -{} Switch the QL into supervisor mode and cause the SSP
    version of A7 to be used.

\item[TRAP \#1] -{} this is the QDOS manager trap and is used to control
    resources in the QL such as baud rates, jobs, memory allocation and
    deallocation etc.

\item[TRAP \#2] -{} this is the QL's I/O manager trap and is used to open
    \& close channels as well as formatting discs and deleting
    files.

\item[TRAP \#3] -{} This allows QDOS to read data from channels, queues, set
    colours etc.

\item[TRAP \#4] -{} Used by the SuperBasic interpreter to switch between A6
    relative and Absolute addresses when calling various routines.

\item[TRAP \#5] to \textbf{TRAP \#15} -{} these are unused on the QL but can be
    redefined.

\end{description}

\section{What Happens When an Exception Occurs?}
\label{ch6-exception-processing}%\hyperlabel{ch6-exception-processing}%

When an exception occurs, some data is put onto the stack prior to
    the exception being processed. Remember, the stack pointer is the SSP and
    not the normal USP variant of the A7 register.

For most exceptions, the data put onto the stack is simply the
    program counter and the status register as follows:

\begin{lstlisting}[firstnumber=1,frame=none,numbers=none,language={}]
High address -> PC low word 
                PC high word 
SSP ----------> Status register word
\end{lstlisting}

so when the exception handler is running, the stack pointer holds
    the value of the SR at the time the exception was caused and 4(A7) holds
    the program counter where the exception was caused.

The above is true for all exceptions apart from BUS ERROR, ADDRESS
    ERROR or RESET. These three have a different stack frame:

\begin{lstlisting}[firstnumber=1,frame=none,numbers=none,language={}]
High address -> PC low word 
                PC high word 
                Status register word 
                Instruction register word 
                Access address low word 
                Access address high word 
SSP ----------> Access type and function code (one word)
\end{lstlisting}

This additional data includes a copy of the first word of the
    instruction that was being processed when the exception was caused, the
    address that was being accessed when the exception was caused and a word
    describing what the processor was trying to do at the time.

\begin{warning}
Note that the value in the program counter on the stack is not
      always the \emph{actual} address of the start of the
      instruction -{} it could be anything from the next word or even the
      address 10 bytes on from the actual address of the instruction -{}
      beware.
\end{warning}

At the end of an exception processing routine an \opcode{RTE} instruction is
    used to restore the status register and the program counter from the
    stack. It follows then that in the case of an ADDRESS or BUS exception
    that this is going to fail unless the additional data is first cleared
    from the stack -{} or a 68020 used instead!

\section{Building an Exception Handler.}
\label{ch6-exception-handler}%\hyperlabel{ch6-exception-handler}%

I suppose we need to built an exception handler now! In the QL you
    build a table of vectors for the following exceptions:

\begin{lstlisting}[firstnumber=1,frame=none,numbers=none,language={}]
Address error 
Illegal 
Divide by zero 
CHK 
TRAPV 
Privilege 
Trace 
Interrupt level 7 
Trap #5 
Trap #6 
Trap #7 
Trap #8 
Trap #9 
Trap #10 
Trap #11 
Trap #12 
Trap #13 
Trap #14 
Trap #15
\end{lstlisting}

And then tell QDOS to use this table for your job. Any exceptions
    that are generated and that are mentioned above will be handled by your
    own routine. Of all of these, the address error needs to have special
    treatment because it has the extra data on the stack.

The problem being that if your instruction caused an error, what
    happens when you handle the exception and \opcode{RTE} -{} does the program fail
    again because it tried to execute the same instruction again? Sometimes
    is the only answer.

The following code will be very useful when you first start writing
    assembler as it will trap the exceptions mentioned above and attempt to
    allow you to carry on. This example should be run on a 68000 or 68008
    ONLY. I do not have the data for exception handling on a 68020 or above so
    Gold Cards, Super Gold Cards etc may cause problems. I don't know.

\section{The Exception Handler Code.}
\label{ch6-handler-code}%\hyperlabel{ch6-handler-code}%
\program{QL Exception Handler}

\begin{lstlisting}[firstnumber=1,caption={Exception Handler for the QL},label={lst:ExceptionHandlerQL}]
*================================================================
* This code adds a `protective barrier' to the QL so that silly 
* programming errors can be intercepted and hopefully handled 
* before the QL crashes out.
*
* This code should only be run on a 68000 or 68008 as the 
* exception stack frame is (probably) different on 68010 and 
* above.
*
* Copyright (c) Norman Dunbar 1999 but permission for unlimited 
* use and abuse is given!
*================================================================

start       lea exceptions,a1       ; Table of exceptions - empty
            lea x_address,a2        ; Address exceptions
            move.l a2,(a1)+         ; Save in table

            lea x_illegal,a2        ; Illegal exceptions
            move.l a2,(a1)+         ; Save in table

            lea x_divide,a2         ; Divide by zero
            move.l a2,(a1)+         ; Save in table

            lea x_check,a2          ; CHK instruction
            move.l a2,(a1)+         ; Save in table

            lea x_trapv,a2          ; TRAPV instruction
            move.l a2,(a1)+         ; Save in table

            lea x_priv,a2           ; Privilege violation
            move.l a2,(a1)+         ; Save in table

            lea x_trace,a2          ; Trace exception
            move.l a2,(a1)+         ; Save in table

            lea x_int_7,a2          ; Interrupt level 7
            move.l a2,(a1)+         ; Save in table

            lea x_trap,a2           ; TRAP #5 to TRAP #15
            moveq #10,d0            ; 11 entries to fill

trap_loop   move.l a2,(a1)+         ; Save one in table
            dbra d0,trap_loop       ; Then do the rest

            lea exceptions,a1       ; Exceptions table, now full
            moveq #-1,d1            ; Job id = 'this job'
            moveq #mt_trapv,d0      ; Set exception table
            trap #1                 ; And do it
            rts                     ; Exit to SuperBasic

*================================================================
* Now the actual exception handlers themselves. Apart from the 
* ADDRESS exception, all have 3 words on the stack when called. 
*================================================================

x_address   lea t_address,a1    ; Message to print
            bsr message_0       ; Print the message
            addq.l #8,A7        ; Tidy extra data off the stack
            rte                 ; Attempt to continue

t_address   dc.w 15
            dc.b 'ADDRESS error.'
            dc.b 10

x_illegal   lea t_illegal,a1    ; Message to print
            bsr message_0       ; Print the message
            addq.l #2,2(a7)     ; Don't do the instruction again!
            rte                 ; Attempt the next instruction

t_illegal   dc.w 21
            dc.b 'ILLEGAL instruction.'
            dc.b 10


x_divide    lea t_divide,a1     ; Message to print
            bsr message_0       ; Print the message
            rte                 ; Attempt to carry on

t_divide    dc.w 16
            dc.b 'DIVIDE BY ZERO.'
            dc.b 10


x_check     lea t_check,a1      ; Message to print
            bsr message_0       ; Print the message
            rte                 ; Attempt to carry on

t_check     dc.w 17
            dc.b 'CHK instruction.'
            dc.b 10


x_trapv     lea t_trapv,a1      ; Message to print
            bsr message_0       ; Print the message
            rte                 ; Attempt to carry on

t_trapv     dc.w 19
            dc.b 'TRAPV instruction.'
            dc.b 10


x_priv      lea t_priv,a1       ; Message to print
            bsr message_0       ; Print the message
            rte                 ; Attempt to carry on

t_priv      dc.w 21
            dc.b 'PRIVILEGE VIOLATION.'
            dc.b 10


x_trace lea t_trace,a1          ; Message to print
            bsr message_0       ; Print the message
            rte                 ; Attempt to carry on

t_trace     dc.w 25
            dc.b 'TRACE - not implemented.'
            dc.b 10


x_int_7 lea t_int_7,a1          ; Message to print
            bsr message_0       ; Print the message
            rte                 ; Attempt to carry on

t_int_7     dc.w 26
            dc.b 'DO NOT PRESS CTRL ALT 7!'
            dc.b 10


x_trap      lea t_trap,a1       ; Message to print
            bsr message_0       ; Print the message
            rte                 ; Attempt to carry on

t_trap      dc.w 39
            dc.b 'TRAP #5 to TRAP #15 - not implemented.'
            dc.b 10


*================================================================
* This routine prints a message to channel 0. The message is at 
* 0(A1) in the usual QDOS format of a size word followed by the 
* text. The UT_ERR0 routine expects an error code in D0 -or- the 
* address of a user defined error message in D0 with bit 31 set 
* to show that it is user defined.
*================================================================
message_0   move.l a1,d0        ; User defined message address
            bset #31,d0         ; Mark it as such
            move.w ut_err0,a2   ; Vectored routine
            jsr (a2)            ; Print the exception message
            moveq #0,d0         ; Ignore any errors
            rts

exceptions  ds.l    19          ; Space for 19 exception handlers

*================================================================
* HERE ENDETH THE CODE.
*================================================================
\end{lstlisting}

\section{How it Works.}
\label{ch6-how-it-works}%\hyperlabel{ch6-how-it-works}%

Now that you have typed the above code into a file, I shall explain
    what is happening. The code begins at the label `start' and sets A1 to the
    address of the label `exceptions' within the program. This is where the
    \opcode{LEA} instruction is useful -{} when writing position independent programs.
    These are programs that can be run at any address and are a requirement if
    you want to write good QDOS programs.

The `exceptions' label identifies the start of the 19 long words of
    data that hold the addresses of the 19 redefined exception vectors as
    detailed above. At the moment the table contains random garbage and needs
    to be initialised \emph{before} we tell QDOS to use the new vectors.

The address of the routine to handle address exceptions,
    `x\_address', is loaded into A2 -{} again using position independent methods,
    and then placed in the table at the first location. You will note that
    `address register with post-{}increment' addressing is used here. This means
    that A1 is automatically incremented by the correct amount -{} 4 in the case
    of the long sized move -{} ready for the next vector to be loaded.

This process is repeated for the illegal, divide by zero, \opcode{CHK},
    \opcode{TRAPV}, privilege violation, trace and interrupt level 7 vectors.

There are 11 vectors left in the table for \opcode{TRAP \#5} through to \opcode{TRAP
    \#15}. Rather than give each of these an individual handler, we point them all to
    the same one as we intend to ignore these instructions when they occur. To
    set these 11 vectors up, we run through a small loop which counts D0 down
    from 10 to -{}1 setting the vector for each of the 11 TRAP exceptions to be
    the single routine at address x\_trap.

Our exceptions table has now been defined and all we have to do is
    tell QDOS that we want to use it. Once again, A1 is set to the start
    address of the exceptions table as required by QDOS, D1 is then set to -{}1
    which implies `the current job' to QDOS. This is used in many of the QDOS
    routines which require a job ID, passing -{}1 means `me'. As we are
    executing this code directly from SuperBasic, that is what the current job
    will be. Once the vectors have been set up for any job, all other jobs
    created by it will use the same vector table.

This means that as the initiating job is SuperBasic, and as most
    other jobs are created by SuperBasic, we have effectively
    created a protection mechanism for every job in the system created \emph{from this point onwards}! If this is the first code loaded on your system, then
    every single job created will be protected by this code.

Trap \#1 is called with D0 set to the value \trap{MT\_TRAPV} -{} a fancy way of
    saying 7 -{} and we return to SuperBasic with any error codes that may
    arise. As there appears to be only `invalid job' returned, it is unlikely
    that there will be any as we are using the current job's own id.

Now that the initialisation has been carried out, the exception
    handlers will just sit there until such time as they are activated.

Most of the handler code is the same -{} we simply trap the exception,
    print a warning message to channel \#0 and attempt to carry on -{} but the
    Address and illegal exception handlers do additional processing.

In the case of an address error, there is an extra 8 bytes of data
    on the stack on top of the `standard' stack frame as discussed above.
    These need to be cleared off before we execute the \opcode{RTE} instruction.

\begin{warning}
this is only true of a QL with 128K or a Trump Card etc. If you
      use QXL or some other card with an upgraded processor, then the stack is
      different and this code won't work properly.
\end{warning}

An Illegal instruction also manipulates the stack, but this time, it
    adds 2 to the address of the failed instruction. This prevents it from
    trying to execute it again when we exit the routine. Of course this may
    not always be successful and can cause further errors along the way -{} if
    the instruction was followed by a word of data for example. Trying to
    execute the data could lead to another exception and so on. What would you
    rather have, a message telling you about it or a lock up with no
    indications?

The messages are defined in the standard QDOS manner of a size word
    followed by the bytes of the message. The appropriate message has its
    address loaded into A1 by the exception handler, and a branch is made to
    the sub-{}routine MESSAGE\_0 which will attempt to display a message to
    channel \#0. If this fails, it will try \#1 before giving up.

If you have a QDOS manual and you look up \vector{UT\_ERR0} (that's a zero by
    the way!) you will see that it takes an error code in D0 as its only
    parameter. We are using it slightly differently as we are defining our own
    messages and not using the Sinclair defined ones such as `invalid channel
    id' or `bad parameter' etc.In order to do this, we load D0 with the
    address of the message but set bit 31 of D0 so that QDOS knows that it is
    an address and not an error code.

The UT\_ERR0 routine lives in the ROM somewhere, I don't know where
    it lives in all ROMs as it could have been moved between ROM releases.
    Because of this, there is a vector table in the ROM at a standard
    position. To get the address of the routine, we simply read the contents
    of the vector table into an address register and \opcode{JSR} to that address.
    (This will be explained later in the series when I cover QDOS).

So now that we have assembled the code all we do is LRESPR it (or
    RESPR(512), LBYTES and then CALL) and that is it. Whenever any exceptions
    occur, the above code will handle them and, most importantly, tell you
    what has happened. Your QL may still be hung -{} but at least you should
    know why!

\section{Coming Up...}
\label{ch6-the-end}%\hyperlabel{ch6-the-end}%

The is the end of quite a complex section on exceptions and their
    handling. If you have stuck with me this far, the rest should be quite a
    lot simpler -{} well, at least until we get to the graphics stuff that is
   :o)

However, the next chapter delves into the features of QDOS that
    allow the humble programmer the ability to write their own assembly
    language procedures and functions to extend the workings of SuperBasic.
    Along the way, the very important maths stack will be examined in quite a
    lot of detail
