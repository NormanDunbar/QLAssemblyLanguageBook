\begin{appendix}

\chapter{How this book Evolved}
\label{appendix-a-how-it-was-done}%\hyperlabel{appendix-a-how-it-was-done}%

This book started life on my PC as text files -{} one for each chapter, well, for a while it did. Eventually, I decided to convert to creating the files directly in \program{DocBook}Docbook format as my main source code, and \emph{those} were converted to text format prior to sending them off to Dilwyn and/or Geoff for inclusion in the QL Today magazine.

These initial non-DocBook text files were manually edited to wrap paragraphs and listings and warnings etc in the appropriate \program{DocBook}Docbook syntax. It's quite amazing how much more difficult it is to change text into \program{DocBook}Docbook than the other way around!

Once all the chapters were Docbook'd, I ran them through a validator to ensure that they were indeed valid XML \emph{and} valid \program{DocBook}Docbook.

The validated chapters were gathered together into a `book' -{} as defined by \program{DocBook}Docbook -{}
and the raw XML processed by a utility named \program{Publican}Publican which allowed me to create numerous different output formats from the same input source file. Sadly, I had a few problems with \program{Publican}Publican - it's a great tool, don't get me wrong, and they even have a version for Windows (it's free too!) but it wasn't really what I needed. I decided to go for a proper system instead. (I still use \program{Publican}Publican for other work.)

\subsection*{Enter \LaTeX{}}

\LaTeX{}\program{\LaTeX{}} is a text processing system as opposed to a PDF generator, it is much used by scientists around the world, amongst others. If it's good enough for CERN, it's good enough for me.

This actual book that you are reading now was slightly different. It was further processed by a utility called dblatex\program{dblatex} to produce  \LaTeX{}\program{\LaTeX{}} format source text and converted into a book by the \program{TeXstudio}TeXstudio application.

I think you'll agree that using \LaTeX{}\program{\LaTeX{}} creates a far nicer version of PDF etc. That version of the book was released into the wild around Christmas 2014. 

Since then, I have manually gone through all the different chapter files to remove the old DocBook\program{DocBook} conversion routines etc, and replace them with plain \LaTeX{}\program{\LaTeX{}} ones instead. This reduced the amount of code that I had to have lying around. The final result is a set of plain \LaTeX{}\program{\LaTeX{}} source files.

These source files were collected together and merged into a \LaTeX{}\program{\LaTeX{}} template designed for books, the \emph{Legrand Orange Book}, which I modified quite a lot to produce the wonderfully typeset version of the pdf book that you are reading. 

There's a lot of work goes into this you know! ;-)

\chapter{Debugging with QMON2}
\label{debugging-with-qmon2}%\hyperlabel{debugging-with-qmon2}%

Many years ago while still working on the Project - QLTdis\program{QLTdis} - I had a small problem. \opcode{ADDX} and \opcode{SUBX} were being decoded as \opcode{ADD} or \opcode{SUB}  when I tested a file containing \opcode{ADDX} and
\opcode{SUBX} instructions. What was going wrong? Well the original code looked like the
following:

\begin{lstlisting}[firstnumber=1,caption={QLTdis Broken Code}]
dtype_24    btst    #8,d0           ; If bit 8 is 0, can't be ADDX/SUBX
            beq.s   t24_not_t30     ; Easy bit done
            move.w  d0,d4           ; Need D4 to hold the op-code

            andi.w  #$00c0,d0       ; Mask bits 7 & 6 of the op-code
            cmpi.w  #$00c0,d0       ; Both set?
            bne.s   t24_not_t30     ; No, skip over type 30 stuff
\end{lstlisting}

As the original programmer of this code, when I read through it, everything
seemed fine -{} as it always does -{} but obviously, something was amiss. What to do?

The rest
of this exciting article, is a brief foray into the art of debugging using
QMON2\program{QMON2}.

QMON2\program{QMON2} is Tony Tebby's original disassembler/monitor tool which allows a QDOSMSQ
job code or SuperBasic extension or CALLed code to be debugged by single
stepping through the guts of the code until you find the bit that isn't doing
what it is supposed to be doing.

I have been using QMON2\program{QMON2} to help me debug code for years and although I don't use
it as often as I should perhaps, I do happen to like it quite a lot. It seems,
unfortunately, that it is no longer available in its English format as Digital
Precision still hold the rights to the program -{} as far as I am aware -{} but in
Germany, you can get a copy from Jochen. Actually, you can get a copy from
Jochen in any country in the world, provided you are able to read and understand
German manuals.

QMON2\program{QMON2} is fine, but as we don't yet have anything like a source code debugger on
the QL, it is a bit difficult to figure out where to put breakpoints in your
code so that you don't spend ages single stepping through code you know works to
find the bit that doesn't work.

George Gwilt has provided a little help here, so not only does he supply you
with a neat little assembler but he also gives you help in debugging as well.

When you have assembled the code for QLTdis\program{QLTdis} there is a listing file created with
the \_LST extension, but another file is created with a \_SYM extension. This file
holds the goodies we need to debug.

The SYM file is binary and holds a list of all your equates in it, plus a list
of all the program labels and their offset from the start of the program. So, if
you think that you have a bug in a specific routine, all you have to do is
decode the SYM file to extract the routine's offset from the start of the
program and set a breakpoint at that place in the code. The problem is, how
exactly do you decode the binary file?

George does not document the SYM file format, so you could assemble a few
routines and see if you can make any sense of the binary file, but there is a
much easier way. Simply by running the SYM\_BIN program supplied with GWASL\program{Gwasl} you
feed it a SYM file and it spits out a text file holding all the data you will
ever need. The output file is named the same as the SYM file but with a further
\_LST extension, so I have `dev2\_source\_qltdis\_sym\_lst' as my file.

The following is a small extract from this file on my system. Yours may well
look different, but don't worry if it does. The first part of the file matches
up with my equates:

\begin{lstlisting}[firstnumber=1,caption={QLTdis Symbol List}]
CON_ID                        EQU   $00000000
CON_ID2                       EQU   $00000004
PRT_ID                        EQU   $00000008
PC_ADDR                       EQU   $0000000C
PC_END                        EQU   $00000010
BLACK                         EQU   $00000000
RED                           EQU   $00000002
GREEN                         EQU   $00000004
WHITE                         EQU   $00000007
LINEFEED                      EQU   $0000000A
OOPS                          EQU   $FFFFFFFF
ERR_NC                        EQU   $FFFFFFFF
INFINITE                      EQU   $FFFFFFFF
ME                            EQU   $FFFFFFFF
\end{lstlisting}

Then we get to the nitty gritty, the labels I have used in my source code and
their offsets from the start of the program. The first one is my label `start'
and it is actually the very first instruction in the file, so it has offset
zero. Following on are all the other labels I used.

\begin{lstlisting}[firstnumber=1,caption={QLTdis Symbol List}]
START                         EQU   *+$00000000
QLTDIS                        EQU   *+$00000010
JOB_INIT                      EQU   *+$0000003E
EXIT                          EQU   *+$0000003A

.... a few dozen lines removed for brevity!

DTYPE_23                      EQU   *+$00000B9A
DTYPE_24                      EQU   *+$00000BAC

.... another few dozen lines removed for brevity!

\end{lstlisting}


We can see that regardless of the start address of the program when loaded
into memory (by QMON2\program{QMON2} or JMON2\program{JMON2}) we can still work out where the code for the
DTYPE\_24 routine, for example, starts simply by adding \$0BAC to the actual start
address of the program.

The following is a small session showing how I debugged through my DTYPE\_24
routine to fix the above mentioned problem. 

So, to set the scene, I have edited the source code for the type 24
instructions, assembled QLTdis\program{QLTdis} and produced a new listing of the
SYM file. I've looked through the listing and found that my entry point for
DTYPE\_24 is at offset \$0BAC. I then start up JMON2\program{JMON2} (in this case, but QMON2\program{QMON2} is
exactly the same):

\begin{lstlisting}[firstnumber=1,caption={Debugging QLTdis with Jmon2}]
jmon 'win1_source_qltdis_qltdis_bin'
\end{lstlisting}

If you try this and get an error, make sure you have LRESPR'd the JMON\_BIN code
for JMON2\program{JMON2} or the QMON\_BIN code for QMON2\program{QMON2} depending on which one you want to use.

When the monitor appears, the very first instruction in the job has already been
executed, so I could be anywhere in the job file. Because I have written the
code myself, I know what the very first instruction is, it is \opcode{BRA.S QLTDIS}.
Because I know this, I know that the instruction I am looking at must be the
code at label `QLTDIS'.

If I was debugging some other code that I did not have the original nicely
commented source files for, then I would not know where I was in the actual job,
or extension, as the first instruction could have sent me off into any location
in its own code or even into the ROM.

In this case I have jumped from label `START' to label `QLTDIS' and there are
quite a few bytes between the two labels. QMON2\program{QMON2} is showing me a register dump and
the address of, the op-{}code word and the next instruction to be executed. For
the sake of brevity, I've omitted the register dump itself.

\begin{lstlisting}[frame=none,numbers=none,]
1A0EB8 6100 BSR.L $1A0EE6
\end{lstlisting}

So, I'm somewhere in the code for QLTdis\program{QLTdis}, but where. I know I'm at the
instruction at address \$1A0EB8 but what is the start address of the job itself?

The QMON2\program{QMON2} command `C' will calculate an address and the option `S' will display
the start address of the job.

\begin{lstlisting}[frame=none,numbers=none,]
QMON> C S
001A0EA8    1707688
\end{lstlisting}

This is the Hexadecimal and decimal values for the start of the QLTdis job I'm
trying to debug. How can I be sure? Try disassembling the start address for a
couple of instructions:

\begin{lstlisting}[frame=none,numbers=none,]
QMON> DI S 5
1A0EA8 600E BRA.S $1A0EB8
1A0EAA 0000 ORI.B #0,D0
1A0EAE 4AFB ILLEGAL
1A0EB0 0006 ORI.B #$4C,D6
1A0EB4 5464 ADDQ.W #2,-(A4)
\end{lstlisting}

The first line is the one to look at, it shows a branch to address \$1A0EB8 that
QMON2\program{QMON2} was showing me originally as the second instruction to be executed. So, the
`S' value does appear to be my label for `START' and this is what I want.

So, I know that the routine I want to check out is `DTYPE\_24' and that it is at
an offset of \$0BAC from start, what address is this? Again using the C command
to calculate an address, I do this:

\begin{lstlisting}[frame=none,numbers=none,]
QMON> C S+$0BAC
001A1A54    1710676
\end{lstlisting}

I now know where my routine starts, again, to check that it is so, I can
disassemble the first few instructions:

\begin{lstlisting}[frame=none,numbers=none,]
QMON> DI S+$0BAC 5
1A1A54 0800 BTST #$8,D0
1A1A54 6732 BEQ.S $1A1A8C
1A1A54 3800 MOVE.W D0,D4
1A1A54 0240 ANDI.W $C0,D0
1A1A54 0800 CMPI.W $C0,D0
\end{lstlisting}

This looks remarkably like the correct code to me, so I can now set a breakpoint
at this address and let QMON2\program{QMON2} tell me when I get there. Of course, if I was
debugging someone else's code, I wouldn't have a handy list of offsets into the
program, so I would have to run through it step by step by step until I found
out where the code I wanted to check was. Once I'd reached that stage, I would
make a note of the address and calculate the offset from the start so that I
could easily set a breakpoint there on my next foray into the debugging
session. It's much easier when you have the source!

Anyway, I set a breakpoint as follows using the `B' command.

\begin{lstlisting}[frame=none,numbers=none,]
QMON> B S+$0BAC
BRP 1A1A54
\end{lstlisting}

I could also have simply used the calculated address from earlier by typing `B
\$1A1A54' which would have had the same effect. Note that if I set a break point
at the same address it will delete the breakpoint at that address. The `B'
command is a toggle.

Again, this is what my code originally looked like when I was debugging the fix
for this instruction type:

\begin{lstlisting}[firstnumber=1,caption={QLTDis Broken Code}]
dtype_24    btst    #8,d0           ; If bit 8 is 0, can't be ADDX/SUBX
            beq.s   t24_not_t30     ; Easy bit done
            move.w  d0,d4           ; Need D4  with the op-code

            andi.w  #$00c0,d0       ; Mask bits 7 & 6 of the op-code
            cmpi.w  #$00c0,d0       ; Both set?
            bne.s   t24_not_t30     ; No, skip over type 30 stuff
\end{lstlisting}

Now I'm ready to go, so I simply type the QMON2\program{QMON2} go command which is `G'.

\begin{lstlisting}[frame=none,numbers=none,]
QMON> G
\end{lstlisting}

The `G' command means, Go until you hit a breakpoint or finish the program. It
causes the program to run at nearly full speed. This means I get all the clear
screens and prompts etc that I would normally get when running the program
without the debugger. I therefore need to enter a start address and so on to get
the disassembler to start working.

I have already loaded a file of assembled \opcode{ADDX} and \opcode{SUBX} instructions into an
area of memory that I allocated with ALCHP and I have its address written down
on paper -{} my own memory is a bit random these days.

After I have typed in the start and end addresses (and the printer device) I
return to the QMON prompt with a register dump and the address, hex code and
decoded instruction for the next instruction to be executed:

\begin{lstlisting}[frame=none,numbers=none,]
At brp SR 0000 --0----- SSP 00028480
D0-D3 0000D300 01BC0924 0000003C FFFFFFFF
D4-D7 0013FFFF 00000000 00000003 0013D300
A0-A3 004C0016 001A1601 001A11DD 001A1130
A4-A7 001A2362 001A11DD 0013A3C8 001A32Fa
1A1A54 0800 BTST #$8,D0
QMON>
\end{lstlisting}

Taking the above a section at a time, we have this first:

\begin{lstlisting}[frame=none,numbers=none,]
At brp SR 0000 --0----- SSP 00028480
\end{lstlisting}

This is telling me that I'm stopped at a breakpoint -{} `at brp' -{} and the
contents of the status register in hex -{} 0000. Next to that is the interrupt
mask value -{} 0 then 5 dashes showing the current state of the CCR flags. As all
are showing dashes, none of the flags are set. Finally, there is the current
value of the `alternative' stack pointer. In this case I'm running in user mode,
so I can see the SSP (supervisor stack pointer) value. 

Below the status line is a register dump showing the current values of all data
and address registers.

\begin{lstlisting}[frame=none,numbers=none,]
D0-D3 0000D300 01BC0924 0000003C FFFFFFFF
D4-D7 0013FFFF 00000000 00000003 0013D300
A0-A3 004C0016 001A1601 001A11DD 001A1130
A4-A7 001A2362 001A11DD 0013A3C8 001A32Fa
\end{lstlisting}

In my case I have the registers split over two lines each for data and address
values. This depends on the width of the channel to which QMON2\program{QMON2} is writing the
register dump.

Below the register dump is the address, the op-{}code word and the disassembled
instruction for the next instruction to be executed. Under that is the QMON2\program{QMON2}
prompt.

\begin{lstlisting}[frame=none,numbers=none,]
1A1A54 0800 BTST #$8,D0
QMON>
\end{lstlisting}

Back to the debugging session. I want to know what is causing my \opcode{ADDX}
instructions to be decoded as \opcode{ADD}. So, I have my source listing for Type\_24
instructions, and because I've hit the breakpoint I set, I know that an \opcode{ADDX} is
coming through the type\_24 decoding routine before jumping into the type\_30
decode -{} or is it? I need to find out.

The register dump shows me the op-{}code in D0.W and also in D7.W, it is \$D300
which is \opcode{ADDX D0,D1}. The op-{}code in binary is as follows, the bit numbers are in
HEX above the individual bits themselves:

\begin{lstlisting}[frame=none,numbers=none,]
   C    8    4    0
1101 0011 0000 0000
\end{lstlisting}

Lets trace through the code and see what happens. Remember that the next
instruction to be executed is showing just above the QMON prompt, so when I
enter the `T' for Trace command, I will be executing the instruction \opcode{BTST
\#8,D0}. Let's do it.

\begin{lstlisting}[frame=none,numbers=none,]
QMON> T
 SR 0000 --0----- SSP 00028480
1A1A58 6732 BEQ.S $1A1A8C
QMON>
\end{lstlisting}

I'm not showing the register dumps, unless there is anything of interest in the
registers.

We have tested bit 8 of D0 and found that it is not zero because the Z flag is
not showing in the list of flags. This has to be an \opcode{ADDX}, \opcode{ADD} or an \opcode{ADDA.L}
instruction. Let's step again.

\begin{lstlisting}[frame=none,numbers=none,]
QMON> T
 SR 0000 --0----- SSP 00028480
1A1A5A 3800 MOVE.W D0,D4
QMON>
\end{lstlisting}

Nothing of interest here, step again:

\begin{lstlisting}[frame=none,numbers=none,]
QMON> T
 SR 0008 --0-N--- SSP 00028480
1A1A5C 0240 ANDI.W #$C0,D0
QMON>
\end{lstlisting}

Now it's starting to get interesting, the `N' flag is showing after we moved
D0.W to D4.W -{} this shows that the most significant bit of the new value in D4.W
is set and thus the value in D4.W is negative (if using signed arithmetic!).
This is how QMON2\program{QMON2} displays flags which have been set, the flag letter is
displayed on the `SR' line.

Step again:

\begin{lstlisting}[frame=none,numbers=none,]
QMON> T
 SR 0004 --0--Z-- SSP 00028480
D0-D3 00000000 01BC0924 0000003C FFFFFFFF
D4-D7 Ommitted
A0-A3 Ommitted
A4-A7 Ommitted
1A1A60 0C40 CMPI.W #$C0,D0
QMON>
\end{lstlisting}

So, we have set the Z flag because D0.W is now holding zero (Although D0.L is
holding zero, the upper word was already zero only the lower word has changed
because the instruction just executed \opcode{AND}ed a word value with DO.W.) The next
instruction is waiting to be executed so lets do it. Step again:

\begin{lstlisting}[frame=none,numbers=none,]
QMON> T
 SR 0009 --0-N--C SSP 00028480
1A1A64 6626 BNE.S $1A1A8C
QMON>
\end{lstlisting}

It looks like we are going to take the branch as the Zero flag is not set. Lets
remind ourselves of what the original source code looked like again:

\begin{lstlisting}[firstnumber=1,caption={QLTdis Broken Code}]
dtype_24    btst    #8,d0           ; If bit 8 is 0, can't be ADDX/SUBX
            beq.s   t24_not_t30     ; Easy bit done
            move.w  d0,d4           ; Need D4 with the op-code word

            andi.w  #$00c0,d0       ; Mask bits 7 & 6 of the op-code
            cmpi.w  #$00c0,d0       ; Both set?
            bne.s   t24_not_t30     ; No, skip over type 30 stuff
\end{lstlisting}

So you can see where we have single stepped through the above code, and we are
just about to jump to label `T24\_NOT\_T30' because this instruction is not a
type\_30. Except, we know that it is an \opcode{ADDX} instruction because that is what I
was testing, and \opcode{ADDX} is a type\_30, so what have I done wrong?

I have tested bits 7 and 6 and found them both to be zero (because the Z flag
was set after I stepped through the \opcode{ANDI.W \$C0,D0} instruction. This means that
the jump should not be taken to T24\_NOT\_T30 because I have not yet ascertained
that the instruction is not an \opcode{ADDX}. With bits 7 and 6 set to 00, I could be
looking at \opcode{ADDX} or \opcode{ADD}. I should not be taking the jump until I have further
tested the value in bits 5 and 4 as per my algorithm above.

This could be why the \opcode{ADDX} is being decoded as \opcode{ADD}, because I have the wrong
condition in my test. In order to fix this, I have to change the source code,
re-{}assemble and try my test again. I do this without the QMON2\program{QMON2} first of all and
if it still fails, I can use QMON2\program{QMON2} to try and find out why again. I need to give
the current job a `G' instruction and then I can ESC from the decoding and exit
the program.

I shall go do that and report back. Hang on here for a bit ......

Ok, I'm back. I made the change from \opcode{BNE.S} to \opcode{BEQ.S} and it worked fine. So
it looks like I have correctly identified the bug. I need more testing though
to make sure I cover all possible op-{}codes. I have followed up my \opcode{ADDX} testing
by passing test files which have \opcode{ADD}, \opcode{ADDA}, \opcode{ADDQ} and \opcode{ADDI} instructions, along
with assorted \opcode{SUB} variants and all appears to be working well.

So there you have it, an example of how I manage to get my code wrong and how I
can use the tools available to try to sort it out. As I mentioned earlier, QMON2
is available from Jochen for a small fee, but only if you understand German
manuals.

Laurence (Lau) Reeves has a different version of QMON2\program{QMON2}, written by himself,
which fixes some bugs but I don't know if this is widely available or if it
comes with a manual. Perhaps he could be persuaded to part with it or make it
available -{} who knows. I'm not sure if he ever wrote a manual for it though.

\end{appendix}
