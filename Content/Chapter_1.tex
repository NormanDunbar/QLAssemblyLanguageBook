\chapter{QL Assembly Language Programming}

\section{Introduction}
\label{ch1-intro}%\hyperlabel{ch1-intro}%

Assembly language is very, very simple.

Not many people will agree at first, but if you think about it, it
    is. You have to tell the processor what you want it to do in very simple
    steps. In SuperBasic, you can multiply two numbers together easily -{} you
    can do it almost as easily in machine code too.

This series of articles is intended to let you in on the basic
    secrets of programming your QL in its own natural language -{} machine code
    or assembly language. (Actually, machine code is what the QL talks, we use
    assembly language which is `English' sounding `words' that get converted
    to machine code by an assembler -{} I will tend to use the two terms as
    one.) To talk directly to the QL, you must learn its language. This series
    should hopefully teach you how to do just that.

I make no assumptions about how much or how little you may already
    know -{} I will start very simple and continue from there. Hopefully you
    will have an assembler, but if not, see below!

I was going to base the series on George Gwilt's
 GWASS\program{Gwass} assembler, which is free and can be
    distributed easily. Unfortunately, it won't run on anything less than a
    68020 which is no good for those of us who are still running on an
    original QL. George, however, has supplied another of his assemblers, 
 GWASL\program{Gwasl} for use in the series. Thanks George.

Most assembly language books tend to give little example programs as
    they go along to try to show the bits of the instruction set that you have
    just learned about. I will attempt to do likewise.

\section{The 6800x Processor}
\label{ch1-6800x}%S\hyperlabel{ch1-6800x}%

The processor we are programming is one of Motorola's 68000 series.
    Be it a 68008 or a 68060 (if you are lucky) all of them have the same
    basic instruction set, although some of the more powerful processors have
    additional instructions. Partly because we have to cater for those on an
    original QL but mostly because I don't have a clue about these additional
    instructions, we will be dealing with the basic instruction set -{} there is
    enough there to keep us happy for a while. Inside the processor there are
    a few different parts, but we are only concerned with the registers -{} the
    rest just does the work and puts the results somewhere, setting a few
    flags along the way. Talking of flags, we will also take a look at the
    status register -{} a very important part of programming.

\subsection{Registers}

Registers are where numbers get loaded into, manipulated and
      written out from. Some instructions operate directly on memory
      locations, but to all intents and purposes, memory is just another
      register but outside of the processor and a lot slower. The 68000 -{}
      which is the term I shall use from now on to describe the entire family
      of processors -{} has different types of registers -{} data, address, status
      and program counter. Data held in registers and in memory is held in
      High Order format. This simply means that the numbers are stored in a
      similar manner to the way in which we would expect them to be -{} the
      `rightmost' end holds the most significant bit and the `leftmost' the
      lowest -{} just the way we write numbers down.

\subsection{Data Registers}

There are 8 data registers named D0 to D7 and these can be used to
      perform manipulations on the numbers that are held in them. Each
      register can hold 32 bits of information. (A bit is a single binary
      digit -{} basically a one or a zero). What these bits actually represent
      depends on the program running at the time. Data registers are normally
      used for manipulating data in the form of bytes, words and long words -{}
      these being 8, 16 and 32 bits long respectively.

\subsection{Address Registers}

There are 9 address registers named A0 to A7. A7 is sometimes
      known as the stack pointer or SP register. What about the other address
      register then? The ninth address register is a duplicate of A7 and is
      the SSP or Supervisor Stack Pointer. When coding the chip, you only have
      access to 8 address registers at any one time -{} you are either using the
      SP or SSP version of A7 but never both at the same time.

Address registers are normally used to hold memory addresses,
      stack pointers etc and cannot be used for byte sized
      manipulations.

\subsection{Status Register}\mc6800x{Status Register}\mc6800x{Flags}

The status register holds a list of flags to tell the processor
      what is happening or has happened internally. The status register is a
      16 bit register in two 8 bit halves. The user byte is held in bits 0 to
      7 ( the lowest end) and the system byte is held in the upper half or
      bits 8 to 15. The layout of the system byte is shown in Figure~\ref{fig:StatusRegisterSystemByte}.


\begin{figure}[h]
\centering
\begin{tabular}{|r|c|c|c|c|c|c|c|}
\hline
Bit 15 & 14 & 13 & 12  & 11 & 10 & 9 & 8 \\
\hline
T & & S & & & I & I & I \\
\hline
\end{tabular}

\caption{Status Register -{} System Byte.}\mc6800x{Status Register!System Byte}
\label{fig:StatusRegisterSystemByte}
\end{figure}

\begin{itemize}[itemsep=0pt]

\item{}Bit 15, `T' is the trace flag -{} this defines whether the
            processor is in `single step' mode or running normally. If set to
            1, the processor is tracing and if 0, is running normally. In
            trace mode the processor `stops' after each instruction has been
            executed and jumps to the Trace exception routine. Exceptions are
            covered later in the series.


\item{}Bit 13, `S' is the supervisor flag -{} this defines whether
            the code being executed is running in user or supervisor mode. If
            set, the processor is in supervisor mode otherwise it is in user
            mode.


\item{}Bits 10, 9 and 8, `III' is the interrupt mask and represents
            a value between 0 and 7 and indicates which of the seven interrupt
            levels are enabled. 

\item{}The other bits in the system byte are not used.

\end{itemize}

\begin{figure}[h]
\centering
\begin{tabular}{|r|c|c|c|c|c|c|c|}
\hline
Bit 7 & 6 & 5 & 4  & 3 & 2 & 1 & 0 \\
\hline
 & & & X & N & Z & V & C \\
\hline
\end{tabular}

\caption{Status Register -{} User Byte.}\mc6800x{Status Register!User Byte}
\label{fig:StatusRegisterUserByte}
\end{figure}
The user byte contains the 5 condition
        code flags which are set or reset by certain instructions and then
        used by arithmetic or comparison instructions. The are used to tell
        later parts of a program what happened recently. The program can
        adjust its operations to suit. Figure~\ref{fig:StatusRegisterUserByte} shows the layout of the user byte, the various flag bits are:
\begin{itemize}[itemsep=0pt]

\item{}Bit 4, `X' is the extended flag. Which is very similar to
            the `C' flag bit but is affected by fewer instructions than `C'
            is. This is used when carrying out very large sized arithmetic
            instructions -{} such as 64 bit adds, for example. When affected it
            is set exactly like the `C' flag.


\item{}Bit 3, `N' is the negative flag. It gets set to 1 if the
            last instruction created a negative number.


\item{}Bit 2, `Z' is the zero flag and is set to 1 if the last
            instruction generated a result of zero.


\item{}Bit 1, `V' is the overflow flag and is set to 1 if the last
            instruction generated an overflow during 2's complement
            arithmetic. See later for details.


\item{}Bit 0, `C' is the carry or borrow flag. And is used when a
            subtraction operation is carried out -{} be it an actual subtraction
            or an implied one.

\item{}The other bits in the user byte are not used.

 \end{itemize}

The flags are used by the branch on condition (\opcode{Bcc})
        instructions, the Decrement and branch (\opcode{DBcc}) instructions or the Set
        (\opcode{Scc}) instructions. These will be explained later.

\subsection{The Program Counter}\mc6800x{Program Counter}

The program counter does just that, it keeps track of where
      exactly the processor is within a program. The program counter always
      points to the address in memory of the next instruction to be executed.
      The program counter can of course be changed by a \opcode{JMP} (jump) instruction
      or a \opcode{BRA} (branch) but it is always ready with the next instruction to be
      executed.

\section{Addressing Modes}
\label{ch1-addressing}%\hyperlabel{ch1-addressing}%

The 68000 has a large number of addressing modes and these can often
    become overwhelming to a new machine code programmer -{} I know. It takes
    some time to understand each and every mode, what it does and why it is
    used. Having said that, you do not need to remember all of their names,
    just what they look like in source code and of course, what they
    do.

From here on, you need to be aware that numbers may be in decimal
    format or hexadecimal. All hexadecimal numbers are prefixed with the
    dollar sign (\$) and wherever this is seen in front of a number (or in some
    cases, what appears to be a word) will be a hexadecimal number. (I will
    assume that you are familiar with hex.) A couple of examples of
    hexadecimal numbers are:

\begin{lstlisting}[frame=none,numbers=none]
$100
$C0FFEE
\end{lstlisting}

which are equivalent to 256 and 12,648,430 respectively.

Without any further hesitation, lets dive right in with the
    addressing modes ...

\subsection{Register Direct}\address{Register Direct}

This is an easy one to start off with. Register direct addressing
      mode simply means that both the source and the destination in the
      instruction are registers either data, address or a mixture of
      both.

Simple examples are:

\begin{lstlisting}[firstnumber=1,]
          MOVE.L    A2,D1
          MOVE.W    D0,D1
          MOVE.L    A1,A3 
\end{lstlisting}

These simply move (actually, they copy) data between various
      registers. The full meaning of the actual instructions will be described
      later on.

\subsection{Absolute}\address{Absolute}

In this mode, the operand of the instruction is simply a memory
      address. This is also quite simple. For example to `zeroise' the
      contents of the first byte of screen memory (assuming a standard QL and
      this is the last time that I will assume anything!)

\begin{lstlisting}[firstnumber=1,]
          CLR.B    $20000  
\end{lstlisting}

There are two variations to this mode, absolute short and absolute
      long. If the address given is a 16 bit word (ie 0 to 7FFF hex or 32767
      decimal) then it refers to addresses in the first 32K of memory. If the
      address given is 8000 hex or 32768 decimal and upwards it refers to
      address FFFF8000 and upwards due to sign extension of the address word.
      This is absolute short, best used for addresses of 0 to 7FFF hex only -{}
      to avoid confusion.

\begin{lstlisting}[firstnumber=1,]
          MOVE.L    $1000,D1    get a long from address $1000
          MOVE.L    $9000,D1    get a long from address $FFFF9000  
\end{lstlisting}

The other variation is absolute long, in this case, the address
      given is a full 32 bits long and refers to the actual address in memory
      -{} there is no ambiguity with absolute long. \lstinline{MOVE.L $123456,D1} -{} gets the
      long word at address \$123456.

\subsection{Relative}\address{Relative}

This mode will probably be the most used with QL programs as all
      code should be relocatable. This means that it never assumes that it is
      running at a specific location in memory. Some early QL programs were
      written to run at a specific location in memory and this caused no end
      of problems when memory expansions became available. I think Psion chess
      was one of the guilty ones.

However, relative addressing simply means, relative to where the
      program counter is. The program counter is always pointing at the
      address of the instruction in memory after the current one. An example
      of relative addressing is this small loop and the jump back to the start
      of the loop:

\begin{lstlisting}[firstnumber=1,]
Start    MOVE.W #1000,D0
Loop     SUBQ #1,D0
         BNE.S Loop(PC)  
\end{lstlisting}

This is a small and totally useless fragment of code. The relative
      address mode is in the \lstinline{BNE.S LOOP(PC)} instruction -{} it says -{} branch to
      the label called `loop', relative to where the program counter is
      currently pointing, if the result of the subtraction was not zero. The
      jump is specified in the code as a negative number, not the actual
      address of where the label `loop' is at.

This negative number (in the example above) is how many bytes are
      to be added to the program counter to get the address of the next
      instruction to be executed. The jump can be forwards as well as
      backwards.

Using relative addressing means that the program can be loaded
      anywhere in memory and still work. If absolute addressing was used, the
      program would always have to be loaded at the same address if a crash
      was to be avoided. The example above is the equivalent of the following
      SuperBasic code:

\begin{lstlisting}[firstnumber=1,language={[Visual]Basic}]
1000 REMark Start
1010 LET D0 = 1000
1020 REMark Loop
1030 LET D0 = D0 - 1
1040 IF D0 <> 0 THEN GOTO (1040 - 10)
\end{lstlisting}

\begin{note}
Because of the slightly different way that assembler works, the
        calculation of the destination line is not quite accurate. When the
        \opcode{BNE.S} instruction is being executed, the program counter is already
        set to the following instruction. In the SuperBasic example above, the
        subtraction of 10 from 1040 should really be 20 from 1050. However, it
        shall remain as above for now.
\end{note}

\subsection{Address Register Indirect}\address{Address Register Indirect}

This mode is called `indirect' because the address register in
      question is not the operand in the instruction. It simply serves as a
      pointer to the operand. In an earlier example we cleared out the first
      byte of screen memory by using absolute addressing like this:

\begin{lstlisting}[firstnumber=1,]
          CLR.B    $20000 
\end{lstlisting}

This instruction could have been carried out using address
      register indirect mode as follows:

\begin{lstlisting}[firstnumber=1,]
          MOVEA.L    #$20000,A1
          CLR.B      (A1) 
\end{lstlisting}

All that this is doing is setting address register A1 with the
      value 131072 (decimal) which is 20000 (hexadecimal). It then clears out
      the first byte at that address. This is the same as this SuperBasic
      example:

\begin{lstlisting}[firstnumber=1,language={[Visual]Basic}]
1000 LET A1 = 131072
1020 POKE A1,0 
\end{lstlisting}

The register's name is put in between a pair of brackets to
      signify that it is the memory address held in the register that will be
      acted upon and not the register itself.

\subsection{Register Indirect With Displacement}\address{Register Indirect With Displacement}

This mode is similar to the above, except that a displacement is
      added or subtracted from the address register to give the final address
      to be operated upon. Using the above example again, we can zeroise the
      first 4 byes of screen memory as follows: 
\begin{lstlisting}[firstnumber=1,]
          MOVEA.L    #$20000,A1
          CLR.W      (A1)
          CLR.W      2(A1) 
\end{lstlisting}


This time we use word sized operations, these simply affect 16
      bits instead of 8 as with the byte sized operations. The displacement is
      the number outside of the brackets and it is added to the address
      registers contents to create the address to be operated upon. The
      displacement can be any signed number that will fit into 16 bits.
      (-{}32768 to +32767)

\subsection{Register Indirect With Displacement And Index}\address{Register Indirect With Displacement And Index}

It's starting to get complicated now. This is another mode where
      we have an address register and a displacement to consider, but this
      time we have an index as well. In this case the displacement has been
      reduced to 8 bits only giving a range of -{}128 to +127. The format of
      this addressing mode is:

\begin{lstlisting}[firstnumber=1,]
          CLR.W    2(A2,A0.L)
\end{lstlisting}

The contents of A2 is added to A0 to get the first address then
      the displacement is added to give the final result. The 16 bits of
      memory at the final address is cleared out. (Like POKE\_W A2 + A0 + 2,
      0). In this case the entire 32 bit value of A0 is added to A2, this is
      indicated by the `.L' after the second register -{} the index.

If the suffix had been omitted or was `.W' (which is the default
      if omitted) then the lower 16 bits of A0 would have been used instead of
      the whole 32. Take note that the 16 bits will be `sign extended' to a
      full 32 bits and this can have unpleasant side effects if the value in
      bit 15 is a 1 as this will cause a negative index to be generated. There
      will be more on sign extension later.

The first register specified is always treated as 32 bit (.L) and
      does not require a `.L' suffix -{} most, if not all, assemblers will
      reject it anyway.

\subsection{Register Indirect With Pre Decrement Or Post Increment}\address{Register Indirect With Pre Decrement Or Post Increment}

These addressing modes are used for stack operations, usually. The
      format of the pre-{}decrement instruction is: 
\begin{lstlisting}[firstnumber=1,]
          MOVE.L    D0,-(A7) 
\end{lstlisting}
       And for post-{}increment it is: 
\begin{lstlisting}[firstnumber=1,]
          MOVE.L    (A7)+,D0 
\end{lstlisting}


The actions carried out are as follows for pre-{}decrement: The
      value in A7 is decremented (reduced) by the size of the data to be
      stored (byte, word or long) then the contents of D0 are stored at the
      location pointed to by A7. In SuperBasic this equates to the following
      code fragment: 
\begin{lstlisting}[firstnumber=1,language={[Visual]Basic}]
1000 LET A7 = A7 - 4
1010 POKE_L A7, D0 
\end{lstlisting}


The opposite action takes place with post-{}increment, as follows:
      The contents of the memory address pointed to by A7 is copied into D0,
      then the address held in A7 id incremented by the size of the data just
      copied ( byte, word or long). Again, this equates to: 
\begin{lstlisting}[firstnumber=1,language={}]
1000 LET D0 = PEEK_L(A7)
1010 Let A7 = A7 + 4 
\end{lstlisting}


\subsection{Immediate}\address{Immediate}

This is probably the simplest of all the addressing modes. It
      simply means that the data specifies the address value. For example:
 
\begin{lstlisting}[firstnumber=1,]
          MOVE.L    #100,D0 
\end{lstlisting}


This copies the value of 100 into data register D0. The hash sign
      indicates that the data is copied directly into the register. Do not get
      confused between this instruction and: 
\begin{lstlisting}[firstnumber=1,]
          MOVE.L    100,D0 
\end{lstlisting}


Note that there is no hash. This instruction means load the
      \textit{contents} of address 100 into register D0. This is not the same! This is
      a good source of confusion for beginners -{} I know all about it, and
      sometimes still make this mistake!

\section{Coming Up...}
\label{ch1-the-end}%\hyperlabel{ch1-the-end}%

In the next chapter, we will take a closer
    look at some of the actual instructions in the 68000 instruction
    set.
