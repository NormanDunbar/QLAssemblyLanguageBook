\chapter{Single Linked Lists Demo Code}
\index{Linked Lists!Single Link - Demo Code}
 
\section{Introduction}
\label{ch11-intro}%hyperlabel{ch11-intro}%

The following code demonstrates the use of single linked lists. It
    should be slotted into the test harness code wrapper in Chapter 10 at the
    appropriate place. It cannot be assembled as it stands -{} it needs to be
    part of the test harness.

\section{How Does The Code Work?}
\label{ch11-single-code}%hyperlabel{ch11-single-code}%

The code is a small example of creating and navigating a linked
    list. The demo starts by creating a list of 5 nodes, each holding one long
    word of data being simply the node number 0 to 4.

The list contents are then printed on the screen shoing the node
    address, the next pointer and the data stored in that node. Once this is
    done, the node with data contents of 3 is located and deleted prior to the
    new list being printed out again.

Finally, each node in the list is deleted.

The first part of the code simply controls the whole demo by calling
    various sub-{}routines to do the hard work, display messages etc on
    screen.

\begin{lstlisting}[firstnumber=1,caption={Single Linked List - Demo Code}]
* =====================================================================
* The DEMO code starts here.
*
* This demo does the following:
*
* Creates a number of nodes and stores a LONG value in each one.
* Lists each node address, its NEXT pointer and data value on screen.
* Deletes a node.
* Lists each node address, its NEXT pointer and data value on screen.
* Finds a node based on its data and displays its details on screen.
* Deletes all the nodes from the list.
* =====================================================================
Demo       bsr     BuildList            ; Build a linked list
           bsr     Before               ; Display 'BEFORE:'
           bsr     ShowList             ; Display list details
           bsr     FindNode             ; Locate a single node
           bne.s   DemoAfter            ; Failed to find node data = 3
           bsr     DeleteNode           ; Delete a single node

DemoAfter  bsr     After                ; Display 'AFTER: '
           bsr     ShowList             ; Show details again
           bsr     KillList             ; Delete entire list
           rts                          ; Done
\end{lstlisting}

Following on from the main control section of the demo, we have our
    much beloved root node which must be initialised to zero as outlined in
    the original article. This is the pointer we will be loading into A0 quite
    often in the demo and it holds the address of the first node in the list.
    At present, there is no list, so the contents are set to zero to indicate
    the very end of the list.

\begin{lstlisting}[firstnumber=last,caption={Single Linked List - Demo Code - Root Node}]
* ---------------------------------------------------------------------
* A location to hold a single long word pointing to the first 'real'
* node in our linked list. This must be initialized to zero.
* ---------------------------------------------------------------------
RootNode   dc.l    0                    ; This is our root node.
\end{lstlisting}

The first of our sub-{}routines follows on. This part builds a list of
    5 nodes in the most simple manner possible -{} it runs a loop which calls
    the sub-{}routine to create a single node and link it into the list. If you
    want a bigger list, change the counter loaded into D7 to one less than the
    number of node you want. Don't forget to adjust the height of your window
    as wll if you want to see all the results on screen at the same
    time.

\begin{lstlisting}[firstnumber=last,caption={Single Linked List - Demo Code - Build List}]
* ---------------------------------------------------------------------
* Build a list of 5 nodes each holding a long word of data.
* ---------------------------------------------------------------------
BuildList  lea     RootNode,a0          ; Pointer to root node address
           moveq   #4,d7                ; How many nodes in D7 = 5
           moveq   #8,d1                ; Each node is 8 bytes long

BuildLoop  bsr.s   BuildNode            ; Create a node, address in A1
           bne     all_done             ; Just die on errors
           move.l  d7,4(a1)             ; Store data value
           bsr.s   AddNode              ; Add to list
           dbra    d7,BuildLoop         ; Do the rest
           rts                          ; Done
\end{lstlisting}

Here's the first of the real list routines. We add a new node into
    the list in the manner outlined in the article. We reject attempts to add
    the rot node into the list -{} but without flagging any errors -{} and, as
    explained, we don't attempt to check if the new node already exists in the
    list because we are creating the node on the heap, so the chances of the
    new node being present already are pretty slim to say the least.

\begin{lstlisting}[firstnumber=last,caption={Single Linked List - Demo Code - Add Node}]
* ---------------------------------------------------------------------
* AddNode - Adds a new node to a list. See text for details.
* Preserves all regsiters.
* No errors returned.
* ---------------------------------------------------------------------
AddNode    cmpa.l  a0,a1                ; Don't add the root node again
           beq.s   AddExit              ; Bale out quietly if attempted
           move.l (a0),(a1)             ; Save first node in new NEXT
           move.l  a1,(a0)              ; Store new node in root node
AddExit    rts
\end{lstlisting}

A new node is built by allocating some space on the common heap but
    we must preserve the working registers, the following code does this for
    us.

\begin{lstlisting}[firstnumber=last,caption={Single Linked List - Demo Code - Build Node}]
* ---------------------------------------------------------------------
* Allocate a single new node
* On entry, D1.L is amount of memory required.
* On exit, A1 holds the address of the new node, with D0 holding errors.
* All registers preserved - unless otherwise stated.
* ---------------------------------------------------------------------
BuildNode  movem.l d1-d3/a0/a2-a3,-(a7) ; Save working registers
           moveq   #MT_ALCHP,d0         ; Set the trap
           moveq   #me,d2               ; I want it for me
           trap    #1                   ; Do it
           move.l  a0,a1                ; Node address into A1
           movem.l (a7)+,d1-d3/a0/a2-a3 ; Restore working registers
           tst.l   d0                   ; Set flags
           rts                          ; Exit
\end{lstlisting}

The following sub-{}routine is called once to display the linked list
    before we do the deletions and again after we have deleting a node. The
    code simply walks through the entire list and prints out the node address,
    the next pointer and the data value by calling other sub-{}routines.

\begin{lstlisting}[firstnumber=last,caption={Single Linked List - Demo Code - Show List}]
* ---------------------------------------------------------------------
* Walk through a linked list displaying the details of each node as we
* go.
* On entry, A0 = root node of the list.
* ---------------------------------------------------------------------
ShowList    lea    RootNode,a0          ; Root node address

ShowLoop    move.l (a0),a0              ; Get address of the next node
            cmpa.l #0,a0                ; Done?
            beq.s  ShowExit             ; Yes
            move.l a0,-(a7)             ; Preserve A0 - it's our node
            bsr.s  ShowNode             ; Display that node's details
            move.l (a7)+,a0             ; Restore the node pointer 
            bra.s  ShowLoop             ; Do the rest of the list

ShowExit    rts                         ; Done
\end{lstlisting}

This next short routine is called with the address of a node in A0.L
    and prints the details of that node to the screen.

\begin{lstlisting}[firstnumber=last,caption={Single Linked List - Demo Code - Show Node}]
* ---------------------------------------------------------------------
* Display the details of a single node in the linked list.
* On entry A0 = the node address.
* ---------------------------------------------------------------------
ShowNode    move.l a0,a5                ; The node address
            move.l con_id2(a4),a0       ; The channel address
            move.l a5,d4                ; The node address
            bsr.s  ShowAddr             ; Print node address
            move.l (a5),d4              ; The NEXT pointer
            bsr.s  ShowNext             ; Print NEXT pointer
            move.l 4(a5),d4             ; The node data
            bsr    ShowData             ; Print the data
            rts
\end{lstlisting}

Obviously, just displaying the list contents isn't very user
    firiendly, so the next couple of routines display a title which informs
    the user if the list being displyed is `before' or `after' the deletion
    of a node.

\begin{lstlisting}[firstnumber=last,caption={Single Linked List - Demo Code - Show Before and After States}]
* ---------------------------------------------------------------------
* Display 'BEFORE:' in the output channel.
* ---------------------------------------------------------------------
Before     lea     BeforeAddr,a1        ; The prompt
           movea.l con_id2(a4),a0       ; Needs channel id
           bsr     Prompt               ; Print it
           rts

BeforeAddr dc.w    B4End-BeforeAddr-2
           dc.b    'BEFORE:',linefeed
B4End      equ     *

* ---------------------------------------------------------------------
* Display 'AFTER:' in the output channel
* ---------------------------------------------------------------------
After      lea     AfterAddr,a1         ; The prompt
           movea.l con_id2(a4),a0       ; Needs channel id
           bsr     Prompt               ; Print it
           rts

AfterAddr  dc.w    AftEnd-AfterAddr-2
           dc.b    linefeed,linefeed,'AFTER:',linefeed
AftEnd     equ     *
\end{lstlisting}

There now follows one of three separate but short routines to
    display on screen, the various parts of a list node. This one simply
    displays the node's address in memory. Following after this routine is a
    number of small sub-{}routines which assist in the displaying of node data
    by converting the contents of D4 to hex and printing it to the
    screen.

\begin{lstlisting}[firstnumber=last,caption={Single Linked List - Demo Code - Show Addresses}]
* ---------------------------------------------------------------------
* Display the node's actual address in memory.
* On entry D4 = the node address.
* ---------------------------------------------------------------------
ShowAddr   lea     MsgAddr,a1           ; Our prompt

ShowPrompt bsr     Prompt               ; Print it
           bsr.s   D4ToHex              ; Convert D4.L to hex
           bsr.s   PrintHex             ; Print it and a linefeed
           rts

MsgAddr    dc.w    AddrEnd-MsgAddr-2
           dc.b    linefeed,'Node address: '
AddrEnd    equ     *

* ---------------------------------------------------------------------
* Print the contents of the buffer to screen.
* ---------------------------------------------------------------------
PrintHex   lea     Buffer,a1            ; What to print
           move.l  con_id2(a4),a0       ; Channel to print to
           bsr     Prompt               ; Do it
           rts

* ---------------------------------------------------------------------
* Convert the long word in D4 to hex ready for printing
* ---------------------------------------------------------------------
D4ToHex    lea     buffer+2,a1          ; Buffer address
           bsr     hex_l                ; Do all 4 bytes = 8 characters
           lea     buffer,a1            ; Buffer again
           move.w  #8,(a1)              ; Store text length
           rts
\end{lstlisting}

The second and third routines to diplsy the details of a node now
    follow. Starting with the code to show the node's NEXT pointer address
    closely followed by the code to print the actual data stored in the
    node.

\begin{lstlisting}[firstnumber=last,caption={Single Linked List - Demo Code - Show Next Address}]
* ---------------------------------------------------------------------
* Display the node's NEXT address in memory.
* On entry D4 = the node's NEXT pointer.
* ---------------------------------------------------------------------
ShowNext   lea     MsgNext,a1           ; Our prompt
           bra.s   ShowPrompt           ; Print it

MsgNext    dc.w    NextEnd-MsgNext-2
           dc.b    '  NEXT pointer: '
NextEnd    equ     *

* ---------------------------------------------------------------------
* Display the node's actual data content.
* On entry D4 = the data.
* ---------------------------------------------------------------------
ShowData   lea     MsgData,a1           ; Our prompt
           bra.s   ShowPrompt           ; Print it

MsgData    dc.w    DataEnd-MsgData-2
           dc.b    '  Data value: '
DataEnd    equ     *
\end{lstlisting}

Next we have the code to locate a single node in the linked list
    based upon the data part of the node. This part is simply the setup
    routine for the following code at FindANode which does the actual scanning
    of the node and calling the compare routine as described in the previous
    chapter.

\begin{lstlisting}[firstnumber=last,caption={Single Linked List - Demo Code - Find Node}]
* ---------------------------------------------------------------------
* Locate a node in the list based on its data value.
* On exit, A1 is the required node's address plus Z set - if found.
*          A1 is undefined plus Z clear - if not found.
* ---------------------------------------------------------------------
FindNode   lea     RootNode,a0          ; Pointer to root node in list
           lea     Compare,a1           ; Node comparison routine
           moveq   #3,d1                ; We are looking for this data
           bsr.s   FindANode            ; Go find it, or not
           rts

* ---------------------------------------------------------------------
* This routine expects the following input registers to be able to scan
* a linked list for the required data and return the address of the
* node holding that data with the Z flag set if found, or the Z flag
* cleared for not found.
*
* A0.L = Rootnode of the list.
* A1.L = Address of Compare routine.
* D1.L = Value to look for in list.
* ---------------------------------------------------------------------
FindANode  moveq   #oops,d0             ; Assume not found (yet)

FindLoop   cmpa.l  #0,a0                ; Reached the end yet?
           beq.s   FindExit             ; Yes, node not found, exit

           move.l  (a0),a3              ; Get NEXT node into A3.L
           jsr     (a1)                 ; Call the comparison routine
           beq.s   FindFound            ; Looks like we found our node

FindNext   move.l  (a0),a0              ; A0 = NEXT node in the list
           bra.s   FindLoop             ; Go around again

FindFound  movea.l a3,a1                ; This is the required node
           moveq   #0,d0                ; Clear error flag

FindExit   tst.l   d0                   ; Set zero flag for success
           rts

* ---------------------------------------------------------------------
* This is the simple compare routine for our FindNode code. On entry, 
* we have the following registers set:
*
* D1.L = The value we want to find in a node in the list.
* A3.L = The address of the node we are searching.
*
* We must preserve A0, A1 and D1.
*----------------------------------------------------------------------
NData      equ   4                      ; Offset to the data
Compare    cmp.l NData(a3),d1           ; Found the data yet?
           rts                          ; Exit with Z set if so
\end{lstlisting}

This next routine is not really required on QDOSMSQ as a terminating
    job always has any allocated heap areas returned to the system by the job
    scheduler routines. Because I'm a lazy typist and in order that I reduce
    the large amounts of code in the magazine, I'm not writing any code here!

If you wish to carry out the list tidying explicitly for yourself as
    an exercise, feel free to do so. As a suggestion, start a loop which keep
    going around the list fetching the NEXT node pointer and deleting that
    from the list using the routines in this code. Once the node has been
    unlinked from the list, you may deallocate its heap area -{} but don't
    forget to preserve those registers.

\begin{lstlisting}[firstnumber=last,caption={Single Linked List - Demo Code - Kill List}]
* ---------------------------------------------------------------------
* QDOSMSQ tidies up rather nicely for us on exit - so we don't have to!
* ---------------------------------------------------------------------
KillList    rts
\end{lstlisting}

The folowing code sets up the demo to delete the node that was just
    `found' by searching for the node holding data 3. This code is called with
    the address of the `3' node in A1.L and it simply sets up the following
    routine which actually scans the list looking to make sure that the node
    we are deleting exists in the list.

\begin{lstlisting}[firstnumber=last,caption={Single Linked List - Demo Code - Delete Node}]
* ---------------------------------------------------------------------
* A demo routine to delete the node whose address is passed in A1.L. 
* Sets Z if found & deleted, clears it otherwise.
* ---------------------------------------------------------------------
DeleteNode  lea     rootnode,a0         ; Address of the root node
            bsr.s   DelANode
            rts
\end{lstlisting}

This is the node deletion code itself. As described in the article,
    we must not delete the root node itself -{} as this isn't allocated on the
    heap. We must also check that the node is in the list by scanning from
    start to finish looking for the node in the list which has a NEXT pointer
    holding the address of the node we want to delete.

We remove a node from the list by copying the soon to be deleted
    node's NEXT pointer into the NEXT pointer of the node before it, thus
    bypassing the node we want to delete.

\begin{warning}
This code only deletes a node from the linked list. It does not
      deallocate the memory on the common heap that was allocated to create
      the node. QDOSMSQ will do this at the end of the demo, but in real life,
      you would need to carry out this task yourself -{} especially as you may
      not want a number of deleted heap areas hanging around in memory
      fragmenting your heap.
\end{warning}

\begin{lstlisting}[firstnumber=last,caption={Single Linked List - Demo Code - Deleting A Node}]
* ---------------------------------------------------------------------
* Routine to delete a node with the address passed in A1.L from the 
* list whose address is passed in A0.L. On exit, Z flag will be set if 
* deleted or cleared if not.
* ---------------------------------------------------------------------
DelANode    moveq   #oops,d0            ; Assume it's going to fail
            cmpa.l  a1,a0               ; Deleting the root node?
            beq.s   DelExit             ; Exit if so.

DelLoop     cmpi.l  #0,(a0)             ; Finished yet?
            beq.s   DelExit             ; Exit not found
            cmpa.l  (a0),a1             ; Found the previous node
*                                       ; to the one we want to delete?
            bne.s   DelNext             ; Not yet, try again

DelFound    move.l  (a1),(a0)           ; Delete the node - set NEXT
*                                       ; of the node BEFORE the one to
*                                       ; be deleted to NEXT of the
*                                       ; node that is being deleted.
            moveq   #0,d0               ; Indicate found and deleted
            bra.s   DelExit             ; Set Z flag on the way out

DelNext     move.l  (a0),a0             ; Get the next node in the list
            bra.s   DelLoop             ; And try again

DelExit     tst.l   d0                  ; Set or clear Z flag
            rts

* =====================================================================
* The DEMO code ends here.
* =====================================================================
\end{lstlisting}

And that is all there is to it. The SingleList demo code should be
    assembled and run in the normal fashion. You'll be able to see that there
    are indeed 5 nodes in the list (in the BEFORE section at the top of the
    screen) then under that, the AFTER section shows a missing node with data
    content 3 -{} we have deleted it from the list.

\section{Coming Up...}
\label{ch11-the-end}%hyperlabel{ch11-the-end}%

In the next chapter the real working demo code for doubly linked
    lists will be shown and explained.

