\chapter{Dedication}

This book is dedicated initially to my wife Alison, who has
    tolerated my need to 'play' with computers for many many years. She
    doesn't understand how or why I can work with computers all day, then come
    home and still want to 'play' with them in my free time. It's a woman
    thing -{} they just don't understand :o)

This book is also dedicated to small army of dedicated users of the
    Sinclair QL computer. It's been around since 1984 and celebrated its 25th
    birthday in 2009. Here's hoping it can manage another 25 years.

Actually, I wouldn't mind another 25 years myself!
\setcounter{secnumdepth}{5}
\addtocontents{toc}{\protect\setcounter{tocdepth}{5}\ignorespaces}
\setcounter{tocdepth}{5}
\setcounter{secnumdepth}{-1}
\addtocontents{toc}{\protect\setcounter{tocdepth}{-1}\ignorespaces}
\setcounter{tocdepth}{-1}

\chapter{Where did this book come from?}

The various chapters in this book were originally published in the
    QL Today magazine. The magazine has been running for over 16 years since
    the demise of all the other QL based commercial magazines on the market at
    the time. This one has now survived longer than all the commercial
    offerings -{} so we must be doing something right!

Dilwyn Jones, the then temporary editor (he lasted 9 and a bit years
    -{} we have a strange definition of 'temporary' in the QL world) originally
    contacted me back in the mid 1990's and asked me to put together a couple
    of articles on beginning assembly language programming. Little knowing how
    long it would last, I agreed and started writing.

Unfortunately, I never had the opportunity to really plan what I was
    going to do and I think I bit off a bit more than I could chew at the time
    (I'm still in the same position today!) so the 'flow' of the following
    chapters may not be quite as nice as it could have been. On the other
    hand, had I planned it all in fine detail, I still wouldn't have started
    writing the first chapter.

The lack of planning has led to a number of articles in the series
    being prefixed by quite a few bug reports outlining problems and downright
    errors in the article (or articles) that went before. While this is good
    for increasing my 'word count', it's not good for the reader. The
    following chapters have been 'bug fixed' so that you should be able to
    trust what you read in each chapter.

Where other readers of QL Today have made comments on the articles,
    these are included -{} if appropriate -{} at the relevant place.

Hopefully, you will find the following useful.
\setcounter{secnumdepth}{5}
\addtocontents{toc}{\protect\setcounter{tocdepth}{5}\ignorespaces}
\setcounter{tocdepth}{5}
\setcounter{secnumdepth}{-1}
\addtocontents{toc}{\protect\setcounter{tocdepth}{-1}\ignorespaces}
\setcounter{tocdepth}{-1}

\chapter{Preface}
\label{introduction}\hyperlabel{introduction}%

Learning QL Assembly Language at home, by yourself, is never going to
  be easy, I know, it's how I learned. There are not many places you can go to
  (at least where I live) to learn this sort of thing. Most computer courses
  are all about learning Windows or Microsoft Office or similar. Not much call
  for Assembly Language then!

However, all that stops right here! You are still learning by
  yourself, but you are not alone! Reading through back issues of QL Today (or
  QL Toady as it is affectionately known after a slight spelling mistake in a
  posting to the ql-{}users mailing list) is never going to be easy -{} especially
  if you are on holiday and have limited luggage allowance. To this end, I
  have taken my original articles from years gone by, tidied them up a bit
  (ok, a lot!) and combined them into this book.

Hopefully I have also fixed all the original errors and included all
  the good advice from my loyal reader(s), especially George Gwilt who seems
  to read everything I write and then comes up with better ways of doing
  things. For this, I am eternally grateful -{} thanks George. I might have
  fixed the spelling mistakes and bad grammer as well -{} time will tell.

\section{Conventions}
\label{preface-conventions}\hyperlabel{preface-conventions}%

The following conventions are used in this book:

If I am presenting code to you then it will appear in the text as
    follows:

\begin{lstlisting}[firstnumber=1,]
return_d7   move.l  bv_rip(a6),a1       ; Because we had no parameters passed
            moveq   #2,d1               ; Size of stack space required
            move.w  bv_chrix,a2         ; Routine to allocate maths stack space
            jsr     (a2)                ; Go get some space NO ERRORS OCCUR!
            move.l  bv_rip(a6),a1       ; New top of stack
            subq.l  #2,a1               ; Make space for our integer result
            move.w  d7,0(a6,a1.l)       ; Stack the result
            move.w  #3,d4               ; Signal word result on stack
            move.l  a1,bv_rip(a6)       ; Store new top of stack for SuperBasic
            clr.l   d0                  ; No errors
            rts                         ; Return result to SuperBasic
\end{lstlisting}

Items of special note or tips will appear as follows:
\begin{DBKadmonition}{warning}{Warning}

This is a \textbf{warning}. These will
      advise to of areas where you need to be careful not to cause serious
      problems.
\end{DBKadmonition}
\begin{DBKadmonition}{warning}{Important}

This is \textbf{important}. These will
      advise to of areas where you need to be careful not to cause
 \emph{really} serious problems.
\end{DBKadmonition}
\begin{DBKadmonition}{}{Note}

This is a \textbf{note}. Notes point out
      something that might not be too obvious, or that I think you should know
      about.
\end{DBKadmonition}

\section{Getting Help}
\label{preface-getting-help}\hyperlabel{preface-getting-help}%

There are lots of places you can get help when struggling with
    Assembly Language:
\begin{itemize}[itemsep=0pt]

\item{}QL Today magazine at http://www.qltoday.com.


\item{}The ql-{}users mailing list, subscribe at
        http://lists.q-{}v-{}d.com/listinfo.cgi/ql-{}users-{}q-{}v-{}d.com.


\item{}The qdosmsq forums at
        http://qdosmsq.dunbar-{}it.co.uk/forum.


\item{}Online QDOS/SMSQ documentation at
        http://qdosmsq.dunbar-{}it.co.uk.


\item{}QDOS/SMSQ Documentation from Jochen Merz.
        http://smsq.j-{}m-{}s.com/.

\end{itemize}
