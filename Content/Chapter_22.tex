\chapter{WMAN, The Window Manager}

\section{Introduction}
\label{ch22-intro}%\hyperlabel{ch22-intro}%

At the end of the last chapter I mentioned that we
    would be delving into the WMAN\program{WMAN} system next. Well, here we are. However,
    before we get down and dirty in the code, I need to make sure you all know
    what I'm talking about, so let's start with a brief introduction/reminder
    to WMAN\program{WMAN} and all its constituent parts.

\section{WMAN}
\label{ch22-wman}%\hyperlabel{ch22-wman}%

Until now, we have been playing with the PE or Pointer Environment
    routines. These allow for a window to be outlined, the pointer to be drawn
    and read and so on. However, to use these few routines to write
    applications with multiple windows and so on, loose items, menus whatever,
    would be quite difficult. This isn't to say that it cannot be done, it's
    just difficult.

What we really need is a utility to allow us the ability to define
    our window structure, the loose items and so on contained within it and
    convert that into what QDOSMSQ really needs to have to be able to give us
    all the goodies we get from the PE, well, WMAN\program{WMAN} is just that.

Using WMAN\program{WMAN} we can define a window and all its contents, then use
    the vectors from WMAN\program{WMAN} to setup, display, remove and interact with our
    application without having to write code to handle everything
    ourselves.

George Gwilt mentioned in a comment about part 20 of this series
    that I treated the call to \pe{IOP\_PINF} as a method of finding out whether or
    not the Pointer Environment had been loaded. While it does indeed do this,
    it also returns a vector to the current location of the WMAN\program{WMAN} utilities in
    memory in A1.L -{} and it is these vectors we will be exploring in the
    coming articles.

\section{A Very Brief Overview Of WMAN}
\label{ch22-overview}%\hyperlabel{ch22-overview}%

Before we go on, we need to know what all the bits of the PE
    actually are, so there now follows a small briefing on that very subject.
    I won't be spending a lot of time in the discussions so if you need
    further information there is a very good ``Idiot's Guide To The PE''
    available on Dilwyn's web site at
    http://www.dilwyn.uk6.net/pe/peig/pe.html if you want to read it online or
    http://www.dilwyn.uk6.net/pe/peig.zip if you want to download it to read
    at your leisure.

\subsection{Selection Keys}
\label{ch22-sel-keys}%\hyperlabel{ch22-sel-keys}%

A selection key is simply the key that you press -{} when the
      pointer is over the appropriate primary window (see below) -{} to activate
      some function or feature of the program in question. It may cause an
      action to be carried out or simply highlight an option is a menu.
      Normally, the selection key is shown underlined, but this is not
      necessary, although it is more helpful to the user of the program if it
      is.

\subsection{Hit and Do}
\label{ch22-hit-do}%\hyperlabel{ch22-hit-do}%

When the mouse buttons are in use then a HIT is what happens when
      you click with the left mouse button and a DO is when you click with the
      right one. On the keyboard, a HIT is when you tap the spacebar and a DO
      is when you tap ENTER. The actions carried out when you HIT or DO may be
      the same or may be different -{} it's all down to how the programmers
      wrote the code.

\subsection{Outline or Primary Window}
\label{ch22-outln}%\hyperlabel{ch22-outln}%

I have mentioned outlines before, however, for the same of
      completeness, I'm reiterating here. The outline (or primary window) is
      the rectangle of your screen that the program will perform all its
      workings within. Any secondary windows (see below) opened by the program
      must be fully contained within the area bounded by the outline.

Of course, some programs allow you to move their windows around
      the screen. This also moves the outline around and wherever the window
      ends up when the user has moved it, becomes the new outlined area and
      all secondary windows will now appear within the new location.

The biggest size that an outline can be is the maximum width and
      height of the screen minus the shadow width and depth.

\subsection{Secondary Windows}
\label{ch22-sec-win}%\hyperlabel{ch22-sec-win}%

Secondary Windows are things like QMENU's file open utilities
      and so on, pop-{}up messages giving you error messages and anything else
      that takes place within the outline or primary window.

\subsection{Information Sub Windows}
\label{ch22-sub-win}%\hyperlabel{ch22-sub-win}%

These are small areas of the primary or secondary windows that
      show static text or little images or whatever. The most commonly seen
      and recognisable ones are those green and white stippled `caption bars'
      that most PE programs have at the top of every window.

Indeed, the caption bar for most PE programs that I know of is set
      up with a green and white stippled information window all the way across
      the top of the window, then on top of that there is another plain white
      information window nicely centralised horizontally on top of the first
      one. The program name or caption is then inserted as an Information
      Objects (see below) into this second information window.

\subsection{Information Objects}
\label{ch22-info-obj}%\hyperlabel{ch22-info-obj}%

Once an information sub window has been created you need something
      to put in it -{} for information purposes. To this end you need to create
      information objects. These can be text or blobs, sprites or patterns
      (see below). The most noticeable ones are the program name shown in the
      `caption bar' of most PE programs.

\subsection{Loose Items}
\label{ch22-loose}%\hyperlabel{ch22-loose}%

Loose items are small `buttons' with text or graphics on them.
      They usually have a border that magically appears when the pointer is
      within the bounds of the loose item in question. A hit or do on a loose
      item will cause some action to be carried out.

The popular loose items known to most users would probably be the
      ZZz, ESC, resize and move ones that appear in the caption bars of may
      PE programs.

\subsection{Application Sub Windows}
\label{ch22-app-sub-win}%\hyperlabel{ch22-app-sub-win}%

There's not much to say about the application sub windows really.
      They are what's left of the primary or secondary window after borders,
      information sub windows and loose items etc have been removed. They are
      the areas of the screen that the program prints its output or allows
      input from the user and so on.

A graphics drawing program, for example, would use the application
      sub window to allow the user to draw whatever it is that they are
      drawing.

\subsection{Pan and Scroll Bars}
\label{ch22-pan-scroll}%\hyperlabel{ch22-pan-scroll}%

These are displayed if the data in an application sub window is
      too wide (pan) or too tall( scroll) to be displayed completely within
      the area of the screen set aside for the application sub window. GUI
      users on other system (Linux or Windows) will be familiar with the
      concept.

At first, these can be a nightmare as a `DO' within the scroll bar
      (or pan bar) will split it and you then end up with two separately
      scrollable (and/or pannable) windows within the application window.
      Could be useful at times I suppose!

\subsection{Sprites, Blobs and Patterns}
\label{ch22-sprites}%\hyperlabel{ch22-sprites}%

A SPRITE is a picture that appears on the display somewhere. A
      pointer is just a sprite that is moved around the screen. Sprites may be
      drawn to look like text, for example, in logos and programmer's names
      etc, or they may be small pictures to represent some function of the
      program.

A BLOB is part of a sprite and holds only data that defines the
      shape. It has no colour information at all. The PATTERN is the part of
      the sprite that holds the colour data. Why separate them like this? I
      suspect it was to save memory -{} why bother having sprites defined with
      the same shape, just different colours -{} by defining the BLOB once and
      the PATTERS for the colours, you save repeating the blob data -{}
      perhaps?

Blobs and patters can be used independently of sprites
      though.

\subsection{Border}
\label{ch22-border}%\hyperlabel{ch22-border}%

The border around the primary and secondary windows, and indeed
      any other object, is optional and up to the programmer. However, most
      programs use borders.

When you move the pointer over a loose item, a border may appear
      around it to indicate that you can carry out some form of action if you
      were to hit or do the loose item in question. Once the pointer is
      outside the loose item boundary, the border may vanish.

\subsection{Shadow}
\label{ch22-shadow}%\hyperlabel{ch22-shadow}%

The shadow for a window is drawn down the right side and along the
      bottom. It is optional and entirely at the discretion of the developer.
      When in use, a shadow gives the impression that the window is hovering
      above the desktop. The shadow is outside the outline and does not
      register hit or do actions. It is purely decorative.

\section{More Useful Utilities From George}
\label{ch22-utilities}%\hyperlabel{ch22-utilities}%

The GWASL\program{Gwasl} assembler that George Gwilt wrote has been used as the
    assembler of choice throughout this long running series. George has come up
    trumps again with another utility that allows the easy generation of
    assembler code that defines a WMAN\program{WMAN} windows definition (more on this later)
    and I've been testing it out. Unfortunately, my holiday got in the way and
    I have a new version of the utility and GWASL\program{Gwasl} to test out at the
    moment.

I'm sure that these programs will soon be available from George's
    usual repository of fine code. In addition, I shall be trying these
    utilities out myself and reporting back.

\section{WMAN Windows Definition.}
\label{ch22-wdef-intro}%\hyperlabel{ch22-wdef-intro}%

As mentioned above, WMAN\program{WMAN} is slightly more involved that the bare
    bones PE in as much as it carries out a huge amount of work on your
    behalf. This is all work that you would have to write into each and every
    program you write using the WMAN\program{WMAN} system (here-{}after known collectively
    as the PE or the Pointer Environment) but in order to take advantage of
    all this hard work, you have to set things up in a standard manner.

If you look back an issue or so, you will notice that up until now,
    all my PE test programs simply opened a console and set an outline before
    entering the main loop to read the pointer, act upon it, repeat as
    necessary. Obviously, my test programs were small and insignificant -{} but
    even though, they could benefit from a bit more added `sparkle'. The WMAN\program{WMAN}
    routines make this possible.

The first thing we have to do is create a definition of our window
    in memory. This will be in a standard format and when done, we call a WMAN\program{WMAN}
    routine (\pe{WM\_SETUP}) to initialise the various internals required to make
    our window work under WMAN\program{WMAN}. Let's now take a look at the standard
    definition as required by WMAN\program{WMAN}.

\section{Standard Windows Definition}
\label{ch22-wdef}%\hyperlabel{ch22-wdef}%

So, now you know what all the bits in a window are, we can get right
    in and start discussing the standard way we have to define a windows and
    all its \emph{decorations}. Let's take a look at one that
 \emph{someone else} prepared earlier.

The following is extracted from a small utility written by \emph{Oliver
    Fink} many years ago. The utility shows various bits of information about
    the running QDOSMSQ system. I have modified the original in a few places
    but the full credit must remain with Oliver. The code is in the public
    domain.

The start of the definition is the main window itself:

\begin{lstlisting}[firstnumber=1,caption={Main Window - Fixed Part}]
;Main window definition :
           dc.w  160                  ; default window width
           dc.w  84                   ; height
           dc.w  146                  ; initial pointer x position 
           dc.w  8                    ; y position
\end{lstlisting}

So far so simple, nothing much here that we haven't met already. All
    we are doing here is telling WMAN\program{WMAN} how big our window is to be and where
    within the window the pointer is to be positioned when the window is first
    drawn.

The above positioning of the pointer is relative to the window
    outline. So in our window which is 160 pixels wide, the pointer is located
    146 pixels along -{} nearly at the far right end. It is located 8 pixels
    down from the top. When drawn on screen, this places the pointer directly
    over the ESC loose item.

When the program is first executed, the PE attempts to position the
    main window on the screen so that the requested position of the pointer is
    superimposed on the current pointer position on screen. This prevents
    disconcerting jumps of the pointer every time you start up a new
    program.

You can see this in action if you move the mouse around on screen,
    note where it ends up, then EX a new program that uses the PE. You will
    see that the main window appears wrapped around where you last saw the
    pointer.

Next we define the attributes for our window.

\begin{lstlisting}[firstnumber=last,caption={Main Window - Window Attributes}]
           dc.b  $00               ; MSbit clear to call CLS
;                                  ; LSbit clear allows cursor keys
;                                  ; to move pointer.
           dc.b  2                 ; shadow depth 
           dc.w  1                 ; border width 
           dc.w  0                 ; border colour (black)
           dc.w  7                 ; paper colour (white)
\end{lstlisting}

Again, there's nothing remarkably difficult here. Bit 7 of the first
    byte tells WMAN\program{WMAN} whether or not the window is to be cleared. Setting bit 7
    says that the window \emph{must not} be cleared. Following
    on, we define a shadow size for the bottom and right edges of our window.
    Bit 0 of this byte enables or disables the ability to move the pointer
    using the cursor keys.

\begin{note}
The ability to disable/enable cursor key pointer movement is only
      mentioned, briefly, in the QPTR\program{QPTR Toolkit} manual under the section on the Working
      Definition.

Disabling the cursor key movement \emph{also}       disables the ability to select items in an application sub-{}window menu
      using the space bar or enter keys. This \emph{could} be a
      bug!
\end{note}

Remembering back to our initial forays into the raw Pointer
    Environment, you may remember that we could have a different shadow depth
    on both of those sides, using WMAN\program{WMAN}, it appears that the shadow must be the
    same down each side. Oh well!

\begin{note}
The documentation says that \emph{for sub-{}windows the shadow
      depth should be zero.} Best we stick to that advice. Remember,
      a sub-{}window is one `embedded' within the main window. See application
      sub-{}windows or information sub-{}windows above.
\end{note}

Next, and finally for the main window, we have the definition of
    where the default pointer sprite for the window is to be found.

\begin{lstlisting}[firstnumber=last,caption={Main Window - Default Pointer}]
           dc.w  0                 ; use default pointer
\end{lstlisting}

This is one of my changes. In a need to reduce the amount of code in
    the magazine and also, to reduce your typing, I've modified Oliver's
    definition to use the default arrow pointer in the main window and in the
    application sub-{}window which will be defined below. Oliver had a custom
    sprite for the main window and another for the application sub-{}window.
    Both have been removed. The original file had this definition (don't type
    this in!) and a chunk of code to define the sprite to be used.

\begin{lstlisting}[firstnumber=1,caption={Do Not Type This In!}]
           ; DO NOT TYPE THIS IN!
           dc.w  sprt-*            ; pointer to pointer sprite
\end{lstlisting}

You should be aware that all pointers in a window definition are
 \emph{word} sized and \emph{relative} to their
    own position in the definition block.

Now, that implies that all object lists must be within plus or minus
    16KB of the pointer position, which might be a problem when there are a
    lot of objects and so on to define. To this end, if bit zero is set -{} an
    odd address -{} then that offset is used as a pointer to a long word which
    itself is a relative pointer to the object in question. Obviously, the
    word length odd pointer obviously has to be made even first, this is done
    simply by clearing bit zero.

In the above, if the Pointer Sprite above was defined a long long
    way away, we would see something like this:

\begin{lstlisting}[firstnumber=1,caption={Do Not Type This In Either!}]
...
           dc.w  sprt-*+1          ; ODD Pointer to long pointer to
;                                  ; our window's pointer sprite
           ...
sprt       dc.l  Real_sprt-*       ; Long pointer to pointer sprite
           ...
Real_sprt  ....                    ; Pointer sprite definition
\end{lstlisting}

If any pointer is to something we don't need, then simply set it to
    zero. So, for example, instead of using Oliver's original `?' sprite
    (shaped like a question mark) we have defined this word pointer as zero
    and get the default arrow sprite instead. In this case, zero means `use
    the default' but in other places, it means `not used'.

Next to be defined are the attributes for all the loose items we
    will be using in the window. Starting with the easy bits:

\begin{lstlisting}[firstnumber=14,caption={Main Window - Current Loose Item - Border Attributes} ]
; menu item attributes
           dc.w  1                 ; Current item border width  
           dc.w  0                 ; Border colour (black)
\end{lstlisting}

As before, it is simple. We define a black border 1 pixel in width.
    When the pointer is over any of our loose items, this border will be drawn
    around it to indicate that `you can do something here'.

Following the border, we have the attributes for the loose items
    that are unavailable, available and selected. These attributes require 8
    bytes each and define paper and ink (for text objects contained within
    them) and also blobs and patterns for the other object types. We are only
    using the paper and ink attributes, but the others must be there. We use
    zero to indicate `not in use'.

\begin{lstlisting}[firstnumber=last,caption={Main Window - Loose Item Attributes}]
; Menu item unavailable
           dc.w  30                ; Paper - green/white stipple
           dc.w  30                ; Ink - green/white stipple 
           dc.w  0                 ; Pointer to blob for pattern
           dc.w  0                 ; Pointer to pattern for blob

; Menu item available
           dc.w  7                 ; Paper - white
           dc.w  0                 ; Ink - black
           dc.w  0                 ; Pointer to blob for pattern
           dc.w  0                 ; Pointer to pattern for blob

; Menu item selected
           dc.w  4                 ; Paper - green
           dc.w  0                 ; Ink - black
           dc.w  0                 ; Pointer to blob for pattern
           dc.w  0                 ; Pointer to pattern for blob
\end{lstlisting}

In this example program, the loose items are never anything except
    available (and very briefly, selected) so the unavailable attributes are
    never used, but they still have to be defined. Oliver has chosen to use a
    paper and ink colour of 30 for unavailable loose items. That value gives a
    pleasant green/white stipple. Don't worry about it because you will never
    see it!

Following the loose item attributes, we have a relative pointer to
    the help window. In this program, we are not using one, so that pointer
    gets the value of zero.

\begin{lstlisting}[firstnumber=last,caption={Main Window - Help Window Details}]
           dc.w  0                 ; Pointer to help window
\end{lstlisting}

That is the end of the fixed part of the window definition. So far
    so good, there has been nothing too difficult yet. I wonder what is
    coming?

The rest of the definition block defines the \emph{repeating
    parts} of the window definition. What exactly does that
    mean?

The documentation has this to say about the repeating parts:

\emph{``To allow for a variety of different layouts within the
    window as the size of the window varies, part of the window definition may
    be repeated several times. The definition should be made in order of
    decreasing window size. The last definition which defines the smallest
    allowable window, should be followed by a word containing -{}1. If the top
    nibble of a layout size word is zero, then the layout may not be scaled.
    If it is \%0100 then it may [be scaled].''}

\begin{note}
This actually shows up what I think is a contradiction in the
      documentation. There are other values that can be used in the top
      nibble, not only \%0000 and \%0100. More on scaling later in the
      series.
\end{note}

So there you have it. The fixed part of the window defines the
    default layout for the window. That layout and all other possible ones
    allowed, need to be defined in the repeating part of the window
    definition.

A window can be scaled by WMAN\program{WMAN} if the definition allows for it. The
    scaling flag is the top nibble (4 bits) of the size words for the window
    layout. If the top nibble is \%0000 then it cannot be scaled and if it is
    \%0100 then it may be scaled.

Scaling applies separately to the width and to the height of the
    different layouts. You don't have to scale vertically and horizontally,
    you can pick one, the other or both as desired.

For simplicity, and because I have not investigated scaling yet,
    Oliver and I will be sticking to non-{}scaled windows for now. Scaling will
    be a subject for a future article.

The following is the repeating parts for our single, non-{}scaling
    layout in our small program.

\begin{lstlisting}[firstnumber=last,caption={Main Window - Repeating Part}]
; Base of repeated part of window definition 
           dc.w  160               ; Width for this layout
           dc.w  84                ; Height for this layout

; Pointers to definition lists
           dc.w  il-*              ; Information sub-windows 
           dc.w  ll-*              ; Loose menu items
           dc.w  al-*              ; Application sub-windows
\end{lstlisting}

The above would be repeated for each and every different allowable
    layout for the window, from the biggest to the smallest. Following on from the very last layout, the smallest allowed, we have the terminating word.

\begin{lstlisting}[firstnumber=last,caption={Main Window - Repeating Part - End Flag}]
           dc.w  -1                ; End flag
\end{lstlisting}

So we allow one and only one layout, which just happens to be
    exactly the same size as the default one defined in the fixed part of the
    definition block. It has three pointers at the end for the information
    sub-{}windows, the loose items and any application sub-{}windows that are
    required in this layout. Each layout will have its own list and they need
    not be the same for each different layout.

We shall pause for breath at this point and discuss these lists in
    the next article. Hopefully, the above was not too taxing and I've
    explained it better that I ever had it explained to me!

\section{Coming Up...}
\label{ch22-the-end}%\hyperlabel{ch22-the-end}%

The next chapter continues our look at the window definition by looking
    into the lists of objects attached to our window.

