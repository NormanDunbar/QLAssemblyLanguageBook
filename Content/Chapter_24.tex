\chapter{Creating Your Own Windows With SETW}
\program{SETW}

\label{ch24}%hyperlabel{ch24}%

\section{Introduction}
\label{ch24-intro}%hyperlabel{ch24-intro}%

In this chapter, I shall be taking a small diversion into one
        of George Gwilt's utility programs. This one, SETW\program{SETW}, allows you to interactively
        create windows for your applications. SETW\program{SETW} then goes away and does all the hard
        work of setting everything up.

\section{Downloading SETW}
\label{ch24-setw}%hyperlabel{ch24-setw}%

SETW\program{SETW}, and other useful utilities, are available from George's web site, \url{http://gwiltprogs.info/} and from there I advise you to download the following three utilities:
\begin{itemize}[itemsep=0pt]

\item{}SETW -{} setwp05.zip


\item{}EasyPEasy -{} peassp02.zip


\item{}GWASL -{} gwaslp07.zip

\end{itemize}

The latest version of GWASL\program{Gwasl} is required to enable you to assemble PE
        programs created using SETW\program{SETW} and using the EasyPEasy\program{EasyPEasy} library files in peassp02.zip.
        As you will require these for the remainder of the tutorial then you should
        download them all now to save time later.

There are other files there similarly names but with a `p' replaced by and
        `s' -{} these are the sources for the utilities and while educational, you don't
        need them.

The files are zipped up using the QDOS version of zip, so copy them from
        wherever you downloaded them to into your QL system (QPC etc) and unzip them using
        the QDOS version of unzip. There is one supplied with the C68\program{C68} system and that
        works fine.

\section{Running SETW}

In order to create correctly written assembly source for GWASL\program{Gwasl}, we need to
        pass a single parameter to SETW\program{SETW} when we execute it. The parameter is "-{}abin" with
        no spaces. This tells SETW\program{SETW} that the code produced will be used to build a binary
        file rather than a relocatable one which will be subsequently linked with other
        relocatable files to produce the final binary.

We GWASL\program{Gwasl} users don't have a linker so all our programs need to be self
        contained, or may include pre-{}assembled modules and libraries using the LIB
        command.

\begin{lstlisting}[firstnumber=1,caption={Executing SETW},language={}]
EX SETW ; '-abin'
\end{lstlisting}

The command above is all we need. If you do not have Toolkit 2, then the
        EXECUTE command from Turbo Toolkit can be used instead.

We will use SETW\program{SETW} to create a file that we will use later on. It will be a
        very simple window with a single information window near the top and a single text
        object within the information window. Feel free to follow along on your own QL
        system as we go.

The program starts by opening a window as big as it can on your screen, it
        displays a few bits of information and prompts for the root name of the various
        files to be created.

For our example, we simply set the name to `hello' -{} without the quotes.
        Type it in and press ENTER.

SETW\program{SETW} will create three files when we are done. They will be created on ram1\_
        (in my case) or wherever you have configured SETW\program{SETW} to put them by default. The
        three files created will be:
\begin{itemize}[itemsep=0pt]

\item{}Ram1\_hello\_wda -{} a file for use by George's TurboPTR\program{TurboPTR} utility. It is
                of no use to us and can be safely deleted when finished.


\item{}Ram1\_hello\_asm -{} a file for use by an assembler, in our case, GWASL\program{Gwasl},
                this is the file we will need.


\item{}Ram1\_hello\_z -{} a file for use with another of George's utilities,
                CPTR\program{Cptr}, a program to help C68\program{C68} users write PE programs. Again, we don't need
                this file and it can safely be deleted.

\end{itemize}

\subsection{Entering Text Objects}

The next screen that appears is titled `ALTER TEXT' and is where we
            enter every text object to be used in our finished utility. We must be very
            careful here and not forget any because SETW\program{SETW} creates code for what we enter
            and we cannot go back and add another if we forget one. (Well possibly we can
            in the generated assembler file, but I have not confirmed this yet.)

To enter your text objects, press the `N' key to create a new text
            object and simply type in the required text. For our example window all we
            need is one single object containing the text `Hello World' (without quotes) -{}
            for the main reason that this is how everyone starts to learn a new language!
            Press ENTER when you have entered the text.

In slightly more complicated programs, there would be a lot more text
            objects to enter, but for now, press the ESC key to exit from the ALTER TEXT
            screen.

\subsection{Entering Sprites, Blobs \& Patterns}

We don't need any sprites, blobs or patterns in this example, so simply
            press the ESC key when prompted for each of these.

\subsection{The Main Window}

The next prompt is to tell SETW\program{SETW} about the main window, how many windows
            are needed and so on. In many cases the default is correct and all we need do
            is press ENTER at each prompt -{} however, make sure you read the prompt and
            think before pressing ENTER -{} once you have done so, there's no going back!
            (Ask me how I know!)

When asked for the number of main windows, accept the default of 1 by
            pressing ENTER.

When asked for the number of loose Items, accept the default of zero by
            pressing ENTER.

When asked for the number of Information Windows, we will need one, so
            press the `1' key and press ENTER.

We are now asked to enter the number of information objects in each
            information window. We require one information object in our one single
            information window. Type `1' and press ENTER.

When asked how many Applications Windows you want, accept the default of
            zero.

Next we are asked to select a shadow size. I find a size of 2 to be
            adequate. Type `2' and press ENTER.

For the border size choose a width of 1.

For all the prompts asking us to select a colour, select option 1 each
            time. Use the arrow keys to highlight the desired option and press ENTER to
            select it. We want "1. Default" for our colours. (I will explain the others
            later on in the series.)

Next we get to choose the sprite to be used as a pointer in the main
            window. I much prefer the standard arrow, so select it as above, and press
            ENTER. If you wish, you can choose another sprite to use as the pointer
            instead -{} it's your choice.

\subsection{Information Windows \& Objects}

Now that all the details for the main window have been entered, or
            default chosen, we get to enter the requirements for each (or in our case,
            one!) Information Window.

First of all we need to enter the border width, I use a width of one
            pixel for all my programs. Type `1' and press ENTER.

Next we need the border colour, as before, select "1. Default" and press
            ENTER.

Select the default again for the paper colour.

We are now asked to select a type for our information objects for this
            information Window. As we only entered a single information object way back at
            the beginning and that was a text object, we should select "text" and press
            ENTER. Other object types would be available if we had entered any sprites,
            blobs or patterns.

Next we see a window appear with the list of (one!) text objects. As
            there is only one, it has been highlighted for us. Press ENTER to select
            it.

Select the default colour again.

When asked for the character sizes for X and Y for this text object,
            select zero for both.

\subsection{Interactively Sizing The Window \& Contents}

Now the fun begins! A window appears that allows us to interactively
            resize the main window and the information window we have created. Once done,
            we can position these items almost at will.

Looking at the window currently being displayed, the lower right corner
            shows the currently defined dimensions for the main window itself. At the top
            left is an outline of the noted dimensions. We can use the arrow keys to
            change the dimensions -{} up makes the window less tall, down makes it taller,
            left makes the window narrower and right makes it wider.

Pressing the ALT key makes the change in size bigger. This saves wearing
            out your keyboard getting the window to the size you would like!

For this demonstration program, we require a window size of 200 wide by
            100 deep. Use the ALT and arrow keys to make the dimensions 200 wide and 100
            deep. When the desired dimensions have been achieved, press ENTER.

We are now asked if a variable window is to be created, this will be
            covered in a future tutorial so for now, type `N'. (There is no need to press
            ENTER.)

Next we need to set the origin of the window. Again the arrow keys move
            things around and the ALT key makes the movements bigger. For the
            demonstration, set the origin to 50, 50. You will get a rough idea of where
            the origin will be as a small dot moves around the screen under the control of
            your arrow keys. Press ENTER when the origin is where you would like it to
            be.

Next up, we get to size our information windows, or window in our case!
            I have decided to make it slightly wider than the space required for the text
            object. That itself is 12 characters of 6 pixels wide or 72 pixels in total.
            As I like to have a bit of leading and trailing space in my information
            windows, use the arrow keys and ALT, as before, to resize information window
            number one to be 74 pixels wide by 12 deep. You may choose a different
            dimension if you like, but it will need to be a minimum of 72 pixels wide to
            hold all the text.

The program starts off in position mode rather than in size mode. You
            may need to press F2 to toggle between the two modes. Check the prompt on
            screen for advice about which mode you are currently in.

Once you have the desired size, press F2 and move the window to a
            position of 62 across by 2 down. If the information window size plus the
            position causes it to extend off the edge(s) of the main window, you will not
            be allowed to position it where you want to. In this case, toggle between size
            and position with the F2 key until you have it correctly sized and
            positioned.

Press ENTER when done. This takes you now to the sizing and positioning
            of the information window object (where the actual text object will be
            placed). If you remember the text object is 12 characters or 72 pixels wide
            which means that we need an object big enough to take that plus a little space
            at the beginning and end. As I like a couple of pixels either end of my
            objects, set the information object to be at position 4 across and 1 down.
            Press ENTER when satisfied.

That's it for our little test window. SETW\program{SETW} now displays some information
            about the files it created and after a pause, or when you press a key, it will
            cycle through all the main windows we created -{} one in our case -{} and display
            them on screen as they have been defined.

At this point, there's not much we can do if it all went horribly wrong.
            We simply have to start again -{} or get down and dirty in the generated
            assembly file! Press ENTER to exit from SETW.

On ram1\_, in my case, we now have the files that SETW\program{SETW} generated for us.
            We are only interested in the hello\_asm file and can happily delete the
            others. Feel free to examine the generated file in an editor and compare what
            has been created with the previous articles where I explain what the
            individual bits of a WMAN\program{WMAN} window definition are.

The assembly source file generated is not able to be assembled as it is
            and then run, it has no code in it to make it a correctly functioning QDOS
            job. That comes later.

Until next time, feel free to generate more windows of your own and get
            to know George's SETW\program{SETW} utility -{} we will be using it in future articles in this
            series.

\section{Coming Up...}
\label{ch24-the-end}%hyperlabel{ch24-the-end}%

We take the file we created with SETW\program{SETW} and feed it into
        another of George's utilities, EasyPEasy, in the next chapter. EasyPeasy tries to make coding for the PE
        much easier. Until next time, happy windowing.

